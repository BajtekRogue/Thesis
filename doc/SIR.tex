\section{Dwa wierzchołki, jedna krawędź}

W celu oswojenia się z bardziej skomplikowanym modelem, jakim jest SIR, przeanalizujmy graf o jednej krawędzi. 
Niech $V=\{u,v\}$ oraz niech $u$ będzie wierzchołkiem startowym. 
Oczywiście $X_u=0$. 
Pierwsza runda jest identyczna jak w modelu SI, a więc $\mathbb{P}[X_v=1]=p$. 
Jeżeli węzeł $v$ nie zostanie poinformowany w rundzie pierwszej, a propagacja nie wygaśnie, to sytuacja się powtórzy. 
Zachodzi to z prawdopodobieństwem $q\beta$. Aby $X_v=t$, potrzebujemy, by ten cykl nastąpił $t-1$ razy. 
Widzimy zatem, że $X_v$ ma rozkład umierający geometryczny:
\[
    X_v\sim \mathrm{KGeo}(p,\alpha), \quad \mathbb{E}[X_v|X_v<\infty]=\frac{1}{1-q\beta}.
\]
Z założeń modelu $p,\alpha\in(0;1)$, a zatem $\mathbb{P}[X_v=\infty]\ne 0$. 
Jest cecha kontrastująca SIR względem SI.

Przyjrzyjmy się teraz zmiennej $Y_t$.
Zauważmy, że $\mathbb{P}[Y_t=2]=\mathbb{P}[X_v\le t]$ a więc
\[
    \mathbb{P}[Y_t=2] = p\frac{1-q^t\beta^t}{1-q\beta}, \quad \mathbb{P}[Y_t=1]=1-p\frac{1-q^t\beta^t}{1-q\beta}.
\]
Stąd
\[
    \mathbb{E}[Y_t] = 1 + p\frac{1-q^t\beta^t}{1-q\beta}.
\]
Przechodząc w granicę $t\to\infty$ dostajemy
\[
    \mathbb{P}[W=1]=1-\frac{p}{1-q\beta}, \quad \mathbb{P}[W=2] = \frac{p}{1-q\beta}.
\]
Co oczywiście daje nam natychmiast
\[
    \mathbb{E}[W] = 1 + \frac{p}{1-q\beta}.
\]

Jeśli chodzi o zmienną $Z$ to propagacja zakończy się gdy wierzchołek $v$ wyzdrowieje lub $u$ zostanie poinformowany.
Dzieje się to z prawdopodobieństwem $1-q\beta$ a więc 
\[
    Z\sim\mathrm{Geo}(1-q\beta) \quad \mathbb{E}[Z]=\frac{1}{1-q\beta}.
\]


\section{Pewne wygaśnięcie}

W modelu SIR wygaśnięcie infekcji jest zdarzeniem pewnym. 
Sformalizujmy ten fakt prostym twierdzeniem.

\begin{theorem}\label{theorem:infection_dies_out_SIR}
Niech $G=(V,E)$ będzie grafem spójnym, $s\in V$ źródłem infekcji oraz $v\in V\setminus\{s\}$.
Wtedy zachodzą następujące tożsamości:
\[
    \mathbb{P}[X_v = \infty] > 0, \quad \mathbb{P}[Z < \infty] = 1.
\]
Ponadto dla $k\in\mathbb{N}$ zachodzi 
\[
    \mathbb{P}[W=k]=\lim_{t\to\infty}\mathbb{P}[Y_t=k], \quad \mathbb{E}[W]=\lim_{t\to\infty}\mathbb{E}[Y_t].
\]
\end{theorem}

\begin{proof}
Ustalmy $v\in V\setminus\{s\}$. 
Jeśli po pierwszej rundzie źródło wyzdrowieje i nie przekaże dalej infekcji to $X_v=\infty$.
Zatem $\mathbb{P}[X_v=\infty]\ge \alpha q > 0$.
Jeśli rozważamy propagacje jako łańcuch Markova to prawdopodobieństwo przejścia do stanu absorbującego, czyli $\mathcal{I}_t=\varnothing$, w czasie $t$ wynosi
\[
    \alpha^{|\mathcal{I}_t|}\prod_{u\in \mathcal{I}_t} q^{|\mathrm{N}(u)\cap\mathcal{S}_t|}.
\]
Istotne jest to, że jest to dodatnia liczba. 
A więc mamy $\mathbb{P}[Z<\infty]=1$.
Zauważmy, że 
\[
    \lim_{t\to\infty} \mathbf{1}_{v\in\mathcal{I}_t\cup\mathcal{R}_t} \xrightarrow{\text{a.s}} \mathbf{1}_{X_v<\infty}.
\]
Ponadto $Y_t\le Y_{t+1}$ oraz $Y_t\le |V|$.
A więc z Twierdzenia Lebesgue dostajemy wzory na rozkład i wartość oczekiwaną zmiennej $W$.
\end{proof}


\section{Grafy ścieżkowe}

Przyjrzyjmy się teraz grafom $\mathrm{P}_n$.
Oczywiście za źródło propagacji wybieramy wierzchołek $s=1$.
Podobnie jak w modelu SI propagacja rozchodzi się miedzy kolejnymi wierzchołkami niezależnie a więc
\[
    X_1=0, \quad X_v=X_{v-1}+U_v, \quad v\in\{2,3,\dots,n\},
\]
gdzie jednak $U_2,U_3,\dots,U_n\sim\mathrm{KGeo}(p,\alpha)$ oraz są niezależne.
Stąd $X_v=U_2+U_3+\cdots+U_v$ a zatem $X_v$ ma rozkład umierający ujemny dwumianowy:
\[
    X_v\sim\mathrm{KNegBin}(v-1,p,\alpha), \quad \mathbb{E}[X_v|X_v<\infty] = \frac{v-1}{1-q\beta}.
\]

Jeśli chodzi o zmienną $Y_t$ to zauważmy, że
\[
    Y_t=\max\{j\in\{1,2,\dots,n\}: X_j\le t\}.
\]
A więc dla $k\in\{1,2,\dots,n\}$ mamy
\[
    \mathbb{P}[Y_t\ge k+1] = \mathbb{P}[X_{k+1}\le t] = \sum_{j=k}^{t} \binom{j-1}{k-1}p^k {(q\beta)}^{j-k}.
\]
Dalej mamy
\[
    \mathbb{E}[Y_t] = \sum_{k=0}^{n-1} \mathbb{P}[Y_t\ge k+1] = 1+\sum_{k=1}^{n-1}\sum_{j=k}^{t} \binom{j-1}{k-1}p^k {(q\beta)}^{j-k}.
\]
Nie ma zbytniej nadziei na zwartą formę tej sumy.
Przejdźmy teraz do zmiennej $W$.
\[
    \mathbb{P}[W\ge k+1] = \lim_{t\to\infty} \sum_{j=k}^{t} \binom{j-1}{k-1}p^k {(q\beta)}^{j-k} = \Big{(\frac{p}{1-q\beta}\Big)}^k.
\]
Oznaczmy $\theta = \frac{p}{1-q\beta}$.
Wtedy
\[
    \mathbb{P}[W=k]=(1-\theta)\theta^{k-1}, \quad k\in\{1,2,\dots,n-1\}, \quad \mathbb{P}[W=n]=\theta^{n-1}.
\]
Ponadto
\[
    \mathbb{E}[W] = \sum_{k=0}^{n-1} \mathbb{P}[W\ge k+1] = \sum_{k=0}^{n-1} \theta^k = \frac{1-\theta^n}{1-\theta}.
\]

Skoncentrujmy naszą uwagę na zmiennej $Z$.
Niech $T_j$ oznacza czas przejścia ze stanu w którym wierzchołek $j$ został dopiero co zainfekowany do kolejnego stanu, gdzie $j\in V$.
Formalnie $T_j=\min\{t\in\mathbb{N}:i=X_j\land \neg (j\in\mathcal{I}_{t+i} \land j+1\in\mathcal{S}_{t+i})\}$.
Wtedy $T_j\sim \mathrm{Geo}(1-q\beta)$, a ponadto $T_1,T_2,\dots$ są niezależne.
Zauważmy, że jeśli $W<n$ to $Z=T_1+\cdots+T_W$ a gdy zaś $W=n$ to $Z=T_1+\cdots+T_{n-1}$.
Zatem $Z=T_1+T_2+\cdots+T_{Q}$ gdzie $Q=\min\{W,n-1\}$.
Będziemy potrzebować rozkładu i wartości oczekiwanej zmiennej $Q$.
Możemy zapisać $Q=W-\mathbf{1}_{W=n}$ a wtedy
\[
    \mathbb{E}[Q]=\mathbb{E}[W]-\mathbb{P}[W=n]=\frac{1-\theta^n}{1-\theta}-\theta^{n-1}=\frac{1-\theta^{n-1}}{1-\theta}.
\]
Dalej $\mathbb{P}[Q=k]=\mathbb{P}[W=k]=(1-\theta)\theta^{k-1}$ dla $k\in\{1,2,\dots,n-2\}$ oraz $\mathbb{P}[Q=n-1]=\mathbb{P}[W\ge n-1] = \theta^{n-2}$.
Zachodzi zatem $\mathbb{P}[Z=t|Q=m]=\mathbb{P}[T_1+\cdots+T_m=t]$ ale $T_1+\cdots+T_m\sim\mathrm{NegBin}(m,1-q\beta)$ (\cref{fact:sum_of_geo_RV}).
Daje nam to $\mathbb{P}[Z=t|Q=m]=\binom{t-1}{m-1} {(1-q\beta)}^m {(q\beta)}^{t-m}$.
Z prawdopodobieństwa całkowitego dostajemy
\[
    \mathbb{P}[Z=t]=\sum_{m=1}^{n-1} \binom{t-1}{m-1}{(1-q\beta)}^m {(q\beta)}^{t-m} \cdot \mathbb{P}[Q=m].
\]
Natomiast wartość oczekiwana zmiennej $Z$ wynosi
\[
    \mathbb{E}[Z]=\mathbb{E}[Q]\cdot\mathbb{E}[T_1]=\frac{1-\theta^{n-1}}{1-\theta} \cdot \frac{1}{1-q\beta} = \frac{1-\theta^{n-1}}{\frac{q\alpha}{1-q\beta}} \cdot \frac{1}{1-q\beta} = \frac{1-\theta^{n-1}}{q\alpha}.
\]


\section{Grafy gwiezdne}

Rozpatrzmy propagacje na rodzinie $\mathrm{S}_n$, z źródłem $0$.
Dla $v\in\{1,2,\dots,n\}$ mamy 
\[
    X_v\sim\mathrm{KGeo}(p,\alpha), \quad \mathbb{E}[X_v|X_v<\infty] = \frac{1}{1-q\beta}.
\]
Zauważmy natomiast, że zmienne $X_1,X_2,\dots,X_n$ sa zależne, bo gdy centralny wierzchołek wyzdrowieje żaden z liści nie może już zostać zainfekowany.

Przyjrzyjmy się teraz zmiennej $Y_t$. 
Połóżmy $C=\min\{\tau\in\mathbb{N}:0\in\mathcal{R}_\tau\}$.
Jeśli centrum zarazi się po $j$ rundach to liście gwiazdy mogą byc zarażane przez $\min\{j,t\}$ rund.
$Y_t$ zachowuje się tak samo jak w modelu SI dla grafów gwiezdnych.
Przypomnijmy, że rozkład ten wynosi $1+B_{j,t}$ gdzie $B_{j,t}\sim\mathrm{Bin}(n,1-q^{\min\{j,t\}})$.
A więc
\[
    \mathbb{P}[Y_t=k+1|C=j] = \binom{n}{k}{(1-q^{\min\{j,t\}})}^k{(q^{\min\{j,t\}})}^{n-k}.
\]
Ponadto $C\sim\mathrm{Geo}(\alpha)$ a więc $\mathbb{P}[C=j]=\alpha\beta^{j-1}$.
Z prawdopodobieństwa całkowitego dostajemy
\begin{equation*}
\begin{aligned}
\mathbb{P}[Y_t=k+1] &=\sum_{j=1}^\infty \mathbb{P}[Y_t=k+1|C=j]\cdot\mathbb{P}[C=j] = \\
&=\sum_{j=1}^\infty \binom{n}{k}{(1-q^{\min\{j,t\}})}^k{(q^{\min\{j,t\}})}^{n-k}\alpha\beta^{j-1} = \\
&= \frac{\alpha}{\beta}\binom{n}{k} \sum_{j=1}^\infty {(1-q^{\min\{j,t\}})}^k{(q^{\min\{j,t\}})}^{n-k}\beta^j.
\end{aligned}
\end{equation*}
Aby obliczyć wartość oczekiwana skorzystamy z \Cref{lemma:Formula_EYt}.
Mamy $\mathbb{P}[X_0\le t]=1$ a dla $v\ne 0$ mamy $\mathbb{P}[X_v\le t] = p\frac{1-{(q\beta)}^t}{1-q\beta}$.
A więc
\[
    \mathbb{E}[Y_t]=\sum_{v\in V}\mathbb{P}[X_v\le t] = 1 + np\frac{1-{(q\beta)}^t}{1-q\beta}.
\]
Licząc granice tych wyrażeń dla $t\to\infty$ dostajemy
\[
    \mathbb{P}[W=k+1] = \frac{\alpha}{\beta}\binom{n}{k} \sum_{j=1}^\infty {(1-q^j)}^k{(q^j)}^{n-k}\beta^j, \quad \mathbb{E}[W]=1+\frac{np}{1-q\beta}.
\]

Propagacja się zakończy gdy centrum wyzdrowieje lub wszystkie liście zostaną zarażone. 
Stąd mamy
\begin{equation*}
\begin{aligned}
\mathbb{P}[Z>t] &=\mathbb{P}[C>t]\cdot\mathbb{P}[Y_t<n+1|C>t]= \\
&=\beta^t(1-\mathbb{P}[Y_t=n+1|C>t]) = \beta^t(1-{(1-q^t)}^n).
\end{aligned}
\end{equation*}
Bezpośrednie liczenie wartości oczekiwanej zmiennej $Z$ nie da nam zwiezłej postaci.
Postarajmy się ją za to oszacować korzystając z \Cref{inequality:approximation_of_sum_by_an_integral}.
Oznaczmy $a=-\log(q),\; b=-\log(\beta)$ oraz $g(x)=e^{-bx}(1-{(1-e^{-ax})}^n)$.
Wtedy
\[
    \mathbb{E}[Z] \approx \int_{0}^{\infty} g(x) \; \mathrm{d}x.
\]
Dalej
\[
    \int_{0}^{\infty} e^{-bx}(1-{(1-e^{-ax})}^n)\; \mathrm{d}x = \int_{0}^{\infty} e^{-bx}\mathrm{d}x - \int_{0}^{\infty} e^{-bx}{(1-e^{-ax})}^n\mathrm{d}x.
\]
Wartość pierwszej całki wynosi $\frac{1}{b}$.
Aby policzyć drugą podstawmy $u=1-e^{-ax}$.
Oczywiście $u(0)=0, \; u(\infty)=1$.
Wtedy też $x=-\frac{1}{a}\log(1-u)$ oraz $\mathrm{d}x=\frac{1}{a}e^{ax}\mathrm{d}u$.
Mamy zatem $e^{-bx}={(1-u)}^{\frac{b}{a}}, \; e^{ax}=\frac{1}{1-u}$.
A więc całka wynosi
\begin{align*}
&\frac{1}{a}\int_{0}^{1} u^n {(1-u)}^{\frac{b}{a}-1} \;\mathrm{d}u = \frac{1}{a} \mathrm{B}\Big(n+1,\frac{b}{a}\Big) = \frac{1}{a} \frac{\Gamma(n+1)\Gamma(\frac{b}{a})}{\Gamma(n+1+\frac{b}{a})} \\[0.3em]
&=\frac{n!}{a} \cdot\frac{\Gamma(\frac{b}{a})}{(n+\frac{b}{a})(n-1+\frac{b}{a})\dots(\frac{b}{a})\Gamma(\frac{b}{a})} = \frac{n!}{b} \cdot \frac{1}{{(\frac{b}{a}+1)}_n}.
\end{align*}
gdzie $\Gamma(x)$ jest funkcją Gamma, $\mathrm{B}(x,y)$ jest funkcją Beta oraz ${(x)}_n=x(x+1)\dots(x+n-1)$ jest symbolem Pochhammer'a.
Ostatecznie otrzymujemy
\[
    \mathbb{E}[Z] \approx \frac{1}{b} \cdot \Big(1 - \frac{n!}{{(\frac{b}{a}+1)}_n}\Big) = \frac{1}{\log(\beta)} \cdot \frac{n!}{{\big(\frac{\log(\beta)}{\log(q)}+1\big)}_n} - \frac{1}{\log(\beta)}.
\]


\section{Eksperymenty}

W modelu SIR symulować będziemy rozkłady zmiennych $W,Z$ oraz ich wartości oczekiwane.
Ustalmy $p=0.2$ oraz $\alpha=0.05$.
Do przeprowadzenia eksperymentu używamy \cref{algo:SIR}.
Rozkłady wyznaczymy na na grafach o $n=100$ wierzchołkach natomiast wartości oczekiwane dla $n\in\{1,2,\dots,300\}$.

Jeśli chodzi o rodzinę $\mathrm{P}_n$ to rozkład $W$ pokrywa się idealnie z przewidywaniami.
Widzimy nawet kumulację masy w ogonie.
Wartości oczekiwane również zbiegają zgodnie z wyznaczonym wzorem.
Podobnie mamy dla zmiennej losowej $Z$.

Dla grafów gwiazd rozkłady zarówno $W$ jak i $Z$ są dobrze przybliżone.
Wartości oczekiwane $W$ są niemal idealnie dopasowane.
Możemy zauważyć dobre przybliżenie na $\mathbb{E}[W]$.


\begin{algorithm}
\caption{Propagacja SIR}
\begin{algorithmic}[1]
\State\textbf{Input:} Graf $G=(V,E)$, prawdopodobieństwo infekcji $p$, prawdopodobieństwo wyzdrowienia $\alpha$, źródło $s\in V$
\State\textbf{Output:} Zbiór wyzdrowiałych wierzchołków $(\mathcal{R}_t)$, czas wymarcia infekcji $Z$
\State$\mathcal{S}_0 \gets V\setminus\{s\}$
\State$\mathcal{I}_0 \gets \{s\}$
\State$\mathcal{R}_0 \gets \varnothing$
\State$t \gets 0$
\While{$\mathcal{I}_t \ne \varnothing$ \textbf{and} $\exists v\in\mathcal{I}_t: \mathcal{S}_t\cap\mathrm{N}(v)\neq\varnothing$}
    \State$\mathcal{I}' \gets \varnothing$
    \For{\textbf{each} $u \in \mathcal{I}_t$}
        \For{\textbf{each} $v \in \mathrm{N}(u)$}
            \If{$v \in \mathcal{S}_t$ \textbf{and} $\text{random}() < p$}
                \State$\mathcal{I}' \gets \mathcal{I}' \cup \{v\}$
            \EndIf%
        \EndFor%
    \EndFor%
    \State$\mathcal{R}' \gets \varnothing$
    \For{\textbf{each} $u \in \mathcal{I}_t$}
        \If{$\text{random}() < \alpha$}
            \State$\mathcal{R}' \gets \mathcal{R}' \cup \{u\}$
        \EndIf%
    \EndFor%
    \State$\mathcal{S}_{t+1} \gets \mathcal{S}_t \setminus \mathcal{I}'$
    \State$\mathcal{I}_{t+1} \gets (\mathcal{I}_t \cup \mathcal{I}') \setminus \mathcal{R}'$
    \State$\mathcal{R}_{t+1} \gets \mathcal{R}_t \cup \mathcal{R}'$
    \State$t \gets t + 1$
\EndWhile%
\State$\mathcal{R}_t \gets \mathcal{R}_t \cup \mathcal{I}_t$
\State$Z \gets t$
\State\Return$(\mathcal{R}_t),\, Z$
\end{algorithmic}%
\label{algo:SIR}
\end{algorithm}

\begin{figure}[ht!]
    \centering
    \begin{subfigure}{0.48\textwidth}
        \centering
        \includegraphics[width=\textwidth]{../img/SIR/path/W_dist.png}
        \caption{Rozkład zmiennej $W$.}
    \end{subfigure}
    \hfill
    \begin{subfigure}{0.48\textwidth}
        \centering
        \includegraphics[width=\textwidth]{../img/SIR/path/W_expectation.png}
        \caption{Wartość oczekiwana $W$.}
    \end{subfigure}
    \caption{Rozkład i wartość oczekiwana zmiennej $W$ dla $\mathrm{P}_n$.}%
    \label{fig:SIR_path_W}
\end{figure}

\begin{figure}[ht!]
    \centering
    \begin{subfigure}{0.48\textwidth}
        \centering
        \includegraphics[width=\textwidth]{../img/SIR/path/Z_dist.png}
        \caption{Rozkład zmiennej $Z$.}
    \end{subfigure}
    \hfill
    \begin{subfigure}{0.48\textwidth}
        \centering
        \includegraphics[width=\textwidth]{../img/SIR/path/Z_expectation.png}
        \caption{Wartość oczekiwana $Z$.}
    \end{subfigure}
    \caption{Rozkład i wartość oczekiwana zmiennej $Z$ dla $\mathrm{P}_n$.}%
    \label{fig:SIR_path_Z}
\end{figure}

\begin{figure}[ht!]
    \centering
    \begin{subfigure}{0.48\textwidth}
        \centering
        \includegraphics[width=\textwidth]{../img/SIR/star/W_dist.png}
        \caption{Rozkład zmiennej $W$.}
    \end{subfigure}
    \hfill
    \begin{subfigure}{0.48\textwidth}
        \centering
        \includegraphics[width=\textwidth]{../img/SIR/star/W_expectation.png}
        \caption{Wartość oczekiwana $W$.}
    \end{subfigure}
    \caption{Rozkład i wartość oczekiwana zmiennej $W$ dla $\mathrm{S}_n$.}%
    \label{fig:SIR_star_W}
\end{figure}

\begin{figure}[ht!]
    \centering
    \begin{subfigure}{0.48\textwidth}
        \centering
        \includegraphics[width=\textwidth]{../img/SIR/star/Z_dist.png}
        \caption{Rozkład zmiennej $Z$.}
    \end{subfigure}
    \hfill
    \begin{subfigure}{0.48\textwidth}
        \centering
        \includegraphics[width=\textwidth]{../img/SIR/star/Z_expectation.png}
        \caption{Wartość oczekiwana $Z$.}
    \end{subfigure}
    \caption{Rozkład i wartość oczekiwana zmiennej $Z$ dla $\mathrm{S}_n$.}%
    \label{fig:SIR_star_Z}
\end{figure}

\clearpage