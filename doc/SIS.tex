\section{Dwa wierzchołki, jedna krawędź}

W celu oswojenia się z bardziej skomplikowanym modelem, jakim jest SIS, przeanalizujmy graf $\mathrm{K}_2$. 
Niech $V=\{u,v\}$ oraz niech $u$ będzie wierzchołkiem startowym. 
Istnieją cztery możliwe stany systemu: $\mathcal{I}_t=\varnothing, \; \mathcal{I}_t=\{u\}, \; \mathcal{I}_t=\{v\}, \; \mathcal{I}_t=\{u,v\}$. 
Stan, w którym żaden wierzchołek nie jest zainfekowany, jest stanem absorbującym. 
Ponadto, z każdego pozostałego stanu możemy przejść do dowolnego innego. 
Oczywiście $X_u=0$. 
Pierwsza runda jest identyczna jak w modelu SI, a więc $\mathbb{P}[X_v=1]=p$. 
Jeżeli węzeł $v$ nie zostanie poinformowany w rundzie pierwszej, a propagacja nie wygaśnie, to sytuacja się powtórzy. 
Zachodzi to z prawdopodobieństwem $q\beta$. Aby $X_v=t$, potrzebujemy, by ten cykl nastąpił $t-1$ razy. 
A więc
\[
    \mathbb{P}[X_v=t]={(q\beta)}^{t-1}p, \quad t\ge 1.
\]
Istnieje jeszcze możliwość, że wierzchołek $v$ nigdy nie zostanie zainfekowany. 
Wtedy
\[
    \mathbb{P}[X_v < \infty ] = \sum_{t=1}^{\infty} \mathbb{P}[X_v=t] = \sum_{t=1}^{\infty} {(q\beta)}^{t-1}p = \frac{p}{1-q\beta},
\] 
a zatem
\[
    \mathbb{P}[X_v=\infty] = 1 - \frac{p}{1-q\beta} = \frac{q\alpha}{1-q\beta}.
\]
Z założeń modelu $q,\alpha\in(0;1)$, a zatem $\mathbb{P}[X_v=\infty]\ne 0$.  
Wynika z tego, że $\mathbb{E}[X_v] = \infty$. 
Powinniśmy się więc spodziewać, że niezależnie od wartości parametrów $p$ oraz $\alpha$ infekcja się nie rozprzestrzeni. 
Nie jest to jednak pożądany rezultat. 
Policzmy zatem wartość oczekiwaną warunkową:
\begin{equation*}
\begin{aligned}
\mathbb{E}[X_v| X_v < \infty]
&= \frac{1}{\mathbb{P}[X_v<\infty]}\sum_{t=1}^{\infty} t \cdot \mathbb{P}[X_v=t] \\
&= \frac{1-q\beta}{p} \sum_{t=1}^{\infty} t {(q\beta)}^{t-1}p = (1-q\beta)\cdot \frac{1}{{(1-q\beta)}^2}=\frac{1}{1-q\beta}.
\end{aligned}
\end{equation*}

Następnie spróbujmy wyznaczyć rozkład $Y_t$. 
Zauważmy, że $Y_t\in\{0,1,2\}$. 
Prawdopodobieństwa przejść są następujące:
\begin{equation*}
\begin{aligned}
\mathbb{P}[Y_{t+1} = 0 | Y_t = 0 ] &= 1,\\
\mathbb{P}[Y_{t+1} = 0 | Y_t = 1 ] &= q\alpha,\\
\mathbb{P}[Y_{t+1} = 0 | Y_t = 2 ] &= \alpha^2,\\
\mathbb{P}[Y_{t+1} = 1 | Y_t = 0 ] &= 0,\\
\mathbb{P}[Y_{t+1} = 1 | Y_t = 1 ] &= p\alpha+q\beta,\\
\mathbb{P}[Y_{t+1} = 1 | Y_t = 2 ] &= 2\alpha\beta,\\
\mathbb{P}[Y_{t+1} = 2 | Y_t = 0 ] &= 0,\\
\mathbb{P}[Y_{t+1} = 2 | Y_t = 1 ] &= p\beta,\\
\mathbb{P}[Y_{t+1} = 2 | Y_t = 2 ] &= \beta^2.
\end{aligned}
\end{equation*}
Z wzoru na prawdopodobieństwo całkowite dla $\ell \in \{0,1,2\}$ mamy:
\[
    \mathbb{P}[Y_{t+1} = \ell] = \sum_{k=0}^{2} \mathbb{P}[Y_{t+1}=\ell | Y_t=k]\cdot \mathbb{P}[Y_t=k].
\]
Oznaczmy $a_t=\mathbb{P}[Y_t=0], \; b_t=\mathbb{P}[Y_t=1], \; c_t=\mathbb{P}[Y_t=2]$. 
Oczywiście $a_0=0, \; b_0=1, \; c_0=0$. 
Stąd otrzymujemy układ równań rekurencyjnych:
\[
\begin{cases}
    a_{t+1} = a_t + q\alpha\cdot b_t + \alpha^2 \cdot c_t\\
    b_{t+1} = (p\alpha+q\beta)\cdot b_t + 2\alpha\beta\cdot c_t\\
    c_{t+1} = p\beta\cdot b_t + \beta^2\cdot c_t.
\end{cases}
\]
Połóżmy
\[
\mathbf{P} = 
\begin{bmatrix}
1 & q\alpha & \alpha^2 \\
0 & p\alpha+q\beta  & 2\alpha\beta \\
0 & p\beta & \beta^2
\end{bmatrix}, \quad \mathbf{y}_t = \begin{bmatrix}
a_t \\
b_t \\
c_t
\end{bmatrix}.
\]
Wtedy $\mathbf{y}_{t+1}=\mathbf{P}\mathbf{y}_t$, a więc $\mathbf{y}_{t}=\mathbf{P}^t\mathbf{y}_0$. 
Jeśli chodzi o wartość oczekiwaną, to mamy $\mathbb{E}[Y_t]=0\cdot a_t + 1\cdot b_t + 2\cdot c_t = b_t+2c_t$. 
Skoro stan bez zainfekowanych wierzchołków jest absorbujący i osiągalny z dowolnego innego stanu, to zachodzi $a_t \to  1$, natomiast $b_t, c_t \to 0$. 
A więc
\[
    \lim_{t\to\infty} \mathbb{E}[Y_t] = 0.
\]

Przyjrzyjmy się teraz zmiennej $Z$. 
Mamy $\mathbb{P}[Z>t]=\mathbb{P}[Y_t\ne 0] = b_t + c_t$. Oznaczmy
\[
\mathbf{Q} = 
\begin{bmatrix}
p\alpha+q\beta  & 2\alpha\beta \\p\beta & \beta^2
\end{bmatrix}, 
\quad \mathbf{z}_t = \begin{bmatrix}
b_t \\
c_t
\end{bmatrix},
\quad \mathbf{1} = \begin{bmatrix}
1 \\
1
\end{bmatrix}.
\]
Wtedy $\mathbb{P}[Z>t] = \mathbf{1}^\top \mathbf{z}_t = \mathbf{1}^\top \mathbf{Q}^t \mathbf{z}_0$. 
Mamy więc
\[
    \mathbb{E}[Z] = \sum_{t=0}^{\infty} \mathbb{P}[Z>t] = \sum_{t=0}^{\infty} \mathbf{1}^\top\mathbf{Q}^t \mathbf{z}_0 = \mathbf{1}^\top{(\mathbf{I}-\mathbf{Q})}^{-1} \mathbf{z}_0.
\]
Dalej:
\[
    {(\mathbf{I} - \mathbf{Q})}^{-1} = 
    \begin{bmatrix}
    1 - p\alpha-q\beta & -2\alpha\beta \\ -p\beta & 1 -\beta^2
    \end{bmatrix}^{-1}
    =   \frac{1}{\Delta}  
    \begin{bmatrix}
    1 -\beta^2 & 2\alpha\beta \\ p\beta & 1 - p\alpha-q\beta 
    \end{bmatrix},
\]
oraz
\[
    \mathbf{1}^\top{(\mathbf{I}-\mathbf{Q})}^{-1} \mathbf{z}_0 = \mathbf{1}^\top \frac{1}{\Delta} \begin{bmatrix}
        1 -\beta^2 \\
        p\beta
    \end{bmatrix}
    = \frac{1-\beta^2+p\beta}{\Delta},
\]
gdzie $\Delta=\det(\mathbf{I} - \mathbf{Q})$. 
Obliczmy teraz ten wyznacznik:
\begin{equation*}
\begin{aligned}
\Delta &=(1 - p\alpha-q\beta)(1 -\beta^2) - (-2\alpha\beta)(-p\beta) \\
&= 1-p\alpha-q\beta-\beta^2+p\alpha\beta^2+q\beta^3-2p\alpha\beta \\
&=(1-\beta^2)(1-q\beta)-p\alpha(1+\beta^2)\\
&= (1-{(1-\alpha)}^2)(1-(1-p)(1-\alpha))-p\alpha(1+{(1-\alpha)}^2) \\
&=\alpha(2-\alpha)(p+\alpha-p\alpha)-p\alpha(2-2\alpha+\alpha^2) \\
&=\alpha(2p+2\alpha-2p\alpha-\alpha p-\alpha^2+p\alpha^2-2p+2\alpha p-p\alpha^2) \\
&=\alpha^2(2-p-\alpha)=\alpha^2(q+\beta).
\end{aligned}
\end{equation*}
Ostatecznie otrzymujemy:
\[
    \mathbb{E}[Z] = \frac{1-\beta^2+p\beta}{\alpha^2(q+\beta)}.
\]


\section{Prawie pewne wygaśnięcie}

W powyższych obliczeniach mogliśmy zauważyć istotną różnicę pomiędzy modelem SI a SIS.
Mianowicie jest niezerowe prawdopodobieństwo, że wierzchołki w grafie nigdy nie zostaną zainfekowane, a oczekiwana liczba zarażonych dąży wraz z upływem czasu do zera.
Dla tak małego grafu jakim jest $\mathrm{K}_2$ jest to dość niespodziewany rezultat. 
Postaramy się teraz udowodnić tą obserwację dla dowolnego grafu.

\begin{theorem}\label{theorem:infection_dies_out}
Niech $G=(V,E)$ będzie grafem spójnym, $s\in V$ źródłem infekcji oraz $v\in V\setminus\{s\}$.
Wtedy zachodzą następujące tożsamości:
\[
    \mathbb{P}[X_v = \infty] > 0,
\]
\[
    \lim_{t\to\infty} \mathbb{E}[Y_t] = 0,
\]
\[
    \mathbb{P}[Z < \infty] = 1.
\]
\end{theorem}

\begin{proof}
Weźmy $v\in V\setminus\{s\}$. 
Jeśli $\{s,v\}\notin E$ to jedną z możliwości, która sprawi, że $X_v=\infty$, to wygaśnięcie infekcji po pierwszej jednostce czasu. 
Zatem $\mathbb{P}[X_v=\infty]\ge \alpha$.
Gdy zaś $\{s,v\}\in E$ to $\mathbb{P}[X_v=\infty]\ge q\alpha$, bo musimy jeszcze zagwarantować, że $v$ nie zostanie zainfekowany zanim infekcja w $s$ wygaśnie.
Skoro $q,\alpha>0$ to otrzymujemy $\mathbb{P}[X_v = \infty] > 0$.

Rozważmy propagację jako łańcuch Markova, gdzie stanem w czasie $t$ jest $\mathcal{I}_t$. 
Prawdopodobieństwo przejścia do stanu bez infekcji wynosi
\[
    r(\mathcal{I}_t)=\alpha^{|\mathcal{I}_t|}\prod_{u\in \mathcal{I}_t} q^{|\mathrm{N}(u)\cap\mathcal{S}_t|}.
\]
Istotne jest to, że jest to dodatnia liczba. 
Kładziemy
\[
    r^*=\min\{r(\mathcal{I}_t):\mathcal{I}_t\subseteq V\}.
\]
Prawdopodobieństwo przetrwania infekcji po $k$ rundach wynosi $(1-r^*)^k$.
Ale skoro $r^*>0$ to wartość ta dąży do $0$ wraz z $k\to \infty$.
Mamy więc $\mathbb{P}[Z=\infty] = 0$ co jest równoważne z $\mathbb{P}[Z<\infty]=1$.

Skoro infekcja prawie na pewno wymrze to $\mathbf{1}_{Y_t} \xrightarrow{\text{a.s}} 0$ a więc i $Y_t \xrightarrow{\text{a.s}} 0$.
Zauważmy, że $0\le Y_t \le |V|$ dla każdego $t$.
Zatem z Twierdzenia Lebesgue'a o zbieżności ograniczonej otrzymujemy
\[
    \lim_{t\to\infty} \mathbb{E}[Y_t] = \mathbb{E}[\lim_{t\to\infty} Y_t] = \mathbb{E}[0]=0.
\]
\end{proof}