\section*{Streszczenie}

W pracy tej dokonano przeglądu procesów probabilistycznej propagacji zachodzących w grafach nieskierowanych. 
Analizie poddano trzy podstawowe modele epidemiologiczne.
Model pierwszy to model SI, w którym wszystkie wierzchołki po zainfekowaniu pozostają permanentnie zarażone.
Drugi to model SIR, w którym zainfekowanie wierzchołki mogą wyzdrowieć.
Trzeci model to SIS, w którym zainfekowanie wierzchołki mogą powtórnie stać się podatne na infekcje.
Głównym celem pracy jest wyznaczenie rozkładów prawdopodobieństwa oraz wartości oczekiwanych zmiennych losowych opisujących przebieg propagacji.
Zmienne te oddają czas zarażenia poszczególnych wierzchołków, liczba zarażonych wierzchołków w danym momencie oraz czas trwania propagacji. 
W celu opisu modeli wykorzystano teorię prawdopodobieństwa, teorię grafów oraz procesów stochastycznych i łańcuchów Markova.
Wyniki teoretyczne zostały zweryfikowane poprzez symulacje komputerowe. 
W pracy wskazano także potencjalne zastosowania praktyczne i możliwości kontynuacji badań.


\section*{Abstract}

This thesis reviews probabilistic propagation processes occurring in undirected graphs.
Three basic epidemiological models are analyzed.
The first model is the SI model, in which all vertices remain permanently infected after infection.
The second is the SIR model, in which infected vertices can recover.
The third model is SIS, in which infected vertices can become susceptible to infection again.
The main goal of this paper is to determine the probability distributions and expected values of random variables describing the propagation process. 
These variables reflect the time of infection of individual vertices, the number of infected vertices at a given moment, as well as the duration of propagation.
Probability theory, graph theory, and stochastic processes, together with Markov chains are used to describe the models.
The theoretical results are verified by computer simulations.
The paper also indicates potential practical applications and possibilities for further research.