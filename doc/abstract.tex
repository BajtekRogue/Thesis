\section*{Streszczenie}

Niniejsza praca poświęcona jest analizie matematycznej procesów probabilistycznej propagacji w grafach nieskierowanych. 
Rozważane są trzy klasyczne modele epidemiologiczne: model SI, w którym wierzchołki po zarażeniu pozostają zainfekowane na zawsze; model SIR, gdzie zainfekowane wierzchołki mogą przejść do stanu wyzdrowiałego; oraz model SIS, w którym wierzchołki mogą ponownie stać się podatne na zarażenie.
Głównym celem pracy jest wyznaczenie rozkładów prawdopodobieństwa oraz wartości oczekiwanych kluczowych zmiennych losowych opisujących dynamikę propagacji, takich jak czas zarażenia poszczególnych wierzchołków, liczba zarażonych wierzchołków w danej chwili oraz czas całkowitego zarażenia lub wygaśnięcia infekcji. 
Praca wykorzystuje narzędzia z teorii prawdopodobieństwa, teorii grafów oraz teorii procesów stochastycznych. 
Wszystkie rezultaty teoretyczne zostały zweryfikowane poprzez symulacje numeryczne. 
Praca kończy się omówieniem możliwych zastosowań praktycznych oraz kierunków dalszych badań.


\section*{Abstract}

This thesis is devoted to the mathematical analysis of probabilistic propagation processes in undirected graphs. 
Three classical epidemiological models are considered: the SI model, where vertices remain infected forever once infected; the SIR model, where infected vertices can transition to a recovered state; and the SIS model, where vertices can become susceptible to infection again.
The main objective is to determine probability distributions and expected values of key random variables describing propagation dynamics, such as infection time of individual vertices, number of infected vertices at a given time, and time to total infection or extinction. 
The work employs tools from probability theory, graph theory, and the theory of stochastic processes. 
All theoretical results were verified through numerical simulations. 
The thesis concludes with a discussion of potential practical applications and directions for future research.