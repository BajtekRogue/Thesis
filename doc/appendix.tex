\appendix
\cleardoublepage
\chapter*{Dodatek}
\addcontentsline{toc}{chapter}{Dodatek}

Ten dodatek zawiera opis kodu źródłowego użytego do przeprowadzenia symulacji propagacji oraz instrukcję jego uruchomienia.

\section*{Struktura plików}
\begin{itemize}
    \item \texttt{utils.py} — konstrukcja grafów i wzory teoretyczne
    \item \texttt{simulations.py} — algorytmy symulacji propagacji
    \item \texttt{plotting.py} — generowanie wykresów
    \item \texttt{main\_SI.py} — skrypt do symulacji w modelu SI
    \item \texttt{main\_SIR.py} — skrypt do symulacji w modelu SIR
    \item \texttt{main\_SIS.py} — skrypt do symulacji w modelu SIS
\end{itemize}

\section*{Wymagania}
\begin{itemize}
    \item System operacyjny: dowolny z obsługą Pythona (np. Linux Ubuntu 22.04, Windows 10/11)
    \item Python 3.11
    \item Menedżer pakietów pip
    \item Pakiety Python: \texttt{numpy}, \texttt{matplotlib}, \texttt{networkx}, \texttt{scipy}, \texttt{sympy}
\end{itemize}

\section*{Uruchomienie skryptów}
Aby uruchomić symulację, należy wykonać następujące kroki:
\begin{enumerate}
    \item Rozpakuj skompresowane pliki \texttt{*.py}.
    \item Utwórz wirtualne środowisko w katalogu projektu: \texttt{python -m venv .venv}.
    \item Aktywuj środowisko: Linux/macOS: \texttt{source .venv/bin/activate}, Windows (PowerShell): \texttt{.venv\textbackslash{}Scripts\textbackslash{}Activate.ps1}.
    \item Zainstaluj wymagane biblioteki w aktywnym środowisku: 
    \texttt{pip install numpy matplotlib networkx scipy sympy}.
    \item Uruchom wybrany skrypt \texttt{python main\_*.py}.
\end{enumerate}

Program zasymuluje propagację w odpowiednim modelu, z konkretnymi parametrami.
Następnie stworzy wykresy przedstawiające rozkłady i wartości oczekiwane zmiennych losowych omawianych w pracy.
Pliki z wykresami automatycznie zapisują się w folderze \texttt{./img}.
Parametry symulacji (rozmiar grafu, prawdopodobieństwa infekcji i wyzdrowienia, liczba iteracji)
definiowane są bezpośrednio w plikach \texttt{main\_*.py}.
