Konkretne modele propagacji znajdują zastosowania w analizie rozprzestrzeniania się licznych zjawisk w sieciach.
Aby to zilustrować, przyjrzymy się bliżej rozprzestrzenianiu się wirusa komputerowego w sieci.
Postaramy się wskazać użyteczność każdego z modeli omawianych w tej pracy.
Rozważmy sieć komputerową reprezentowaną przez graf nieskierowany $G=(V,E)$.
Wierzchołki w $G$ odpowiadają komputerom, a krawędzie oznaczają bezpośrednie połączenie sieciowe między urządzeniami np. w sieci lokalnej, klastrze serwerów lub infrastrukturze chmurowej.
Załóżmy, że pojedynczy komputer $s\in V$ zostaje zainfekowany przez wirusa.
Ponadto zakładamy, że co pewien ustalony cykl czasu zarażone komputery próbują zarazić sąsiedni sprzęt z prawdopodobieństwem równym $p$.
Wartość $p$ określa skuteczność transmisji wirusa przez istniejące połączenia sieciowe.


\section{Model SI — Sieć bez zabezpieczeń}

Rozważmy starszą infrastrukturę IT, która nie posiada żadnych aktywnych zabezpieczeń antywirusowych ani mechanizmów automatycznego wykrywania zagrożeń.
W takim środowisku, gdy komputer zostanie raz zainfekowany, pozostaje w tym stanie permanentnie.
Ten stan rzeczy może zostać opisany modelem SI.
Stan $S$ reprezentuje urządzenia podatne na infekcję, a $I$ te już zainfekowane.

Dla komputera $v \in V$ zmienna $X_v$ określa po jakim czasie zostanie on zainfekowany.
W kontekście działań obronnych można priorytetyzować te urządzenia o mniejszej wartości $\mathbb{E}[X_v]$.
Zmienna $Y_t$ określa, ile komputerów jest zainfekowanych w chwili $t\in\mathbb{N}$.
Jej rozkład pokazuje, czy atak rozwija się szybko, czy powoli.
Wartość $Z$ to moment, w którym cała infrastruktura zostaje skompromitowana.
Wartość oczekiwana $\mathbb{E}[Z]$ wskazuje średni czas, jaki administrator ma na reakcję, zanim straci kontrolę nad całą siecią.
Znajomość rozkładu tych zmiennych pozwala na projektowania topologii sieci odpornej na szybkie rozprzestrzenianie się zagrożeń.


\section{Model SIR — Sieć z automatycznymi łatkami bezpieczeństwa}

Rozważmy nowoczesną infrastrukturę korporacyjną z aktywnym systemem zarządzania aktualizacjami i automatycznymi mechanizmami odpowiedzi na zachodzące w niej incydenty.
Gdy komputer zostaje zainfekowany, system bezpieczeństwa może wykryć zagrożenie i zastosować łatkę zabezpieczającą, przenosząc tym samym komputer do stanu odpornego ($R$).
Prawdopodobieństwo $\alpha$ modeluje skuteczność systemu automatycznych łatek.
To bardziej realistyczne założenie w kontekście nowoczesnych systemów, gdzie zaktualizowane oprogramowanie jest odporne na znane exploity.
Istnieje możliwość, że wirus wygaśnie zanim dotrze do wszystkich węzłów.

Zmienna $X_v$ nadal określa czas zarażenia $v$.
Do strefy dużego ryzyka należą urządzenia z dużą wartością $\mathbb{P}[X_v<\infty]$.
W tym modelu $Y_t$ jest zdefiniowana jako liczba komputerów, które nie są już w stanie podatnym.
Wraz z czasem zmienna ta stabilizuje się, osiągając wartość równą $W$.
Zmienna $W$ określa całkowitą liczbę komputerów, które zostały zainfekowane w trakcie całej epidemii.
Wartość oczekiwana $\mathbb{E}[W]$ daje średnią liczbę skompromitowanych systemów, co bezpośrednio przekłada się na szacowanie kosztów naprawy i przywrócenia systemów.
Zmienna $Z$ określa moment, w którym epidemia wygasa i można uznać sytuację za opanowaną.
Wartość $\mathbb{E}[Z]$ jest istotna dla planowania czasu trwania trybu kryzysowego w organizacji.


\section{Model SIS — Sieć z tymczasowymi zabezpieczeniami}

Rozważmy sieć z mechanizmami tymczasowego oczyszczania, takimi jak regularne restarty systemów, czy okresowe czyszczenie pamięci podręcznej.
W takim środowisku zainfekowany komputer może wrócić do stanu podatnego z prawdopodobieństwem $\alpha$ w każdej jednostce czasu.
Model ten jest szczególnie odpowiedni dla serwerów z krótkotrwałymi sesjami, czy z automatycznymi resetami.

Zmienna $X_v$ określa czas pierwszego zarażenia komputera $v$.
W kontekście praktycznym, $X_v$ pozwala zrozumieć, które komputery są najbardziej narażone na początkową falę ataku.
$Y_t$ ma szczególne znaczenie, ponieważ liczba zainfekowanych komputerów oscyluje w czasie.
To pozwala na oszacowanie średniego obciążenia zespołu IT obsługą incydentów albo
planowanie pojemności systemów monitoringu bezpieczeństwa.
Zmienna $Z$ określa moment, w którym wirus całkowicie znika z sieci.
Jej rozkład i wartość $\mathbb{E}[Z]$ informują o średnim czasie, przez który wirus będzie krążył w sieci bez aktywnej interwencji.