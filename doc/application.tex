
Modele propagacji SI, SIR i SIS mają szerokie zastosowanie w analizie rozprzestrzeniania się zjawisk w sieciach.
Jednym z naturalnych zastosowań jest modelowanie rozprzestrzeniania się wirusa komputerowego w sieci.
W tym rozdziale przedstawimy szczegółowy scenariusz aplikacyjny dla każdego z trzech modeli, pokazując jak zmienne losowe zdefiniowane wcześniej pozwalają na analizę różnych aspektów epidemii cyfrowej.
Rozważamy sieć komputerową reprezentowaną przez graf $G = (V, E)$, gdzie wierzchołki odpowiadają poszczególnym komputerom, a krawędzie reprezentują bezpośrednie połączenia sieciowe (np.\ w sieci lokalnej, klastrze serwerów lub infrastrukturze chmurowej).
Wirus rozpoczyna swoje rozprzestrzenianie od pojedynczego zainfekowanego komputera $s \in V$.
W każdym cyklu czasowym zainfekowane komputery próbują zarazić bezpośrednio połączone z nimi komputery z prawdopodobieństwem $p$, które modeluje skuteczność transmisji wirusa przez połączenie sieciowe.


\section{Model SI — Sieć bez zabezpieczeń}

Rozważmy starszą infrastrukturę IT, która nie posiada żadnych aktywnych zabezpieczeń antywirusowych ani mechanizmów automatycznego wykrywania zagrożeń.
Może to być na przykład sieć przemysłowych systemów SCADA, które często działają na przestarzałym oprogramowaniu bez regularnych aktualizacji bezpieczeństwa, lub izolowana sieć wewnętrzna firmy z przestarzałymi systemami operacyjnymi.
W takim środowisku, gdy komputer zostanie raz zainfekowany, pozostaje w tym stanie permanentnie — nie ma mechanizmu samoczynnego usunięcia wirusa.
Model SI doskonale opisuje tę sytuację, gdzie $S$ reprezentuje komputery podatne na infekcję, a $I$ komputery już zainfekowane.

Dla każdego komputera $v \in V$ zmienna $X_v$ określa, w której godzinie cyklu ten konkretny komputer zostanie zainfekowany.
Jest to kluczowa informacja dla administratora sieci próbującego zrozumieć dynamikę ataku.
Rozkład prawdopodobieństwa $X_v$ pozwala odpowiedzieć na pytanie: jakie jest prawdopodobieństwo, że dany komputer zostanie zarażony w określonym czasie?
Wartość oczekiwana $\mathbb{E}[X_v]$ wskazuje średni czas, po którym można spodziewać się infekcji tego komputera.
W kontekście praktycznym, znajomość rozkładu $X_v$ pozwala na odpowiednią priorytetyzacje działań obronnych — komputery o małym $\mathbb{E}[X_v]$ powinny być zabezpieczone jako pierwsze, planowanie harmonogramu ręcznej interwencji technicznej a także oszacowanie okna czasowego na izolację krytycznych systemów.

Zmienna $Y_t$ określa, ile komputerów jest zainfekowanych w chwili $t$.
Rozkład $Y_t$ pokazuje możliwe scenariusze rozwoju epidemii — czy atak rozwija się szybko, czy powoli.
Wartość oczekiwana $\mathbb{E}[Y_t]$ daje średnią liczbę zainfekowanych komputerów po czasie $t$.
Z punktu widzenia zarządzania infrastrukturą, $Y_t$ pozwala na monitorowanie tempa rozprzestrzeniania się wirusa czy ocenę skuteczności częściowych działań obronnych.

Zmienna $Z$ opisuje, po ilu cyklach czasowych wszystkie komputery w sieci zostaną zainfekowane.
Jest to moment, w którym cała infrastruktura zostaje skompromitowana.
Rozkład $Z$ określa prawdopodobieństwo różnych scenariuszy czasowych dla całkowitego przejęcia sieci.
Wartość oczekiwana $\mathbb{E}[Z]$ wskazuje średni czas, jaki administrator ma na reakcję zanim stracimy kontrolę nad całą siecią.
Znajomość $Z$ jest krytyczna dla określenia maksymalnego czasu reakcji zespołu bezpieczeństwa czy też oceny ryzyka związanego z konkretną architekturą sieci.
Pozwala to na projektowania topologii sieci odpornej na szybkie rozprzestrzenianie się zagrożeń.


\section{Model SIR — Sieć z automatycznymi łatkami bezpieczeństwa}

Rozważmy nowoczesną infrastrukturę korporacyjną z aktywnym systemem zarządzania aktualizacjami i automatycznymi mechanizmami odpowiedzi na incydenty.
Gdy komputer zostaje zainfekowany, system bezpieczeństwa może wykryć zagrożenie i zastosować łatkę zabezpieczającą, przenosząc komputer do stanu odpornego ($R$).
Prawdopodobieństwo $\alpha$ modeluje skuteczność systemu automatycznych łatek — im większe $\alpha$, tym szybciej komputery są naprawiane.
W modelu SIR komputer przechodzi przez stany: podatny ($S$) $\to$ zainfekowany ($I$) $\to$ odporny ($R$).
Stan $R$ jest absorpcyjny — załatany komputer nie może być ponownie zainfekowany tym samym wariantem wirusa.
To realistyczne założenie w kontekście nowoczesnych systemów, gdzie zaktualizowane oprogramowanie jest odporne na znane exploity.

W modelu SIR zmienna $X_v$ nadal określa czas pierwszego zarażenia komputera $v$.
Jednak w przeciwieństwie do modelu SI, nie każdy komputer musi zostać kiedykolwiek zainfekowany — istnieje możliwość, że wirus wygaśnie zanim dotrze do wszystkich węzłów.
Zachodzi bowiem $\mathbb{P}[X_v = \infty] > 0$, co oznacza, że część komputerów może uniknąć infekcji całkowicie.
Z tego powodu interesujące jest warunkowe oczekiwanie $\mathbb{E}[X_v | X_v < \infty]$ — średni czas zarażenia dla tych komputerów, które rzeczywiście zostały zainfekowane.
Rozkład $X_v$ pozwala na identyfikację komputerów w strefie największego ryzyka.

W modelu SIR definiujemy $Y_t$ jako liczbę komputerów, które nie są już w stanie podatnym.
Pokazuje to postęp epidemii z innej perspektywy — ile systemów zostało już „dotkniętych" (zarażonych lub załatanych).
Wraz z czasem $Y_t$ stabilizuje się, osiągając wartość równą $W$.

Zmienna $W$ określa całkowitą liczbę komputerów, które zostały zainfekowane w trakcie całej epidemii.
To kluczowa metryka dla oceny skali incydentu bezpieczeństwa.
Rozkład $W$ pokazuje możliwe scenariusze zasięgu ataku — czy zainfekowane zostaną tylko nieliczne komputery w pobliżu źródła, czy znaczna część sieci.
Wartość oczekiwana $\mathbb{E}[W]$ daje średnią liczbę skompromitowanych systemów, co bezpośrednio przekłada się na szacowanie kosztów naprawy i przywrócenia systemów, ocenę potencjalnego wycieku danych lub szkód biznesowych, planowanie zasobów zespołu reagowania na incydenty.

Zmienna $Z$ określa moment, w którym epidemia wygasa — żaden nowy komputer nie może już zostać zainfekowany, ponieważ wszystkie zainfekowane komputery albo wyzdrowiały, albo nie mają podatnych sąsiadów.
To moment, w którym incydent bezpieczeństwa można uznać za opanowany.
Rozkład $Z$ i wartość $\mathbb{E}[Z]$ są krytyczne dla planowania czasu trwania trybu kryzysowego w organizacji, oszacowania czasu, przez który zespół bezpieczeństwa musi działać w trybie podwyższonej gotowości. 
Warto jest też dokonać oceny, jak parametry $p$ (skuteczność transmisji) i $\alpha$ (skuteczność łatek) wpływają na czas trwania incydentu.


\section{Model SIS — Sieć z tymczasowymi zabezpieczeniami}

Rozważmy sieć z mechanizmami tymczasowego oczyszczania, takimi jak regularne restarty systemów, okresowe czyszczenie pamięci podręcznej, lub rotacyjne odświeżanie obrazów systemowych.
W takim środowisku zainfekowany komputer może wrócić do stanu podatnego z prawdopodobieństwem $\alpha$ w każdym cyklu.
Model ten jest szczególnie odpowiedni dla serwerów z krótkotrwałymi sesjami czy z automatycznymi resetami.
W modelu SIS komputer może być zarażany wielokrotnie, a wirus może krążyć w sieci przez długi czas, powodując powtarzające się fale infekcji.
W przeciwieństwie do modeli SI i SIR, tutaj nie ma stanu końcowego — sieć może oscylować między różnymi stanami zarażenia.

Podobnie jak w modelu SIR, nie każdy komputer musi zostać kiedykolwiek zainfekowany — wirus może wygasnąć zanim dotrze do odległych węzłów.
Zachodzi $\mathbb{P}[X_v = \infty] > 0$.
Interesuje nas $\mathbb{E}[X_v | X_v < \infty]$ — średni czas do pierwszego zarażenia, jeśli do niego dojdzie.
W kontekście praktycznym, $X_v$ pozwala zrozumieć, które komputery są najbardziej narażone na początkową falę ataku.

W modelu SIS zmienna $Y_t$ ma szczególne znaczenie, ponieważ liczba zainfekowanych komputerów może oscylować w czasie.
Rozkład $Y_t$ pokazuje możliwe stany sieci w różnych momentach — ile komputerów może być jednocześnie skompromitowanych.
To pozwala na oszacowanie średniego obciążenia zespołu IT obsługą incydentów albo
planowanie pojemności systemów monitoringu bezpieczeństwa.
Warto zauważyć, że $\mathbb{E}[Y_t] \to 0$ wraz z $t \to \infty$ — średnio liczba zarażonych komputerów dąży do zera, co oznacza, że wirus ostatecznie wygasa.
Jednak rozkład $Y_t$ dla skończonych $t$ może wykazywać znaczną wariancję.

Zmienna $Z$ określa moment, w którym wirus całkowicie znika z sieci — wszystkie komputery są w stanie podatnym i żaden nie jest zainfekowany.
W przeciwieństwie do modelu SIR, gdzie epidemia wygasa z braku podatnych celów, w modelu SIS wirus wygasa losowo z powodu równoczesnego „wyleczenia" wszystkich zainfekowanych komputerów.
Rozkład $Z$ i wartość $\mathbb{E}[Z]$ informują o średnim czasie, przez który wirus będzie krążył w sieci czy skuteczności strategii „poczekaj aż wirus sam wygaśnie" w porównaniu z aktywną interwencją.