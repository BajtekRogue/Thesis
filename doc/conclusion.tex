Celem pracy było wyznaczenie rozkładów i wartości oczekiwanych zmiennych $X_v$, $Y_t$, $Z$, dla modeli SI, SIR, SIS oraz zmiennej $W$ dla modelu SIR, dla konkretnych rodzin grafów.
W większości rozważanych przypadków udało się znaleźć dokładne wzory na szukane wartości.
Jednakże dla niektórych z nich wyznaczyliśmy przybliżenia lub asymptotykę.
Ponadto udowodniliśmy pewność całkowitej infekcji lub jej wymarcia adekwatnie do modelu.
W modelu SI znaleźliśmy też uniwersalne ograniczenie na czas całkowitej propagacji.
Symulacje w Pythonie metodą Monte Carlo potwierdziły poprawność wyników.
Przeprowadzona analiza pomogła zrozumieć wpływ topologii grafu na szybkość, skuteczność i zasięg rozchodzenia się propagacji.

Głównym problemem teoretycznej analizy było występowanie cykli w grafie, co już w modelu SI sprawiło znaczący problem.
W dalszych badaniach trzeba wziąć pod uwagę, że w rodzinach grafów zawierających sporo cykli wyznaczenie dokładnych rozkładów prawdopodobieństwa zapewne nie będzie możliwe.
W symulacji modelu SIS również kłopotliwy był dobór wielkości parametrów $p$ i $\alpha$ w taki sposób, żeby zapewnić wymarcie infekcji w sensownym czasie.

Kontynuacja pracy może obejmować rozpatrzenie bardziej realistycznych rodzajów sieci takich jak np. grafy Erdősa-Rényiego.
Można także rozważyć uzależnienie parametrów propagacji od wierzchołków i krawędzi grafu, tj. $p_e$ dla $e\in E$ oraz $\alpha_v$ dla $v\in V$.
Na myśl nasuwa się także możliwość zbadania bardziej skomplikowanych modeli epidemiologicznych, takich jak SEIR (Susceptible-Exposed-Infected-Recovered).


Podsumowując, modele probabilistycznej propagacji w grafach stanowią ogromne pole badawcze, które łączy ze sobą rachunek prawdopodobieństwa, teorie grafów i procesy stochastyczne. 
Niniejsza praca omawia podstawowe metody analizy propagacji i prezentuje wnioski uzyskane w toku badania.