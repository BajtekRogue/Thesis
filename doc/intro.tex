\section{Motywacja}

Propagację zjawisk w skupiskach ludzkich można zauważyć od zamierzchłych czasów.
Rozpowszechniały się przede wszystkim epidemie chorób zakaźnych, ale także wiadomości i
pogłoski. 
Współcześnie transmitowane są wirusy komputerowe oraz treści reklamowe i treści plotkarskie.
Zrozumienie mechanizmów działania propagacji ma podstawowe znaczenie w jej zastosowaniu. 
Pozwala przewidzieć rozwój epidemii, wprowadzać skuteczne działania: prowadzić szeroką kampanię informacyjną, zabezpieczać sieć cyfrową przed wirusami, czy wreszcie optymalizować algorytmy wspomagające działania.
Najbardziej obrazową metodą matematyczną do przedstawienia rozprzestrzeniania się relacji są grafy, w których wierzchołki reprezentują jednostki, czyli ludzi lub komputery, a krawędzie określają kontakty społeczne lub połączenia sieciowe.
Zastosowanie teorii grafów i rachunku prawdopodobieństwa pozwala stworzyć najlepsze modele stochastyczne, które uwzględniają losowy charakter propagacji. 
W przypadku epidemii należy pamiętać, że nie każdy kontakt prowadzi do transmisji, a poszczególne jednostki mają różną podatnością na zarażenie.
Modele propagacji znajdą zastosowanie przede wszystkim w epidemiologii, cyberbezpieczeństwie, marketingu, teorii algorytmów oraz w naukach społecznych.
Szczególną uwagę przyciąga ostatnia pandemia koronawirusa, kiedy to zwrócono się w stronę modeli matematycznych, celem prognozy zakażeń.


\section{Cel pracy}

Celem niniejszej pracy jest szeroka analiza probabilistycznych modeli propagacji przy użyciu grafów, ze szczególnym uwzględnieniem wpływu topologii sieci na dynamikę propagacji.
Szczegółowe cele pracy:
\begin{itemize}
    \item zamodelowanie rozprzestrzeniania się informacji w sieciach przy użyciu procesów grafowych,
    \item wyznaczenie rozkładów prawdopodobieństwa i wartości oczekiwanych kluczowych zmiennych losowych dla wybranych rodzin grafów,
    \item przeprowadzenie symulacji komputerowych w Pythonie metodą Monte Carlo w celu weryfikacji wyników teoretycznych,
    \item zbadanie, jak topologia grafu wpływa na tempo i zasięg propagacji,
    \item omówienie praktycznego scenariusza ilustrującego użyteczność wprowadzonych modeli.
\end{itemize}


\section{Zakres pracy}

Praca składa się z $9$ rozdziałów.

\textbf{Rozdział 1.}
Omawia motywację i cele pracy.

\textbf{Rozdział 2.}
Zawiera notację używaną w całej pracy, definicje analizowanych rodzin grafów, wykorzystywane rozkłady prawdopodobieństwa wraz z ich własnościami oraz zbiór faktów, sum i nierówności stosowanych w dowodach i obliczeniach.

\textbf{Rozdział 3.}
Wprowadza formalne definicje trzech badanych modeli: SI, SIR i SIS.
Ponadto dla każdego modelu zawiera opisy kluczowych zmiennych losowych.

\textbf{Rozdział 4.}
Ilustruje praktyczne zastosowanie modeli na przykładzie rozprzestrzeniania się wirusa komputerowego.

\textbf{Rozdział 5.}
Prezentuje szczegółową analizę modelu SI dla rodzin grafów ścieżkowych, gwiezdnych, cyklicznych, pełnych oraz drzew.
Wyprowadza ogólne ograniczenia na czas całkowitego zarażenia. 
Dodatkowo zawiera wyniki symulacji komputerowej.

\textbf{Rozdział 6.}
Wprowadza rozkłady ułatwiające modelowanie propagacji z możliwością jej przerwania, potrzebne do analizy modeli SIR oraz SIS.

\textbf{Rozdział 7.}
Przedstawia analizę modelu SIR dla rodzin grafów ścieżkowych i gwiezdnych oraz wyniki symulacji komputerowej.

\textbf{Rozdział 8.}
Przedstawia analizę modelu SIS dla grafów pełnych oraz wyniki symulacji komputerowej.

\textbf{Rozdział 9.}
Przedstawia podsumowanie pracy, wnioski i możliwości dalszych badań.