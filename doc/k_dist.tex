W celu ułatwienia analizy modeli \textbf{SIR} oraz \textbf{SIS} wprowadzimy nowe rozkłady Prawdopodobieństwa uwzgledniajace możliwość przerwania propagacji.
Te zmienne losowe będą przyjmować wartość nieskończoność w sytuacji, w której nastąpi przerwanie przed pożądanym rezultatem.
Jako że nastepuje to z dodatnim prawdopodobieństwem, $\mathbb{P}[X=\infty]>0$, to ich wartości oczekiwane wynosić będą nieskończoność, $\mathbb{E}[X]=\infty$.
Interesować nas więc bedzie wartość oczekiwana warunkowa przy warunku, że wartość zmiennej losowej jest skończona a więc $\mathbb{E}[X|X<\infty]$.

\section{Rozkład umierający geometryczny}
Rozkład umierający geometryczny jest wariantem rozkładu geometrycznego, w którym proces może zostać przerwany z prawdopodobieństwem $\alpha\in(0;1)$ po każdej próbie.
Zmienna $X$ ma rozkład umierający geometryczny, jeżeli opisuje liczbę prób Bernoulliego potrzebnych do uzyskania pierwszego sukcesu w przypadku, w którym sukces nastąpi przed przetrwaniem eksperymentu.
Jeśli natomiast proces zostanie zabity szybciej niż pierwsza porażka to wtedy $X=\infty$.
Dla wygody oznaczmy $q=1-p, \; \beta=1-\alpha$.
Aby sukces nastąpił po $k$ rundach to potrzebujemy $k-1$ niepowodzeń próby jak i jej przerwania a następnie sukcesu.
A zatem
\[
    \mathbb{P}[X=k] = p(q\beta)^{k-1}, \quad k \in \mathbb{N}_+.
\]
Dystrybuanta jest więc równa:
\[
    \mathbb{P}[X\le t] = \sum_{k=0}^{t} \mathbb{P}[X=k] = \sum_{k=0}^{t} p(q\beta)^{k-1} = p\frac{1-(q\beta)^t}{1-q\beta}.
\]
Ponadto
\[
    \mathbb{P}[X < \infty] = \sum_{k=0}^{\infty} \mathbb{P}[X=k] = \sum_{k=0}^{\infty} p(q\beta)^{k-1} = \frac{p}{1-q\beta}. 
\]
Suma wszystkich prawdopodobieństw wynosi $1$.
A więc
\[
    \mathbb{P}[X=\infty] = 1 - \frac{p}{1-q\beta} = \frac{q\alpha}{1-q\beta}.
\]
Wartość oczekiwana wynosi nieskończoność.
Natomiast
\begin{equation*}
\begin{aligned}
\mathbb{E}[X| X < \infty]
&= \frac{1}{\mathbb{P}[X<\infty]}\sum_{t=1}^{\infty} t \cdot \mathbb{P}[X_v=t] = \frac{1-q\beta}{p} \sum_{t=1}^{\infty} t {(q\beta)}^{t-1}p\\
& = (1-q\beta)\cdot \frac{1}{{(1-q\beta)}^2}=\frac{1}{1-q\beta}.
\end{aligned}
\end{equation*}
Oznaczamy $X \sim \mathrm{KGeo}(p,\alpha)$.


\section{Rozkład umierający dwumianowy}
Rozkład umierający dwumianowy jest wariantem rozkładu dwumianowego, w którym doświadczenia mogą zostać przerwane z prawdopodobieństwem $\alpha\in(0;1)$ po każdej próbie.
Zmienna $X$ ma rozkład umierający dwumianowy, jeżeli opisuje liczbę sukcesów podczas $n$ prób Bernoulliego, gdzie po każdej próbie proces doświadczeń może zostać zabity.
Dla wygody oznaczmy $q=1-p, \; \beta=1-\alpha$.
Niech $F$ będzie zmienną określającą czas porażki.
Dla $1\le j\le n-1$ mamy $\mathbb{P}[F=j]=\alpha\beta^{j-1}$ oraz $\mathbb{P}[F=n]=\beta^{n-1}$.
Jeśli proces zostanie zabity po $j$ rundach to wtedy $X$ ma rozkład $\mathrm{Bin}(j,p)$.
A zatem korzystając z prawdopodobieństwa całkowitego dla $k\in\{0,1,\dots,n\}$ mamy
\begin{equation*}
\begin{aligned}
\mathbb{P}[X=k] &=\sum_{j=k}^{n} \mathbb{P}[X=k|F=j]\cdot\mathbb{P}[F=j]= \\
&=\binom{n}{k}p^k q^{n-k}\beta^{n-1}+\alpha\sum_{j=k}^{n-1}\binom{j}{k}p^k q^{j-k}\beta^{j-1}.   
\end{aligned}    
\end{equation*}
Dalej mamy $\mathbb{E}[X|F=j]=jp$ oraz
\begin{equation*}
\begin{aligned}
    \mathbb{E}[F]&=\sum_{j=1}^{n}j\cdot\mathbb{P}[F=j]=n\beta^{n-1}+\sum_{j=1}^{n-1} j\alpha\beta^{j-1}=\\
    &=n\beta^{n-1}+\alpha\frac{1-n\beta^{n-1}+(n-1)\beta^n}{(1-\beta)^2}  =\\
    &=\frac{1}{\alpha}(n\alpha\beta^{n-1}+1-n\beta^{n-1}+n\beta^n-\beta^n)=\\
    &=\frac{1}{\alpha}(1-\beta^n+n\beta^{n-1}(\alpha+\beta-1))=\frac{1-\beta^n}{\alpha}.
\end{aligned}    
\end{equation*}
Z prawa całkowitej wartości oczekiwanej mamy
\[
    \mathbb{E}[X]=\mathbb{E}[\mathbb{E}[X|F]]=p\mathbb{E}[F]=\frac{p}{\alpha}(1-\beta^n).
\]
Oznaczamy $X\sim\mathrm{KBin}(n,p,\alpha)$.


\section{Rozkład umierający ujemny dwumianowy}
Rozkład umierający ujemny dwumianowy jest wariantem rozkładu ujemnego dwumianowego, w którym doświadczenia mogą zostać przerwane z prawdopodobieństwem $\alpha\in(0;1)$ po każdej próbie.
Zmienna $X$ ma rozkład umierający ujemny dwumianowy, jeżeli opisuje liczbę prób Bernoulliego potrzebnych do uzyskania $m$ sukcesów próbach Bernoulliego, gdzie po każdej próbie proces może się zakończyć.
Jeśli proces zostanie zabity szybciej niż zajdzie $m$ sukcesów to przyjmujemy $X=\infty$.
Dla wygody oznaczmy $q=1-p, \; \beta=1-\alpha$.
Alternatywnie istnieją niezależne zmienne $Y_1,Y_2,\dots,Y_m\sim\mathrm{KGeo}(p,\alpha)$ takie, że $X=Y_1+Y_2+\dots Y_m$.
Aby wyznaczyć rozkład $X$ ustalmy $k\ge m$ i niech $K=\{\mathbf{y}\in\mathbb{N}_+^m:y_1+\cdots+y_m=k\}$.
Wtedy
\begin{equation*}
\begin{aligned}
    \mathbb{P}[X=k] &=\sum_{\mathbf{y}\in K} \mathbb{P}[Y_1=y_1,\dots,Y_m=y_m]=\sum_{\mathbf{y}\in K} \prod_{j=1}^{m}\mathbb{P}[Y_j=y_j] \\
    &= \sum_{\mathbf{y}\in K} \prod_{j=1}^{m} p(q\beta)^{y_j-1}=\sum_{\mathbf{y}\in K} p^m(q\beta)^{y_1+\cdots +y_m-m}  \\
    &=\sum_{\mathbf{y}\in K}p^m(q\beta)^{k-m} = |K|\cdot p^m(q\beta)^{k-m}.
\end{aligned}    
\end{equation*}
Z \Cref{fact:stars_and_bars} mamy $|K|=\binom{k-1}{m-1}$.
Zatem
\[
    \mathbb{P}[X=k] = \binom{k-1}{m-1} p^m (q\beta)^{k-m}, \quad k\ge m.
\]
Dalej mamy
\begin{equation*}
\begin{aligned}
    \mathbb{P}[X<\infty] &=\sum_{k=m}^{\infty} \mathbb{P}[X=k] = \sum_{k=m}^{\infty} \binom{k-1}{m-1} p^m (q\beta)^{k-m} =\\
    &= p^m(q\beta)^{-m} \sum_{k=m}^{\infty} \binom{k-1}{m-1} (q\beta)^k =p^m (q\beta)^{-m} \frac{(q\beta)^m}{(1-q\beta)^m} \\
    &= \frac{p^m}{(1-q\beta)^m}.
\end{aligned}    
\end{equation*}
co daje nam
\[
    \mathbb{P}[X=\infty]=1-\frac{p^m}{(1-q\beta)^m}.
\]
Jeśli chodzi o wartość oczekiwaną to
\begin{equation*}
\begin{aligned}
    &\mathbb{E}[X|X<\infty]=\frac{1}{\mathbb{P}[X<\infty]}\sum_{k=m}^{\infty} k\cdot \mathbb{P}[X=k]\\
    &=\frac{(1-q\beta)^m}{p^m}\sum_{k=m}^{\infty} k \binom{k-1}{m-1} p^m (q\beta)^{k-m} = \\
    &=\frac{(1-q\beta)^m}{(q\beta)^m} \sum_{k=m}^{\infty} k \binom{k-1}{m-1} (q\beta)^k = \\
    &=\frac{(1-q\beta)^m}{(q\beta)^m} \cdot \frac{m(q\beta)^m}{(1-q\beta)^{m+1}}=\frac{m}{1-q\beta}. 
\end{aligned}    
\end{equation*}
Oznaczamy $X\sim\mathrm{KNegBin}(m,p,\alpha)$.