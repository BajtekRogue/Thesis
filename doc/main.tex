\documentclass[a4paper,12pt]{report}

% Kodowanie i język
\usepackage[utf8]{inputenc}
\usepackage[T1]{fontenc}
\usepackage[polish]{babel}

% Matematyka
\usepackage{amsmath}
\usepackage{amssymb}
\usepackage{amsthm}

% Twierdzenia itp.
\newtheorem{definition}{Definicja}
\newtheorem{fact}{Fakt}
\newtheorem{theorem}{Twierdzenie}
\newtheorem{lemma}{Lemat}

% Grafika i reszta
\usepackage{graphicx}
\usepackage{caption}
\usepackage{subcaption}
% \usepackage[none]{hyphenat}
\usepackage{csquotes}

% Hiperłącza i sprytne referencje
\usepackage{hyperref}

% Spolszczenie nazw dla cleveref
\usepackage[nameinlink,capitalise]{cleveref}

\crefname{theorem}{Twierdzenie}{twierdzenia}
\Crefname{theorem}{Twierdzenia}{Twierdzenia}

\crefname{definition}{definicja}{definicje}
\Crefname{definition}{Definicja}{Definicje}

\crefname{fact}{Fakt}{fakty}
\Crefname{fact}{Faktu}{Fakty}

\crefname{lemma}{lemat}{lematy}
\Crefname{lemma}{Lematu}{Lematy}


% Bibliografia
\usepackage[backend=biber,style=numeric]{biblatex}
\addbibresource{bibliography.bib}

% Strona tytułowa
\title{Probabilistyczne modele propagacji w grafach.}
\author{Bartosz Łabuz}
\date{\today}

\begin{document}

\maketitle
\tableofcontents
\newpage

\chapter{Wstęp}
\section{Motywacja i zastosowania}
Propagację wirusów podczas epidemii ludzkość obserwowała już od starożytności. W dzisiejszych czasach, wraz z rozwojem internetu i mediów społecznościowych, mamy możliwość doświadczyć również dynamicznej propagacji informacji. Aby efektywnie rozprzestrzenić informacje, nie można robić tego "na ślepo", lecz trzeba wykorzystać wiedzę teoretyczną. Najbardziej naturalną metodą matematycznej reprezentacji relacji międzyludzkich są grafy: wierzchołkami grafu są ludzie, a krawędzie określają, czy dane osoby mają ze sobą kontakt. Połączenie teori grafów z rachunkiem prawdopodobieństwa pozwala stworzyć dokładny i praktyczny model propagacji informacji.

\section{Cel pracy}
Celem niniejszej pracy jest 
\begin{itemize}
    \item teoretyczna analiza procesów losowej propagacji w grafach,
    \item wyznaczenie rozkładu prawdopodobieństwa propagacji na wybrancyh rodzinach grafów,
    \item symulacja propagacji w środowisku komputerowym w celu zweryfikowania wyników teoretycznych.
\end{itemize}


\section{Zakres pracy}
Praca obejmuje:
\begin{itemize}
    \item wstęp teoretyczny z zakresu teorii grafów i rachunku prawdopodobieństwa,
    \item opis badanych modeli propagacji: SI, SIR, SIS,
    \item implementację symulacji w Pythonie/C++,
    \item analizę wyników i wnioski dotyczące wpływu struktury grafu na propagację.
\end{itemize}


\chapter{Podstawy matematyczne}
\section{Notacja}

Przez $\mathbb{N}$ oznaczamy zbiór liczb naturalnych $\{0,1,2,\dots\}$, a przez $\mathbb{N}_+ = \{1,2,3,\dots\}$.  Moc zbioru $A$ oznaczamy $|A|$. Logarytm naturalny z $x$ oznaczamy $\log(x)$. 
Dla $n\in\mathbb{N}_+$ przez $H_n=1+\frac{1}{2}+\dots+\frac{1}{n}$ oznaczamy $n$'tą liczbę harmoniczną.

Niech $G=(V,E)$ będzie grafem prostym nieskierowanym. Stopień wierzchołka $v \in V$ oznaczamy $\deg(v)$. Zbiór sąsiadów $v\in V$ oznaczamy $\mathrm{N}(v)$. Odległość między $u$ i $v$ oznaczamy $\mathrm{d}(u,v)$ dla $u,v\in V$. Ekscentryczność $v\in V$ oznaczamy $\epsilon(v) = \max_{u\in V} \mathrm{d}(u,v)$. Przez $\delta(G)$ i $\Delta(G)$ oznaczamy odpowiednio minimalny i maksymalny stopień wierzchołka w grafie $G$.

Jeśli $\mathbb{P}$ jest miarą prawdopodobieństwa na przestrzeni $\Omega$ to prawdopodobieństwo zdarzenia $A$ oznaczamy $\mathbb{P}[A]$. Dla zmiennej losowej $X:\Omega\to\mathbb{R}$ jej wartość oczekiwaną oznaczamy $\mathbb{E}[X]$ a jej wariancje $\mathrm{Var}[X]$. Funkcję masy prawdopodobieństwa oznaczamy $\mathbb{P}[X=t]$ a dystrybuante $X$ oznaczamy $F_X(t)$ dla $t\in\mathbb{R}$. Jeśli zmienne losowe $X_1,X_2,\dots, X_n$ są niezależne i o jednakowych rozkładach to mówimy, że są IID.


\section{Rodziny grafów}

\graphfamily{Graf ścieżkowy}
Dla $n \in \mathbb{N}_+$ graf ścieżkowy ma zbiór wierzchołków $V = \{1, 2, \dots, n\}$ oraz zbiór krawędzi $E = \{\{i, i+1\} : i \in \{1, 2, \dots, n-1\}\}$. Oznaczamy go przez $\mathrm{P}_n$.

\graphfamily{Graf gwiazda}
Dla $n \in \mathbb{N}_+$ graf gwiazda ma zbiór wierzchołków $V = \{0, 1, \dots, n\}$ oraz zbiór krawędzi $E = \{\{0, i\} : i \in \{1, 2, \dots, n\}\}$. Oznaczamy go przez $\mathrm{S}_n$.

\graphfamily{Graf pełny}
Dla $n \in \mathbb{N}_+$ graf pełny ma zbiór wierzchołków $V = \{1, 2, \dots, n\}$ oraz zbiór krawędzi $E = \{\{i, j\} : i, j \in \{1, 2, \dots, n\} \land i \ne j\}$. Oznaczamy go przez $\mathrm{K}_n$.

\graphfamily{Graf cykliczny}
Dla $n \in \mathbb{N}_+$ graf cykliczny ma zbiór wierzchołków $V = \{1, 2, \dots, n\}$ oraz zbiór krawędzi 
$E = \{\{i, i+1\} : i \in \{1, 2, \dots, n-1\}\} \cup \{\{n, 1\}\}$. Oznaczamy go przez $\mathrm{C}_n$.


\section{Rozkłady prawdopodobieństwa}

\distribution{Rozkład Bernoulliego}
Próba Bernoulliego to doświadczenie losowe, którego wynik może być jednym z dwóch:
\begin{itemize}
    \item sukces z prawdopodobieństwem  $p \in (0;1)$
    \item porażka z prawdopodobieństwem  $1 - p$
\end{itemize}  
Zmienna losowa $X$ przyjmująca wartość $1$ w przypadku sukcesu i $0$ w przypadku porażki ma rozkład Bernoulliego. Oznaczamy $X \sim \mathrm{Ber}(p)$.
 

\distribution{Rozkład dwumianowy}
Rozkład dwumianowy opisuje liczbę sukcesów w $n$ próbach Bernoulliego. Niech $X$ będzie zmienną losową przyjmującą wartości w $\{0,1,\dots,n\}$, a każda próba ma prawdopodobieństwo sukcesu $p \in (0;1)$.  
Wtedy:
\[
\mathbb{P}[X = k] = \binom{n}{k}p^k(1-p)^{n-k}, \quad k \in \{0,1,\dots,n\}.
\]
Wartość oczekiwana i wariancja mają postać:
\[
    \mathbb{E}[X] = np, \quad \mathrm{Var}[X] = np(1-p)
\]
Oznaczamy $X \sim \mathrm{Bin}(n,p)$.

\distribution{Rozkład geometryczny}
Rozkład geometryczny opisuje liczbę prób Bernoulliego potrzebnych do uzyskania pierwszego sukcesu.  
Niech $X$ będzie zmienną losową przyjmującą wartości w $\mathbb{N}_+$, a każda próba ma prawdopodobieństwo sukcesu $p \in (0;1)$.  
Wtedy:
\[
    \mathbb{P}[X = k] = p(1 - p)^{k-1}, \quad k \in \mathbb{N}_+.
\]
Dystrybuanta jest równa:
\[
    \mathbb{P}[X\le t] = 1 = (1-p)^t
\]
Wartość oczekiwana i wariancja mają postać:
\[
    \mathbb{E}[X] = \frac{1}{p}, \quad \mathrm{Var}[X] = \frac{1 - p}{p^2}
\]
Oznaczamy $X \sim \mathrm{Geo}(p)$.

\distribution{Rozkład ujemny dwumianowy}
Rozkład ujemny dwumianowy opisuje liczbę prób Bernoulliego potrzebnych do uzyskania $m$ sukcesów.  
Niech $X$ oznacza liczbę prób, przy czym każda próba ma prawdopodobieństwo sukcesu $p \in (0;1)$, a liczba sukcesów $m \in \mathbb{N}_+$ jest ustalona.  
Wtedy:
\[
\mathbb{P}[X = k] = \binom{k-1}{m-1} p^m (1 - p)^{k - m}, \quad k \ge m.
\]
Wartość oczekiwana i wariancja mają postać:
\[
    \mathbb{E}[X] = \frac{m}{p}, \quad \mathrm{Var}[X] = \frac{m(1 - p)}{p^2}
\]
Oznaczamy $X \sim \mathrm{NegBin}(m, p)$.

\distribution{Rozkład normalny}
Zdefiniujmy funkcje
\[
    \varphi(t)=\frac{1}{\sqrt{2\pi}}e^{-\frac{t^2}{2}}, \quad \mathbf{\Phi}(t)=\int_{-\infty}^{t} \varphi(x)\;\mathrm{d}x
\]
Niech $\mu \in \mathbb{R}$ oraz $\sigma > 0$. Zmienna losowa $X$ ma rozkład normalny, jeśli jej funkcja gęstości wyraża się wzorem:
\[
f_X(t) = \frac{1}{\sigma}\cdot\varphi\Big(\frac{t-\mu}{\sigma}\Big), \quad t \in \mathbb{R}.
\]
Dystrybuanta jest równa:
\[
\mathbb{P}[X \le t] = \mathbf{\Phi}\Big(\frac{t-\mu}{\sigma}\Big), \quad t \in \mathbb{R}.
\]
Wartość oczekiwana i wariancja:
\[
\mathbb{E}[X] = \mu, \quad \mathrm{Var}[X] = \sigma^2.
\]
Oznaczenie: $X \sim \mathcal{N}(\mu, \sigma^2)$.

Jeśli $\mu = 0$ oraz $\sigma = 1$ to mówimy, że $X$ ma rozkład standardowy normalny. Zauważmy, że $\varphi$ oraz $\mathbf{\Phi}$ są odpowiednio PDF jak i CDF takiego rozkładu.

\section{Tożsamości i nierówności}

\begin{fact}\label{F:approximation_of_sum_by_an_integral}
Niech $a,b\in\mathbb{N}$, $a<b$ oraz $f:[a;b]\to\mathbb{R}$ będzie funkcją ciągłą i monotoniczą.
Jeśli $f$ jest rosnąca to
\[
    \int_{a}^b f(x)\; \mathrm{d}x \le \sum_{k=a}^{b} f(k)\le f(b) + \int_{a}^b f(x)\; \mathrm{d}x
\]
Jeśli $f$ jest malejąca to 
\[
    \int_{a}^b f(x)\; \mathrm{d}x \le \sum_{k=a}^{b} f(k)\le f(a) + \int_{a}^b f(x)\; \mathrm{d}x
\]
\end{fact}

\begin{fact}\label{F:harmonic_upper_bound}
Niech $n\in\mathbb{N}_+$. Wtedy
\[
    H_n \le 1 + \log(n)
\]
\end{fact}

\begin{fact}\label{F:log_vs_x}
Niech $x \in (0;1)$. Wtedy
\[
    \frac{1}{\log(\frac{1}{1-x})} \le \frac{1}{x}
\]
\end{fact}

\begin{fact}[Nierówność między średnimi]\label{F:AM_GM}
Niech $x_1,x_2,\dots,x_n\ge 0$. Wtedy
\[
    \sqrt[n]{x_1\cdots x_n} \le \frac{x_1  + \cdots + x_n}{n}
\]
Równoważnie możemy zapisać
\[
    \log(x_1\cdots x_n) \le n\cdot \log\left(\frac{x_1 + \cdots + x_n}{n}\right)
\]
\end{fact}

\begin{fact}\label{F:binomial_0}
Niech $n\in\mathbb{N}$ oraz $x,y\in\mathbb{R}$. Wtedy
\[
    \sum_{k=0}^{n} \binom{n}{k} x^k y^{n-k}= (x+y)^n
\]
\end{fact}

\begin{fact}\label{F:binomial_1}
Niech $n\in\mathbb{N}$ oraz $x,y\in\mathbb{R}$. Wtedy
\[
    \sum_{k=0}^{n} k\binom{n}{k} x^k y^{n-k} = nx(x+y)^{n-1}
\]
\end{fact}

\begin{fact}\label{F:geo_0}
Niech $n\in\mathbb{N}$ oraz $x\in\mathbb{R}\setminus\{1\}$. Wtedy
\[
    \sum_{k=0}^{n} x^k = \frac{1-x^{n+1}}{1-x}
\]
\end{fact}

\begin{fact}\label{F:geo_1}
Niech $n\in\mathbb{N}$ oraz $x\in\mathbb{R}\setminus\{1\}$. Wtedy
\[
    \sum_{k=0}^{n} k\cdot x^k=\frac{x}{(1-x)^2}\cdot (nx^{n+1}-(n+1)x^n+1)
\]
\end{fact}

\begin{fact}\label{F:geo_0_inf}
Niech $x\in (-1;1)$. Wtedy
\[
    \sum_{k=0}^{\infty} x^k = \frac{1}{1-x}
\]
\end{fact}

\begin{fact}\label{F:geo_1_inf}
Niech $x\in (-1;1)$. Wtedy
\[
    \sum_{k=0}^{\infty} k\cdot x^k=\frac{x}{(1-x)^2}
\]
\end{fact}

\begin{fact}\label{F:max_CDF}
Niech $X_1,X_2,\dots, X_n:\Omega\to\mathbb{R}$ będą IID o CDF równej $F_X$. Zdefiniujmy zmienną losową $Y = \max\{X_1,X_2,\dots, X_n\}$. Wtedy 
\[
    F_Y(t)=F_X^n(t)
\]
\end{fact}

\begin{fact}\label{F:min_CDF}
Niech $X_1,X_2,\dots, X_n:\Omega\to\mathbb{R}$ będą IID o CDF równej $F_X$. Zdefiniujmy zmienną losową $Y = \min\{X_1,X_2,\dots, X_n\}$. Wtedy 
\[
    F_Y(t)=1-(1-F_X(t))^n
\]
\end{fact}

\begin{fact}\label{F:sum_of_geo_RV}
Niech $X_1, X_2, \dots, X_m$ będą niezależnymi zmiennymi losowymi o rozkładzie geometrycznym $\mathrm{Geo}(p)$ oraz $Y=X_1 + X_2 + \cdots + X_m$. Wtedy 
\[
    Y \sim \mathrm{NegBin}(m, p)
\]
\end{fact}

\begin{fact}\label{F:sum_of_normal_RV}
Niech $X \sim \mathcal{N}(\mu_1,\sigma_1^2)$ oraz $Y \sim \mathcal{N}(\mu_2,\sigma_2^2)$ będą niezależnymi zmiennymi losowymi. Wtedy 
\[
    X + Y \sim \mathcal{N}(\mu_1+\mu_2,\sigma_1^2+\sigma_2^2)
\]
\end{fact}

\begin{fact}\label{F:E_abs_normal}
Niech $X \sim \mathcal{N}(\mu,\sigma^2)$. Wtedy
\[
    \mathbb{E}[|X|] = 2\sigma\cdot \varphi\Big(\frac{\mu}{\sigma}\Big)+\mu\cdot (2\mathbf{\Phi}\Big(\frac{\mu}{\sigma}\Big)-1)
\]
\end{fact}

\begin{fact}\label{F:montonicity_of_expectation}
Niech $X,Y:\Omega\to\mathbb{R}$ będą zmiennymi losowymi takim, że dla każdego $\omega\in\Omega$ zachodzi $X(\omega)\le Y(\omega)$. Wtedy. 
\[
    \mathbb{E}[X] \le \mathbb{E}[Y]
\]
\end{fact}

\begin{fact}[Nierówność Jensena dla wartości oczekiwanej]\label{F:Jensen} 
Niech $n\in\mathbb{N}_+$ oraz $g:\mathbb{R}^n\to\mathbb{R}$ będzie funkcją wypukłą zaś $X_1,X_2,\dots, X_n:\Omega\to\mathbb{N}$ będą zmiennymi losowymi (niekoniecznie niezależnymi). Wtedy
\[
    g(\mathbb{E}[X_1],\dots, \mathbb{E}[X_n]) \le \mathbb{E}[g(X_1,\dots,X_n)]
\]
Jeśli $g$ jest wklęsła to nierówność zachodzi w drugą stronę.
\end{fact}

\chapter{Modele propagacji losowej}


Dany jest graf spójny nieskierowany $G = (V, E)$. Propagacja na takim grafie jest procesem stochastycznym. Zakładamy, że czas dla tego procesu jest dyskretny i mierzony w jednostkach naturalnych, zatem za zbiór chwil przyjmujemy $\mathbb{N}$.  
Niech $\mathcal{Q}$ będzie skończonym zbiorem stanów, jakie mogą przyjmować wierzchołki $G$.  
W każdej chwili $t \in \mathbb{N}$ każdy wierzchołek $v \in V$ znajduje się w pewnym stanie $Q \in \mathcal{Q}$.  
Definiujemy zmienną losową $ \mathbf{X} : \mathbb{N}\times V \to \mathcal{Q} $,
taką, że $\mathbf{X}_t(v) = Q$ wtedy i tylko wtedy, gdy wierzchołek $v$ w chwili $t$ znajduje się w stanie $Q$.

\section{Model SI}

Model \textbf{Susceptible—Infected (SI)} opisuje propagację w sieci, w której każdy wierzchołek znajduje się w jednym z dwóch stanów: podatny ($S$) lub zainfekowany ($I$).  
Początkowo ustalony wierzchołek $s \in V$ znajduje się w stanie $I$, natomiast pozostałe wierzchołki są w stanie $S$. Mamy więc $\mathcal{Q} = \{S, I\}$.
W każdej jednostce czasu dowolny zainfekowany wierzchołek może zarazić każdego swojego sąsiada z prawdopodobieństwem $p$, dla ustalonego $p \in (0;1)$.  
Wierzchołek raz zainfekowany pozostaje w tym stanie na zawsze.  
W modelu \textbf{SI} liczba zainfekowanych wierzchołków jest funkcją niemalejącą w czasie.
Dla upraszczenia notacji kładziemy: 
\begin{itemize}
    \item $q=1-p$
    \item $\mathcal{S}_t=\{v\in V: \mathbf{X}_t(v) = S\}$
    \item $\mathcal{I}_t=\{v\in V: \mathbf{X}_t(v) = I\}$
\end{itemize}
Rozkład prawdopodobieństwa w tym modelu jest definiowany przez następujące zależności:
\[
\mathbf{X}_0(v) =
\begin{cases}
I, & \text{jeśli } v = s \\[4pt]
S, & \text{jeśli } v \neq s
\end{cases}
\]
\[
\begin{aligned}
\mathbb{P}[\mathbf{X}_{t+1}(u) = I \mid \mathbf{X}_t(u) = S]
 &= 1 - \prod_{v \in \mathrm{N}(u) \;\cap\; \mathcal{I}_t} q \\[6pt]
\mathbb{P}[\mathbf{X}_{t+1}(u) = S \mid \mathbf{X}_t(u) = S]
 &= \prod_{v \in \mathrm{N}(u) \;\cap\; \mathcal{I}_t} q \\[6pt]
\mathbb{P}[\mathbf{X}_{t+1}(u) = I \mid \mathbf{X}_t(u) = I]
 &= 1 \\[6pt]
\mathbb{P}[\mathbf{X}_{t+1}(u) = S \mid \mathbf{X}_t(u) = I]
 &= 0
\end{aligned}
\]
Zdefiniujmy teraz zmienne losowe opisujące istotne własności.
Dla każdego $v \in V$ definiujmy zmienną losową
\[
X_v = \min \{ t \in \mathbb{N} : v \in \mathcal{I}_t \}
\]
która określa pierwszą chwilę czasu zarażenia wierzchołka $v$.
Jeśli taka chwila nie istnieje (tzn.\ w danym przebiegu procesu wierzchołek $v$ nigdy się nie zarazi), to przyjmujemy $X_v = \infty$.
Zauważmy, że dla każdego $t \in \mathbb{N}$ zachodzi 
\[
    \mathbb{P}[\mathbf{X}_t(v) = I] = \mathbb{P}[X_v \le t]
\]
Następnie dla każdego $t\in\mathbb{N}$ definiujemy zmienną losową 
\[
    Y_t = |\mathcal{I}_t|
\]
oznaczającą liczbę zainfekowanych wierzchołków w chwili $t$. Dodatkowo definiujemy zmienną losową opisującą czas całkowitego zarażenia grafu:
\[
    Z = \min \{ t \in \mathbb{N} : \mathcal{I}_t = V\}
\]
Alternatywnie możemy zapisać $Z = \max_{v \in V} X_v$.

W modelu \textbf{SI} interesują nas następujące wielkości:
\begin{itemize}
    \item rozkład prawdopodobieństwa zmiennych $X_v$, $Y_t$ oraz $Z$
    \item wartości oczekiwane zmiennych, $\mathbb{E}[X_v]$, $\mathbb{E}[Y_t]$ oraz $\mathbb{E}[Z]$
    \item ograniczenia dolne, górne oraz asymptotyka powyższych wartości oczekiwanych kiedy wyznaczenie ich dokładnej wartości nie będzie możliwe
\end{itemize}



\section{Model SIS}

Model \textbf{Susceptible—Infected—Susceptible (SIS)} rozszerza model \textbf{SI} o powracanie wierzchołków zarażonych do stanu podatnego. Wierzchołek zainfekowany może powrócić do stanu podatnego z prawdopodobieństwem $\alpha \in (0;1)$. Tutaj mamy również $\mathcal{Q} = \{S, I\}$.
W modelu \textbf{SIS} liczba zainfekowanych wierzchołków może oscylować w czasie i nie musi osiągnąć stanu pełnego zakażenia. Dla upraszczenia notacji kładziemy $\beta=1-\alpha $.
Rozkład prawdopodobieństwa w tym modelu jest definiowany przez następujące zależności:
\[
\begin{aligned}
\mathbf{X}_0(v) =
\begin{cases}
I, & \text{jeśli } v = s \\[4pt]
S, & \text{jeśli } v \neq s
\end{cases} \\
\mathbb{P}[\mathbf{X}_{t+1}(u) = I \mid \mathbf{X}_t(u) = S]
 &= 1 - \prod_{v \in \mathrm{N}(u) \;\cap\; \mathcal{I}_t} q \\[6pt]
\mathbb{P}[\mathbf{X}_{t+1}(u) = S \mid \mathbf{X}_t(u) = S]
 &= \prod_{v \in \mathrm{N}(u) \;\cap\; \mathcal{I}_t} q \\[6pt]
\mathbb{P}[\mathbf{X}_{t+1}(u) = I \mid \mathbf{X}_t(u) = I]
 &= \beta \\[6pt]
\mathbb{P}[\mathbf{X}_{t+1}(u) = S \mid \mathbf{X}_t(u) = I]
 &= \alpha
\end{aligned}
\]


\section{Model SIR}

Model \textbf{Susceptible—Infected—Recovered (SIR)} rozszerza model \textbf{SI} o dodanie trzeciego stanu. Stanem tym jest $R$ (Recovered). Mamy zatem $\mathcal{Q} = \{S, I, R\}$.
Stan $R$ jest trwały — wierzchołek, który wyzdrowiał, nie może już ani się zarazić, ani nikogo zakazić. Zarażony wierzchołek może przejść z $I$ do stanu $R$ z prawdopodobieństwem $\gamma \in (0;1)$. Dla upraszczenia notacji kładziemy 
\begin{itemize}
    \item $\delta=1-\gamma$
    \item $\mathcal{R}_t=\{v\in V: \mathbf{X}_t(v) = R\}$
\end{itemize}
Rozkład prawdopodobieństwa w tym modelu jest definiowany przez następujące zależności:
\[
\begin{aligned}
\mathbf{X}_0(v) =
\begin{cases}
I, & \text{jeśli } v = s \\[4pt]
S, & \text{jeśli } v \neq s
\end{cases} \\
\mathbb{P}[\mathbf{X}_{t+1}(u) = I \mid \mathbf{X}_t(u) = S]
 &= 1 - \prod_{v \in \mathrm{N}(u) \;\cap\; \mathcal{I}_t} q \\[6pt]
\mathbb{P}[\mathbf{X}_{t+1}(u) = S \mid \mathbf{X}_t(u) = S]
 &= \prod_{v \in \mathrm{N}(u) \;\cap\; \mathcal{I}_t} q \\[6pt]
\mathbb{P}[\mathbf{X}_{t+1}(u) = R \mid \mathbf{X}_t(u) = I]
 &= \gamma \\[6pt]
\mathbb{P}[\mathbf{X}_{t+1}(u) = I \mid \mathbf{X}_t(u) = I]
 &= \delta \\[6pt]
\mathbb{P}[\mathbf{X}_{t+1}(u) = Q \mid \mathbf{X}_t(u) = R]
 &= 
\begin{cases}
1, & \text{dla } Q = R \\[4pt]
0, & \text{dla } Q \in \{S, I\}
\end{cases}
\end{aligned}
\]



\chapter{Analiza modelu SI}
\section{Dwa wierzchołki, jedna krawędź}

Na samym początku rozważmy najprostrzy graf, czyli o dwoch wierzchołkach $u,v$ połączonych krawędzią. Za wierzchołek startowy wybierzmy $u$. W tym przypadku istnieją tylko dwa możliwe stany systemu: $(I,S)$ oraz $(I,I)$. Przejście ze stanu $(I, S)$ do $(I, I)$ następuje z prawdopodobieństwem $p$ w każdej jednostce czasu. Zatem czas zarażenia drugiego wierzchołka $X_v$ ma rozkład geometryczny, $X_v \sim \mathrm{Geo}(p)$. Jeśli chodzi o rozkład $Y_t$ to mamy:
\begin{itemize}
    \item $\mathbb{P}[Y_t=1]=q^t$, bo próba zarażenia musiałaby nie udać się $t$ razy
    \item $\mathbb{P}[Y_t=2]=1-q^t$
\end{itemize}
Stąd $\mathbb{E}[Y_t]=1\cdot q^t + 2 \cdot (1-q^t) = 2-q^t$. Jeśli chodzi o zmienną $Z$ to zachodzi $Z=\max\{X_u,X_v\}=X_v$ a więc również $Z\sim \mathrm{Geo}(p)$.

\section{Analiza dla grafów $\mathrm{P}_n$}

Jako pierwszą rodzinę grafów rozważmy grafy ścieżkowe $\mathrm{P}_n$. Bez straty ogólności niech $V=\{1,2,\dots,n\}$. Załóżmy, że proces zaczyna się w wierzchołku $s=1$. Zatem infekcja rozchodzi się po grafie "od lewej do prawej". Dla tej rodziny grafów uda nam się wyznaczyć dokładny rozkład prawdopodobieństwa. \\
Dla ścieżki $\mathrm{P}_n$ z wierzchołkiem początkowym $s=1$,  
czasy zarażenia kolejnych wierzchołków tworzą ciąg zmiennych losowych
\[
X_1 = 0, \quad X_{k} = X_{k-1} + U_k, \quad k\in\{2,3,\dots,n\},
\]
gdzie $U_1,Z_2,\dots,U_n \sim \mathrm{Geo}(p)$ oraz $U_1,U_2,\dots,U_n$ są niezależne. 

Widzimy zatem, że
\[
X_k \sim U_1 + U_2 + \dots + U_{k-1},
\]
a więc z faktu ~\ref{F:sum_of_geo_RV} $X_k$ ma rozkład ujemny dwumianowy o parametrach $(k-1, p)$, 
\[
X_k\sim \mathrm{NegBin}(k-1, p).
\]

Ponadto mamy:
\begin{itemize}
    \item $\mathbb{E}[X_k] = \frac{k-1}{p}$,
    \item $\mathrm{Var}[X_k] = \frac{(k-1)(1-p)}{p^2}$.
\end{itemize}

Aby obliczyć rozkład $Y_t$ zauważmy, że liczba dodatkowych zakażeń poza startowym wierzchołkiem do czasu $t$ to po prostu liczba sukcesów w $t$ niezależnych prób Bernoulliego. Musimy jednak pamiętać, że $Y_t$ nie może przekroczyć $n$. Zatem mamy dokładnie
\[
Y_t = \min\{n, 1 + B_t\}, \quad \text{gdzie} \quad B_t \sim \mathrm{Bin}(t,p).
\]
Pozwala to na wyznaczenie PMF dla $Y_t$:

Dla $1 \le k \le n-1$ mamy:
\[
\mathbb{P}[Y_t=k] = \mathbb{P}[B_t=k-1] = \binom{t}{k-1} p^{k-1} q^{t-k+1},
\]  

oraz dla $k = n$ mamy:
\[
\mathbb{P}[Y_t=n] = \mathbb{P}[B_t \ge n-1] = \sum_{j=n-1}^{t} \binom{t}{j} p^j q^{t-j}.
\]

Przejdźmy teraz do obliczania wartości oczekiwanej $Y_t$:
\begin{align*}
\mathbb{E}[Y_t] 
&= \sum_{k=1}^{n-1} k \cdot \mathbb{P}[Y_t=k] + n \cdot \mathbb{P}[Y_t=n] \\
&= \sum_{k=1}^{n-1} k \cdot \binom{t}{k-1} p^{k-1} q^{t-k+1} 
   + n \cdot \sum_{j=n-1}^{t} \binom{t}{j} p^j q^{t-j}.
\end{align*}

W pierwszej sumie podstawiamy $j = k-1$, co pozwala nam złączyć obie sumy i otrzymać:
\[
    \mathbb{E}[Y_t] = \sum_{j=0}^{t} \min\{n, 1+j\} \binom{t}{j} p^j q^{t-j}.
\]

Policzmy teraz asymptotykę dla $n \to \infty$. Wtedy $n > 1 + j$ dla wszystkich $0 \le j \le t$, a więc:
\[
    \lim_{n \to \infty}\mathbb{E}[Y_t] = \sum_{j=0}^{t} (1+j) \binom{t}{j} p^j q^{t-j}.
\]
Rozdzielając sumę na dwa składniki, otrzymujemy:
\[
    \lim_{n \to \infty}\mathbb{E}[Y_t] = \sum_{j=0}^{t} \binom{t}{j} p^j q^{t-j} 
+ \sum_{j=0}^{t} j \binom{t}{j} p^j q^{t-j}.
\]
Korzystając z ~\ref{F:binomial} oraz ~\ref{F:binomial1} otrzymujemy
\[
    (p+q)^t+tp(p+q)^{t-1}=1+tp
\]
Zatem
\[
    \lim_{n \to \infty}\mathbb{E}[Y_t] = 1+tp.
\]

Czas całkowitego zainfekowania grafu $\mathrm{P}_n$ to $Z = \max\{X_1,X_2,\dots,X_n\} = X_n$. Zatem rozkład zmiennej $Z$ jest już nam znany a wartość oczekiwana wynosi $\mathbb{E}[Z]=\frac{n-1}{p}$.

\noindent
Jeśli wierzchołek początkowy $s \ne 1, n$, to proces rozprzestrzeniania się infekcji możemy rozdzielić na dwa niezależne procesy stochastyczne, 
zachodzące na podgrafach indukowanych przez zbiory:
\[
V_1 = \{1, 2, \dots, s\}, \qquad 
V_2 = \{s, s+1, \dots, n\}.
\]
Każdy z tych procesów ma charakter modelu SI na ścieżce, 
z tym że infekcja w wierzchołku $s$ pełni rolę źródła w obu częściach.


\section{Analiza dla grafów $\mathrm{S}_n$}

Następnie rozpatrzmy rodzinę grafów gwiazd $\mathrm{S}_n$. Przyjmujemy $V=\{0,1,2,\dots,n\}$ oraz, że wierzchołek $0$ jest środkiem gwiazdy. W modelu dla tej rodziny zakładamy również $s=0$. Propagacja rozchodzi się tutaj po każdym ramieniu gwiazdy niezależnie. 
Stąd mamy $X_v \sim \mathrm{Geo}(p)$ dla każdego $v\in\{1,2,\dots,n\}$. Zauważmy ponadto, że zmienne $X_1,X_2,\dots,X_n$ są od siebie niezależne.\\

Kwestia $Y_t$ jest również dość prosta. Skoro każdy propagacja działa na każdym wierzchołku niezależnie to zmienna $Y_t$ jest wynikiem $n$ prób Bernoulliego. Sukces pojedynczej próby to prawdopodobieństwo, że zmienna $X_v$ o rozkładzie geometrycznym po conajwyżej $t$ jednostkach czasu osiągnie swój sukces. A więc jest to $\mathbb{P}[X_v\le t]$ co jest równe $1-q^t$. Podsumowując mamy
\[
    Y_t \sim \mathrm{Bin}(n, 1-q^t)
\]
Stąd oczywiście otrzymujemy $\mathbb{E}[Y_t] = n\cdot (1-q^t)$. \\

Przejdźmy teraz to zmiennej $Z$. Przypomnijmy, że $Z=\max\{X_1,x_2,\dots,X_n\}$. Skoro zmienne te są IID, to z ~\ref{F:max_CDF} mamy 
\[
    \mathbb{P}[Z\le t] = (1-q^t)^n
\]
Policzmy teraz wartość oczekiwaną $Z$ na mocy ~\ref{F:expected_value_tail_sum}:
\begin{align*}
\mathbb{E}[Z] 
&= \sum_{k=1}^{\infty} \mathbb{P}[Z\ge k] 
 = \sum_{k=1}^{\infty} 1 - \mathbb{P}[Z\le k-1]
 = \sum_{k=1}^{\infty}  1 - (1-q^{k-1})^n  \\
&= \sum_{k=0}^{\infty}  1 - (1-q^k)^n 
 = \sum_{k=0}^{\infty} \left( 1 - \sum_{j=0}^{n} \binom{n}{j} (-1)^j q^{kj} \right) 
 = \sum_{k=0}^{\infty} \sum_{j=1}^{n} \binom{n}{j} (-1)^{j+1} q^{kj} \\
&= \sum_{j=1}^{n} \sum_{k=0}^{\infty} \binom{n}{j} (-1)^{j+1} (q^j)^k 
 = \sum_{j=1}^{n} \binom{n}{j} \frac{(-1)^{j+1}}{1-q^j}.
\end{align*}
Nie jest to jednak przyzwoity wynik i nie ma postaci zwartej. Spróbujmy zatem wyznaczyć asymptotykę $\mathbb{E}[Z]$. Skoro $\mathbb{E}[Z] = \sum_{k=0}^{\infty}  1 - (1-q^k)^n$ to kładząc $f(x) = 1 - (1 - e^{-\alpha x})^n$ gdzie $\alpha = -\log(q)$ z \ref{T:approximation_of_sum_by_an_integral} możemy oszacować tą sumę. Oczywiście $f(0)=1$ a $f(\infty)=0$ oraz $f$ jest malejąca a więc
\[
    \int_{0}^{\infty} f(x) \; \mathrm{d}x \le \mathbb{E}[Z] \le  1 + \int_{0}^{\infty} f(x) \; \mathrm{d}x
\]
Całka ta jest równa $\frac{-1}{\log(q)} H_n$ (~\ref{F:integral_1}). Finalnie, podstawiając $H_n \sim \log(n)$ otrzymujemy
\[
    \mathbb{E}[Z] \sim -\frac{\log(n)}{\log(q)}
\]


\section{Analiza dla drzew}

Rozważmy drzewo $G = (V, E)$ oraz ustalony wierzchołek początkowy $s \in V$, 
który traktujemy jako \textbf{korzeń drzewa}. Niech $v\in V\setminus\{s\}$ oraz $d=\mathrm{d}(s,v)$. Istnieje dokładnie jedna ścieżka od $s$ do $v$, powiedzmy $s,v_1,...,v_{d-1}, v$. Ponieważ infekcja rozprzestrzenia się od korzenia $s$ wzdłuż krawędzi drzewa, 
każde zakażenie wymaga sukcesu w niezależnym doświadczeniu Bernoulliego o prawdopodobieństwie $p$.
W konsekwencji, aby infekcja dotarła z $s$ do $v$, 
musi wystąpić $\mathrm{d}(s,v)$ kolejnych sukcesów. Zatem rozkład $X_v$ pokrywa się z rozkładem tej zmiennej dla grafu $\mathrm{P}_{d+1}$ na wierzchołkach $\{s,v_1,...,v_{d-1}, v\}$. 
Zatem 
\[
    X_v\sim \mathrm{NegBin}(\mathrm{d}(s,v),p)
\]
oraz
\begin{itemize}
    \item $\mathbb{E}[X_v] = \frac{\mathrm{d}(s,v)}{p}$
    \item $\mathrm{Var}[X_v] = \frac{\mathrm{d}(s,v)(1 - p)}{p^2}$
\end{itemize}


Niech $\{\ell_1,\dots, \ell_m\}$ będą liściami w $G$. Wtedy mamy 
\[
    Z = \max_{1\le i \le m} X_{\ell _i}
\] 
Połóżmy $d_i=\mathrm{d}(s,\ell_i)$ dla $1\le i \le m$ oraz bez starty ogólności niech $d_1\ge d_2\ge\dots\ge d_m$. Zauważmy, że $d_1$ to wysokość drzewa, $d_1=h$.
Wiemy, że funkcja $\max\{x_1,\dots, x_m\}$ jest wypukła więc z nierówności Jensena ~\ref{T:Jensen} otrzymujemy
\[
    \mathbb{E}[Z]=\mathbb{E}[\max_{1\le i \le m} X_{\ell _i}] \ge \max_{1\le i \le m} \mathbb{E}[X_{\ell _i}] = \max_{1\le i \le m} \frac{d_i}{p} = \frac{d_1}{p} = \frac{h}{p}
\]
Dla ograniczenie górnego korzystamy z ~\ref{F:max_inequality} oraz ~\ref{F:montonicity_of_expectation} i dostajemy
\[
    \mathbb{E}[Z]=\mathbb{E}[\max_{1\le i \le m} T_{\ell _i}] \le \sum_{i=1}^{m}\mathbb{E}[T_{\ell _i}] = \sum_{i=1}^{m} \frac{d_i}{p} = \frac{1}{p} \sum_{i=1}^{m} d_i \le \frac{1}{p} md_1=\frac{mh}{p}
\]
Ostatecznie 
\[
  \frac{h}{p} \le \mathbb{E}[Z] \le \frac{mh}{p} 
\]
Dla grafu $G=\mathrm{P}_n$ mamy $m=1, \; h=n-1$ a więc nierówności zamieniają sie w równość, z resztą zgodnie z poprzednimi wynikami. Oszacowania na $\mathbb{E}[Z]$ zdaje się więc nie móc poprawić w ogólności względem $m$ oraz $h$. \\
TODO : Generować losowe drzewa i zasymulować, może coś się wywnioskuje. Zgaduje, że mimo wszystko $\mathbb{E}[Z]$ powinno być $\mathcal{O}(n)$.


\newpage
\printbibliography

\end{document}
