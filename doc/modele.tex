
Dany jest graf spójny nieskierowany $G = (V, E)$. 
Propagacja na takim grafie jest procesem stochastycznym. 
Zakładamy, że czas dla tego procesu jest dyskretny i mierzony w jednostkach naturalnych, zatem za zbiór chwil przyjmujemy $\mathbb{N}$.  
Niech $\mathcal{Q}$ będzie skończonym zbiorem stanów, jakie mogą przyjmować wierzchołki $G$.  
W każdej chwili $t \in \mathbb{N}$ każdy wierzchołek $v \in V$ znajduje się w pewnym stanie $Q \in \mathcal{Q}$.  
Definiujemy zmienną losową $\mathbf{X} : \mathbb{N}\times V \to \mathcal{Q}$, taką, że $\mathbf{X}_t(v) = Q$ wtedy i tylko wtedy, gdy wierzchołek $v$ w chwili $t$ znajduje się w stanie $Q$.  

\section{Model SI}

Model \textbf{Susceptible—Infected (SI)} opisuje propagację w sieci, w której każdy wierzchołek znajduje się w jednym z dwóch stanów: podatny ($S$) lub zainfekowany ($I$).  
Mamy więc $\mathcal{Q} = \{S, I\}$.  
Początkowo ustalony wierzchołek $s \in V$ znajduje się w stanie $I$, natomiast pozostałe wierzchołki są w stanie $S$. 
A więc
\[
\mathbf{X}_0(v) =
\begin{cases}
I, & \text{jeśli } v = s,\\
S, & \text{jeśli } v \neq s.
\end{cases}
\]
W każdej jednostce czasu dowolny zainfekowany wierzchołek może zarazić każdego swojego sąsiada z prawdopodobieństwem $p$, dla ustalonego $p \in (0;1)$.  
Wierzchołek raz zainfekowany pozostaje w tym stanie na zawsze.  
W modelu \textbf{SI} liczba zainfekowanych wierzchołków jest funkcją niemalejącą w czasie.  
Dla uproszczenia notacji kładziemy 
\[
    q=1-p,\quad \mathcal{S}_t=\{v\in V: \mathbf{X}_t(v) = S\}, \quad \mathcal{I}_t=\{v\in V: \mathbf{X}_t(v) = I\}.
\]
Rozkład prawdopodobieństwa w tym modelu jest definiowany przez następujące zależności:
\begin{equation*}
\begin{aligned}
\mathbb{P}[\mathbf{X}_{t+1}(u) = S \mid \mathbf{X}_t(u) = S]
    &= \prod_{v \in \mathrm{N}(u) \cap \mathcal{I}_t} q, \\[4pt]
\mathbb{P}[\mathbf{X}_{t+1}(u) = I \mid \mathbf{X}_t(u) = S]
    &= 1 - \prod_{v \in \mathrm{N}(u) \cap \mathcal{I}_t} q, \\[4pt]
\mathbb{P}[\mathbf{X}_{t+1}(u) = S \mid \mathbf{X}_t(u) = I]
    &= 0, \\[4pt]
\mathbb{P}[\mathbf{X}_{t+1}(u) = I \mid \mathbf{X}_t(u) = I]
    &= 1.
\end{aligned}
\end{equation*}

Zdefiniujmy teraz zmienne losowe opisujące istotne dla nas własności.  
Dla każdego $v \in V$ definiujemy zmienną $X_v$ określającą chwilę czasu zarażenia wierzchołka $v$.  
Formalnie
\[
    X_v = \min \{ t \in \mathbb{N} : v \in \mathcal{I}_t \}.
\]
Jeśli taka chwila nie istnieje (tzn.\ w danym przebiegu procesu wierzchołek $v$ nigdy się nie zarazi), to przyjmujemy $X_v = \infty$.  
Później udowodnimy (\cref{theorem:total_infection}), że w modelu \textbf{SI} mamy $\mathbb{P}[X_v=\infty]=0$. 
Wyznaczenie rozkładu $X_v$ jak i $\mathbb{E}[X_v]$ da nam sporo informacji o propagacji na grafie w zależności od jego topologi.

Następnie dla każdego $t\in\mathbb{N}$ definiujemy zmienną losową $Y_t$ oznaczającą liczbę zainfekowanych wierzchołków w chwili $t$. 
Zatem
\[
    Y_t = |\mathcal{I}_t|.
\]
Interesować nas będzie rozkład prawdopodobieństwa $Y_t$ oraz wartość oczekiwana $\mathbb{E}[Y_t]$.
Pokażemy, że $\mathbb{E}[Y_t] \to |V|$ wraz z $t\to\infty$.
Dlatego też nie będziemy badać asymptotyki $\mathbb{E}[Y_t]$ względem $t$.

Dodatkowo definiujemy zmienną $Z$ opisującą czas całkowitego zarażenia grafu:
\[
    Z = \min \{ t \in \mathbb{N} : \mathcal{I}_t = V\}.
\]
Jeśli ta sytuacją nigdy by nie nastąpiła to mielibyśmy $Z=\infty$.
Jednakże dla propagacji \textbf{SI} zachodzi zainfekowanie całego grafu jest zdarzeniem pewnym, $\mathbb{P}[Z<\infty]=1$.
Alternatywnie możemy zapisać $Z = \max_{v \in V} X_v$.  
Wyznacznie rozkładu $Z$ oraz wartości oczekiwanej dla konkretnych rodzin grafów będzie głównym celem w tym modelu.
Dodatkowo interesować nas będzie asymptotyczne oszacowanie $\mathbb{E}[Z]$ względem liczby wierzchołków.


\section{Model SIS}

Model \textbf{Susceptible—Infected—Susceptible (SIS)} rozszerza model \textbf{SI} o powracanie wierzchołków zarażonych do stanu podatnego.  
Wierzchołek zainfekowany może powrócić do stanu podatnego z prawdopodobieństwem $\alpha \in (0;1)$.  
Tutaj mamy również $\mathcal{Q} = \{S, I\}$.  
W modelu \textbf{SIS} liczba zainfekowanych wierzchołków może oscylować w czasie i nie musi osiągnąć stanu pełnego zakażenia.  
Dla uproszczenia notacji kładziemy $\beta=1-\alpha$.  
Przyjmujemy, że w każdej rundzie wierzchołki najpierw przekazują infekcję, a potem dopiero mogą powrócić w stan podatności.  
Rozkład prawdopodobieństwa w tym modelu jest definiowany przez następujące zależności:
\begin{equation*}
\begin{aligned}
\mathbb{P}[\mathbf{X}_{t+1}(u) = S \mid \mathbf{X}_t(u) = S]
    &= \prod_{v \in \mathrm{N}(u) \cap \mathcal{I}_t} q, \\[4pt]
\mathbb{P}[\mathbf{X}_{t+1}(u) = I \mid \mathbf{X}_t(u) = S]
    &= 1 - \prod_{v \in \mathrm{N}(u) \cap \mathcal{I}_t} q, \\[4pt]
\mathbb{P}[\mathbf{X}_{t+1}(u) = S \mid \mathbf{X}_t(u) = I]
    &= \alpha, \\[4pt]
\mathbb{P}[\mathbf{X}_{t+1}(u) = I \mid \mathbf{X}_t(u) = I]
    &= \beta.
\end{aligned}
\end{equation*}

Zmienne losowe opisujące istotne własności są tutaj podobne jak w modelu \textbf{SI}.  
Dla każdego $v \in V$ kładziemy
\[
X_v = \min \{ t \in \mathbb{N} : v \in \mathcal{I}_t \},
\]
oraz dla $t\in\mathbb{N}$ definiujemy 
\[
    Y_t = |\mathcal{I}_t|.
\]
Dodatkowo definiujemy zmienną losową opisującą czas wygaśnięcia infekcji:
\[
    Z = \min \{ t \in \mathbb{N} : \mathcal{I}_t = \varnothing\}.
\]
Wygaśnięcie infekcji jest zachodzi z prawdopodobieństwem $1$ (patrz \cref{theorem:infection_dies_out}).
W kontraście dla modelu \textbf{SI} mamy $\mathbb{P}[X_v=\infty] > 0$ a więc $\mathbb{E}[X_v]=\infty$.
Przekierujemy naszą uwagę zatem na $\mathbb{E}[X_v|X_v<\infty]$.
Dalej wykażemy, że $\mathbb{E}[Y_t]$ dążą do $0$, a więc także nie będą nas interesować.
Skupimy się tylko na rozkładzie $Y_t$.
Dla zmiennej $Z$ przyjrzyjmy się zarówno rozkładowi jak i wartości oczekiwanej.

\section{Model SIR}

Model \textbf{Susceptible—Infected—Recovered (SIR)} rozszerza model \textbf{SI} o dodanie trzeciego stanu.  
Stanem tym jest $R$ (Recovered). 
W tym modelu mamy $\mathcal{Q} = \{S, I, R\}$.  
Stan $R$ jest trwały — wierzchołek, który wyzdrowiał, nie może już ani się zarazić, ani nikogo zakazić.  
Zarażony wierzchołek może przejść z $I$ do stanu $R$ z prawdopodobieństwem $\alpha \in (0;1)$.  
Dla uproszczenia notacji kładziemy 
\[
    \beta=1-\alpha,\quad \mathcal{R}_t=\{v\in V: \mathbf{X}_t(v) = R\}.
\]
Rozkład prawdopodobieństwa w tym modelu jest definiowany przez następujące zależności:
\begin{equation*}
\begin{aligned}
    \mathbb{P}[\mathbf{X}_{t+1}(u) = S \mid \mathbf{X}_t(u) = S]
        &= \prod_{v \in \mathrm{N}(u) \cap \mathcal{I}_t} q, \\[4pt]
    \mathbb{P}[\mathbf{X}_{t+1}(u) = I \mid \mathbf{X}_t(u) = S]
        &= 1 - \prod_{v \in \mathrm{N}(u) \cap \mathcal{I}_t} q, \\[4pt]
    \mathbb{P}[\mathbf{X}_{t+1}(u) = R \mid \mathbf{X}_t(u) = S]
        &= 0, \\[4pt]
    \mathbb{P}[\mathbf{X}_{t+1}(u) = S \mid \mathbf{X}_t(u) = I]
        &= 0, \\[4pt]
    \mathbb{P}[\mathbf{X}_{t+1}(u) = I \mid \mathbf{X}_t(u) = I]
        &= \beta, \\[4pt]
    \mathbb{P}[\mathbf{X}_{t+1}(u) = R \mid \mathbf{X}_t(u) = I]
        &= \alpha, \\[4pt]
    \mathbb{P}[\mathbf{X}_{t+1}(u) = S \mid \mathbf{X}_t(u) = R]
        &= 0, \\[4pt]
    \mathbb{P}[\mathbf{X}_{t+1}(u) = I \mid \mathbf{X}_t(u) = R]
        &= 0, \\[4pt]
    \mathbb{P}[\mathbf{X}_{t+1}(u) = R \mid \mathbf{X}_t(u) = R]
        &= 1.
\end{aligned}
\end{equation*}

Tak jak poprzednio rozważmy zmienne 
\[
    X_v=\min\{t\in\mathbb{N}:v\in\mathcal{I}_t\}.
\]
W modelu \textbf{SIR} również zachodzi $\mathbb{P}[X_v=\infty]>0$.
A więc poza rozkładem $X_v$ możemy wyznaczyć $\mathbb{E}[X_v|X_v<\infty]$.
Istnieje stanu $R$ narzuca pomysł rozważania podobnej zmiennej na pierwszy czas wyzdrowienia wierzchołka $v$.
Ale transmisja ze stanu $I$ do $R$ na pojedynczym wierzchołku jest rozkładem $\mathrm{Geo}(\alpha)$, a więc zmienna ta była by po prostu sumą rozkładu geometrycznego z $X_v$.
Z tego powodu tego nie będziemy jej rozważać. 
Zamiast rozważać liczbe tylko zainfekowanych lub tylko wyzdrowiałych wierzchołków będziemy rozważać liczbe nie podatnych wierzchołków po $t$ krokach.
Kładziemy więc
\[
    Y_t=|\mathcal{I}_t\cup\mathcal{R}_t|.
\]
Podobnie jak w \textbf{SIS} zdarzenie $\mathbb{P}[\mathcal{I}_t=\varnothing]\to1$ wraz z $t\to\infty$ a więc możemy zdefiniować zmienną $Z$ oznaczającą czas wygaśnięcia infekcji.
Dla uproszczenia będziemy rozważać moment w którym żaden nowy wierzchołek nie może byc zarażony.
A więc 
\[
    Z = \min\{t\in\mathbb{N}: \forall v\in\mathcal{I}_t \quad \mathcal{S}_t\cap\mathrm{N}(v)=\varnothing\}.
\]
Dodatkowo kładziemy zmienną $W$ będącą liczba finalnie wyzdrowiałych wierzchołków.
\[
    W = |\{v\in V : X_v < \infty\}|.
\]
Rozkłady $Z,W$ jak i ich wartości oczekiwane będą głownym zainteresowaniem w tym modelu.