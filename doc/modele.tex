

Dany jest graf spójny nieskierowany $G = (V, E)$. Propagacja na takim grafie jest procesem stochastycznym. Zakładamy, że czas dla tego procesu jest dyskretny i mierzony w jednostkach naturalnych, zatem za zbiór chwil przyjmujemy $\mathbb{N}$.  
Niech $\mathcal{Q}$ będzie skończonym zbiorem stanów, jakie mogą przyjmować wierzchołki $G$.  
W każdej chwili $t \in \mathbb{N}$ każdy wierzchołek $v \in V$ znajduje się w pewnym stanie $Q \in \mathcal{Q}$.  
Definiujemy zmienną losową $ \mathbf{X} : \mathbb{N}\times V \to \mathcal{Q} $,
taką, że $\mathbf{X}_t(v) = Q$ wtedy i tylko wtedy, gdy wierzchołek $v$ w chwili $t$ znajduje się w stanie $Q$.

\section{Model SI}

Model \textbf{Susceptible—Infected (SI)} opisuje propagację w sieci, w której każdy wierzchołek znajduje się w jednym z dwóch stanów: podatny ($S$) lub zainfekowany ($I$).  
Początkowo ustalony wierzchołek $s \in V$ znajduje się w stanie $I$, natomiast pozostałe wierzchołki są w stanie $S$. Mamy więc $\mathcal{Q} = \{S, I\}$.
W każdej jednostce czasu dowolny zainfekowany wierzchołek może zarazić każdego swojego sąsiada z prawdopodobieństwem $p$, dla ustalonego $p \in (0;1)$.  
Wierzchołek raz zainfekowany pozostaje w tym stanie na zawsze.  
W modelu \textbf{SI} liczba zainfekowanych wierzchołków jest funkcją niemalejącą w czasie.
Dla upraszczenia notacji kładziemy: 
\begin{itemize}
    \item $q=1-p$
    \item $\mathcal{S}_t=\{v\in V: \mathbf{X}_t(v) = S\}$
    \item $\mathcal{I}_t=\{v\in V: \mathbf{X}_t(v) = I\}$
\end{itemize}
Rozkład prawdopodobieństwa w tym modelu jest definiowany przez następujące zależności:
\[
\mathbf{X}_0(v) =
\begin{cases}
I, & \text{jeśli } v = s \\[4pt]
S, & \text{jeśli } v \neq s
\end{cases}
\]
\[
\begin{aligned}
\mathbb{P}[\mathbf{X}_{t+1}(u) = I \mid \mathbf{X}_t(u) = S]
 &= 1 - \prod_{v \in \mathrm{N}(u) \;\cap\; \mathcal{I}_t} q \\[6pt]
\mathbb{P}[\mathbf{X}_{t+1}(u) = S \mid \mathbf{X}_t(u) = S]
 &= \prod_{v \in \mathrm{N}(u) \;\cap\; \mathcal{I}_t} q \\[6pt]
\mathbb{P}[\mathbf{X}_{t+1}(u) = I \mid \mathbf{X}_t(u) = I]
 &= 1 \\[6pt]
\mathbb{P}[\mathbf{X}_{t+1}(u) = S \mid \mathbf{X}_t(u) = I]
 &= 0
\end{aligned}
\]
Zdefiniujmy teraz zmienne losowe opisujące istotne własności.
Dla każdego $v \in V$ definiujmy zmienną losową
\[
X_v = \min \{ t \in \mathbb{N} : \mathbf{X}_t(v) = I \}
\]
która określa pierwszą chwilę czasu zarażenia wierzchołka $v$.
Jeśli taka chwila nie istnieje (tzn.\ w danym przebiegu procesu wierzchołek $v$ nigdy się nie zarazi), to przyjmujemy $X_v = \infty$.
Zauważmy, że dla każdego $t \in \mathbb{N}$ zachodzi 
\[
    \mathbb{P}[\mathbf{X}_t(v) = I] = \mathbb{P}[X_v \le t]
\]
Następnie dla każdego $t\in\mathbb{N}$ definiujemy zmienną losową 
\[
    Y_t = |\mathcal{I}_t|
\]
oznaczającą liczbę zainfekowanych wierzchołków w chwili $t$. Dodatkowo definiujemy zmienną losową opisującą czas całkowitego zarażenia grafu:
\[
    Z = \max_{v \in V} X_v
\]
W modelu \textbf{SI} interesują nas następujące wielkości:
\begin{itemize}
    \item rozkład prawdopodobieństwa zmiennych $X_v$, $Y_t$ oraz $Z$
    \item wartości oczekiwane zmiennych, $\mathbb{E}[X_v]$, $\mathbb{E}[Y_t]$ oraz $\mathbb{E}[Z]$
    \item ograniczenia dolne, górne oraz asymptotyka powyższych wartości oczekiwanych kiedy wyznaczenie ich dokładnej wartości nie będzie możliwe
\end{itemize}



\section{Model SIS}

Model \textbf{Susceptible—Infected—Susceptible (SIS)} rozszerza model \textbf{SI} o powracanie wierzchołków zarażonych do stanu podatnego. Wierzchołek zainfekowany może powrócić do stanu podatnego z prawdopodobieństwem $\alpha \in (0;1)$. Tutaj mamy również $\mathcal{Q} = \{S, I\}$.
W modelu \textbf{SIS} liczba zainfekowanych wierzchołków może oscylować w czasie i nie musi osiągnąć stanu pełnego zakażenia. Dla upraszczenia notacji kładziemy $\beta=1-\alpha $.
Rozkład prawdopodobieństwa w tym modelu jest definiowany przez następujące zależności:
\[
\begin{aligned}
\mathbf{X}_0(v) =
\begin{cases}
I, & \text{jeśli } v = s \\[4pt]
S, & \text{jeśli } v \neq s
\end{cases} \\
\mathbb{P}[\mathbf{X}_{t+1}(u) = I \mid \mathbf{X}_t(u) = S]
 &= 1 - \prod_{v \in \mathrm{N}(u) \;\cap\; \mathcal{I}_t} q \\[6pt]
\mathbb{P}[\mathbf{X}_{t+1}(u) = S \mid \mathbf{X}_t(u) = S]
 &= \prod_{v \in \mathrm{N}(u) \;\cap\; \mathcal{I}_t} q \\[6pt]
\mathbb{P}[\mathbf{X}_{t+1}(u) = I \mid \mathbf{X}_t(u) = I]
 &= \beta \\[6pt]
\mathbb{P}[\mathbf{X}_{t+1}(u) = S \mid \mathbf{X}_t(u) = I]
 &= \alpha
\end{aligned}
\]


\section{Model SIR}

Model \textbf{Susceptible—Infected—Recovered (SIR)} rozszerza model \textbf{SI} o dodanie trzeciego stanu. Stanem tym jest $R$ (Recovered). Mamy zatem $\mathcal{Q} = \{S, I, R\}$.
Stan $R$ jest trwały — wierzchołek, który wyzdrowiał, nie może już ani się zarazić, ani nikogo zakazić. Zarażony wierzchołek może przejść z $I$ do stanu $R$ z prawdopodobieństwem $\gamma \in (0;1)$. Dla upraszczenia notacji kładziemy 
\begin{itemize}
    \item $\delta=1-\gamma$
    \item $\mathcal{R}_t=\{v\in V: \mathbf{X}_t(v) = R\}$
\end{itemize}
Rozkład prawdopodobieństwa w tym modelu jest definiowany przez następujące zależności:
\[
\begin{aligned}
\mathbf{X}_0(v) =
\begin{cases}
I, & \text{jeśli } v = s \\[4pt]
S, & \text{jeśli } v \neq s
\end{cases} \\
\mathbb{P}[\mathbf{X}_{t+1}(u) = I \mid \mathbf{X}_t(u) = S]
 &= 1 - \prod_{v \in \mathrm{N}(u) \;\cap\; \mathcal{I}_t} q \\[6pt]
\mathbb{P}[\mathbf{X}_{t+1}(u) = S \mid \mathbf{X}_t(u) = S]
 &= \prod_{v \in \mathrm{N}(u) \;\cap\; \mathcal{I}_t} q \\[6pt]
\mathbb{P}[\mathbf{X}_{t+1}(u) = R \mid \mathbf{X}_t(u) = I]
 &= \gamma \\[6pt]
\mathbb{P}[\mathbf{X}_{t+1}(u) = I \mid \mathbf{X}_t(u) = I]
 &= \delta \\[6pt]
\mathbb{P}[\mathbf{X}_{t+1}(u) = Q \mid \mathbf{X}_t(u) = R]
 &= 
\begin{cases}
1, & \text{dla } Q = R \\[4pt]
0, & \text{dla } Q \in \{S, I\}
\end{cases}
\end{aligned}
\]

