\section{Motywacja i zastosowania}

Propagację zjawisk w sieciach ludzkość obserwuje od starożytności — od epidemii chorób zakaźnych, przez rozprzestrzenianie się plotek i pogłosek, aż po współczesne wirusy komputerowe i kampanie marketingowe w mediach społecznościowych. 
Zrozumienie mechanizmów tej propagacji ma kluczowe znaczenie praktyczne: pozwala przewidywać rozwój epidemii, projektować skuteczne kampanie informacyjne, zabezpieczać infrastrukturę cyfrową przed atakami czy optymalizować algorytmy w sieciach rozproszonych.
Najbardziej naturalną reprezentacją matematyczną sieci relacji są grafy: wierzchołki reprezentują jednostki (ludzi, komputery, węzły sieci), a krawędzie określają połączenia między nimi (kontakty społeczne, łącza sieciowe, kanały komunikacji).
Połączenie teorii grafów z rachunkiem prawdopodobieństwa pozwala stworzyć precyzyjne modele stochastyczne, które uwzględniają losowy charakter propagacji — nie każdy kontakt prowadzi do transmisji, a różne jednostki mogą różnić się podatnością na zarażenie.
Modele propagacji znajdą zastosowanie w epidemiologii (modelowanie rozprzestrzeniania się chorób zakaźnych), cyberbezpieczeństwie (analiza rozprzestrzeniania się wirusów komputerowych), marketingu (viralowe kampanie reklamowe), teorii algorytmów (propagacja wartości ekstremalnych w sieciach rozproszonych) oraz naukach społecznych (dyfuzja innowacji, rozprzestrzenianie się opinii).


\section{Cel pracy}

Celem niniejszej pracy jest kompleksowa analiza probabilistycznych modeli propagacji w grafach, ze szczególnym uwzględnieniem wpływu topologii sieci na dynamikę procesów stochastycznych.
Konkretnie, praca ma na celu: 
\begin{itemize}
    \item zamodelowanie rozprzestrzeniania się informacji w sieciach przy użyciu procesów grafowych,
    \item teoretyczna analiza procesów losowej propagacji w grafach,
    \item teoretyczne wyznaczenie rozkładów prawdopodobieństwa i wartości oczekiwanych kluczowych zmiennych losowych dla wybranych rodzin grafów,
    \item przeprowadzenie symulacji komputerowych w celu weryfikacji wyników teoretycznych,
    \item zbadanie, jak topologia grafu wpływa na tempo i zasięg propagacji,
    \item przedstawienie praktycznego scenariusza aplikacyjnego ilustrującego użyteczność modeli.
\end{itemize}


\section{Zakres pracy}

Praca składa się z następujących części:

\textbf{Rozdział 2: Podstawy matematyczne.}
Zawiera notację używaną w całej pracy, definicje badanych rodzin grafów, podstawowe rozkłady prawdopodobieństwa wraz z ich własnościami, oraz zbiór faktów, sum i nierówności wykorzystywanych w dowodach i obliczeniach.

\textbf{Rozdział 3: Modele propagacji losowej.}
Wprowadza formalne definicje trzech badanych modeli: SI, SIR i SIS.
Dla każdego modelu definiujemy zmienne losowe opisujące kluczowe aspekty propagacji.
Przedstawiamy również diagramy przejść między stanami dla każdego modelu.

\textbf{Rozdział 4: Scenariusz aplikacyjny.}
Ilustruje praktyczne zastosowanie modeli na przykładzie rozprzestrzeniania się wirusa komputerowego w różnych topologiach sieciowych.
Pokazujemy, jak zmienne losowe zdefiniowane w Rozdziale 3 pozwalają na analizę ryzyka, planowanie zasobów i projektowanie odpornych architektur.

\textbf{Rozdział 5: Analiza modelu SI.}
Przeprowadzamy szczegółową analizę teoretyczną modelu SI dla wszystkich rodzin grafów ścieżkowych, gwiezdnych, cyklicznych, pełnych oraz drzew.
Wyznaczamy rozkłady prawdopodobieństwa i wartości oczekiwane zmiennych zdefiniowanych dla tego modelu.
Wyprowadzamy ogólne ograniczenia na czas zarażenia. 
Prezentujemy wyniki symulacji potwierdzające teoretyczne przewidywania.

\textbf{Rozdział 6: Rozkłady wymierające.}
Wprowadzamy rozkłady umierający geometryczny i umierający ujemny dwumianowy, które modelują propagację z możliwością przerwania procesu. 
Rozkłady te są fundamentem analizy modeli SIR i SIS, gdzie infekcja może wygasnąć z dodatnim prawdopodobieństwem.

\textbf{Rozdział 7: Analiza modelu SIR.}
Koncentruje się na modelach SIR dla grafów ścieżkowych i gwiazdowych.
Wyznaczamy rozkłady i wartości oczekiwane odpowiednich zmiennych losowych.
Wyniki weryfikujemy symulacjami.

\textbf{Rozdział 8: Analiza modelu SIS.}
W modelu SIS ograniczamy się do grafu pełnego, gdzie możliwość ponownej infekcji prowadzi do złożonej dynamiki oscylacyjnej.
Wykorzystujemy formalizm łańcuchów Markova do wyznaczenia interesujących nas rozkładów.
Przeprowadzamy symulacje dla różnych wartości parametrów pokazujące oscylacyjny charakter propagacji.

% \textbf{Rozdział 9: Extrema Propagation.}
% Przedstawia zastosowanie wyników z modelu SI do analizy algorytmu Basic Extrema Propagation, który wykorzystuje mechanizm propagacji do efektywnego obliczania wartości ekstremalnych w sieciach rozproszonych.