\section{Motywacja i zastosowania}
Propagację wirusów podczas epidemii ludzkość obserwowała już od starożytności. W dzisiejszych czasach, wraz z rozwojem internetu i mediów społecznościowych, mamy możliwość doświadczyć również dynamicznej propagacji informacji. Aby efektywnie rozprzestrzenić informacje, nie można robić tego ``na ślepo'', lecz trzeba wykorzystać wiedzę teoretyczną. Najbardziej naturalną metodą matematycznej reprezentacji relacji międzyludzkich są grafy: wierzchołkami grafu są ludzie, a krawędzie określają, czy dane osoby mają ze sobą kontakt. Połączenie teori grafów z rachunkiem prawdopodobieństwa pozwala stworzyć dokładny i praktyczny model propagacji informacji.

\section{Cel pracy}
Celem niniejszej pracy jest 
\begin{itemize}
    \item teoretyczna analiza procesów losowej propagacji w grafach,
    \item wyznaczenie rozkładu prawdopodobieństwa propagacji na wybranych rodzinach grafów,
    \item symulacja propagacji w środowisku komputerowym w celu zweryfikowania wyników teoretycznych.
\end{itemize}


\section{Zakres pracy}
Praca obejmuje:
\begin{itemize}
    \item wstęp teoretyczny z zakresu teorii grafów i rachunku prawdopodobieństwa,
    \item opis badanych modeli propagacji: SI, SIR, SIS,
    \item implementację symulacji w Pythonie,
    \item analizę wyników i wnioski dotyczące wpływu struktury grafu na propagację.
\end{itemize}
