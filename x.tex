\documentclass[a4paper,12pt]{book}

% ------------------------ Packages ------------------------
\usepackage[utf8]{inputenc}
\usepackage[T1]{fontenc}
\usepackage[polish]{babel}

\usepackage{indentfirst}
\setlength{\parindent}{0.7cm}

\frenchspacing

\usepackage{titlesec}
\usepackage{newtxtext}
\usepackage{amsmath}
\usepackage{amssymb}
\usepackage{amsthm}
\usepackage{mathtools}
\usepackage{algorithm}
\usepackage{algpseudocode}
\usepackage{graphicx}
\usepackage{caption}
\usepackage{subcaption}
\usepackage{tikz}
\usepackage{csquotes}
\usepackage{fancyhdr}
\usepackage[table, svgnames]{xcolor}

\usepackage{hyperref}
\hypersetup{
	colorlinks	= true,
	linkcolor	= blue,
    citecolor	= blue,
	urlcolor	= blue
}
\urlstyle{same}  

\usepackage[nameinlink,capitalise]{cleveref}  

% ------------------------ Bibliography ------------------------
\usepackage[backend=biber,style=numeric]{biblatex}
\addbibresource{../bibliography.bib}

% ------------------------ Counters ------------------------
\usepackage{chngcntr}
\counterwithin{figure}{chapter}
\renewcommand{\thefigure}{\thechapter.\arabic{figure}}

% ------------------------ Margins ------------------------
\usepackage[
    a4paper,
    top=25mm,
    bottom=25mm,
    inner=30mm,    % lewy margines (2.5cm + 0.5cm oprawy)
    outer=25mm,    % prawy margines
    bindingoffset=0mm  % już uwzględnione w inner
]{geometry}

\titleformat{\chapter}
  {\normalfont\bfseries\fontsize{14}{16}\selectfont} % pogrubione 14 pkt
  {\thechapter}{1em}{}

\titleformat{\section}
  {\normalfont\bfseries\fontsize{13}{15}\selectfont} % pogrubione 13 pkt
  {\thesection}{1em}{}

\newcommand{\db}[2]{{\color{red}{#2}}{\color{green}{#1}} {\color{teal}{DB}}} % komentarze Dominik Bojko

% ------------------------ Captions ------------------------
% Podpisy tabel i rysunków (10pt)
\captionsetup{
    font=small,           % 10pt
    labelfont=bf,         % pogrubiona etykieta
    skip=10pt,            % odstęp 10pt
    justification=centering
}

% Odstępy wokół floatów (10pt)
\setlength{\textfloatsep}{10pt plus 2pt minus 2pt}
\setlength{\floatsep}{10pt plus 2pt minus 2pt}
\setlength{\intextsep}{10pt plus 2pt minus 2pt}

% Styl dla pierwszych stron rozdziałów - OPCJONALNE
\fancypagestyle{chapterBeginStyle}{
	\fancyhf{}
	\fancyfoot[C]{\thepage}
	\renewcommand{\headrulewidth}{0pt}
	\renewcommand{\footrulewidth}{0pt}
}

% ------------------------ Theorems ------------------------------
\newtheorem{fact}{Fakt}
\newtheorem{theorem}{Twierdzenie}
\newtheorem{lemma}{Lemat}
\newtheorem{summ}{Suma}
\newtheorem{inequality}{Nierówność}

% Cleveref names
\crefname{theorem}{Twierdzenie}{Twierdzenia}
\Crefname{theorem}{Twierdzenia}{Twierdzenia}
\crefname{fact}{Fakt}{Faktu}
\Crefname{fact}{Faktu}{Faktu}
\crefname{lemma}{Lemat}{Lematu}
\Crefname{lemma}{Lematu}{Lematu}
\crefname{inequality}{Nierówność}{Nierówności}
\Crefname{inequality}{Nierówności}{Nierówności}
\crefname{summ}{Suma}{Sumy}
\Crefname{summ}{Sumy}{Sumy}
\crefname{figure}{Wykres}{Wykresie}
\Crefname{figure}{Wykresie}{Wykresie}
\crefname{algorithm}{Algorytm}{Algorytmu}
\Crefname{algorithm}{Algorytmu}{Algorytmu}

% ------------------------ Custom commands ------------------------
\makeatletter
\newcommand{\graphfamily}[1]{%
  \vspace{0.8em}%
  \noindent\textbf{#1}\par
  \vspace{0.3em}%
  \@afterheading%
}

\newcommand{\distribution}[1]{%
  \vspace{0.8em}%
  \noindent\textbf{#1}\par
  \vspace{0.3em}%
  \@afterheading%
}
\makeatother

% ------------------------ Algorithm Polish names ------------------
\makeatletter
\renewcommand{\ALG@name}{Algorytm}
\makeatother

\floatname{algorithm}{Algorytm}

% ------------------------ Document ------------------------


% \date{\today}

\author{Bartosz Łabuz}
\title{Probabilistyczne modele propagacji w grafach}
% \supervisor{dr Dominik Bojko}
% \keywords{keywords}


\begin{document}
\textbf{Kierunek: Informatyka Algorytmiczna (INA)}

\maketitle

\chapter*{Streszczenie}
\section*{Streszczenie}

Praca jest o propagacji w grafach.

\section*{Abstract}

This paper is about propagation in graphs.

\tableofcontents
\newpage

\chapter{Wstęp}
\section{Motywacja}

Propagację zjawisk w skupiskach ludzkich można zauważyć od zamierzchłych czasów.
Rozpowszechniały się przede wszystkim epidemie chorób zakaźnych, ale także wiadomości i
pogłoski. 
Współcześnie transmitowane są wirusy komputerowe oraz treści reklamowe i treści plotkarskie.
Zrozumienie mechanizmów działania propagacji ma podstawowe znaczenie w jej zastosowaniu. 
Pozwala przewidzieć rozwój epidemii, wprowadzać skuteczne działania: prowadzić szeroką kampanię informacyjną, zabezpieczać sieć cyfrową przed wirusami, czy wreszcie optymalizować algorytmy wspomagające działania.
Najbardziej obrazową metodą matematyczną do przedstawienia rozprzestrzeniania się relacji są grafy, w których wierzchołki reprezentują jednostki, czyli ludzi lub komputery, a krawędzie określają kontakty społeczne lub połączenia sieciowe.
Zastosowanie teorii grafów i rachunku prawdopodobieństwa pozwala stworzyć najlepsze modele stochastyczne, które uwzględniają losowy charakter propagacji. 
W przypadku epidemii należy pamiętać, że nie każdy kontakt prowadzi do transmisji, a poszczególne jednostki mają różną podatnością na zarażenie.
Modele propagacji znajdą zastosowanie przede wszystkim w epidemiologii, cyberbezpieczeństwie, marketingu, teorii algorytmów oraz w naukach społecznych.
Szczególną uwagę przyciąga ostatnia pandemia koronawirusa, kiedy to zwrócono się w stronę modeli matematycznych, celem prognozy zakażeń.


\section{Cel pracy}

Celem niniejszej pracy jest szeroka analiza probabilistycznych modeli propagacji przy użyciu grafów, ze szczególnym uwzględnieniem wpływu topologii sieci na dynamikę propagacji.
Szczegółowe cele pracy:
\begin{itemize}
    \item zamodelowanie rozprzestrzeniania się informacji w sieciach przy użyciu procesów grafowych,
    \item wyznaczenie rozkładów prawdopodobieństwa i wartości oczekiwanych kluczowych zmiennych losowych dla wybranych rodzin grafów,
    \item przeprowadzenie symulacji komputerowych w Pythonie metodą Monte Carlo w celu weryfikacji wyników teoretycznych,
    \item zbadanie, jak topologia grafu wpływa na tempo i zasięg propagacji,
    \item omówienie praktycznego scenariusza ilustrującego użyteczność wprowadzonych modeli.
\end{itemize}


\section{Zakres pracy}

Praca składa się z $9$ rozdziałów.

\textbf{Rozdział 1.}
Omawia motywację i cele pracy.

\textbf{Rozdział 2.}
Zawiera notację używaną w całej pracy, definicje analizowanych rodzin grafów, wykorzystywane rozkłady prawdopodobieństwa wraz z ich własnościami oraz zbiór faktów, sum i nierówności stosowanych w dowodach i obliczeniach.

\textbf{Rozdział 3.}
Wprowadza formalne definicje trzech badanych modeli: SI, SIR i SIS.
Ponadto dla każdego modelu zawiera opisy kluczowych zmiennych losowych.

\textbf{Rozdział 4.}
Ilustruje praktyczne zastosowanie modeli na przykładzie rozprzestrzeniania się wirusa komputerowego.

\textbf{Rozdział 5.}
Prezentuje szczegółową analizę modelu SI dla rodzin grafów ścieżkowych, gwiezdnych, cyklicznych, pełnych oraz drzew.
Wyprowadza ogólne ograniczenia na czas całkowitego zarażenia. 
Dodatkowo zawiera wyniki symulacji komputerowej.

\textbf{Rozdział 6.}
Wprowadza rozkłady ułatwiające modelowanie propagacji z możliwością jej przerwania, potrzebne do analizy modeli SIR oraz SIS.

\textbf{Rozdział 7.}
Przedstawia analizę modelu SIR dla rodzin grafów ścieżkowych i gwiezdnych oraz wyniki symulacji komputerowej.

\textbf{Rozdział 8.}
Przedstawia analizę modelu SIS dla grafów pełnych oraz wyniki symulacji komputerowej.

\textbf{Rozdział 9.}
Przedstawia podsumowanie pracy, wnioski i możliwości dalszych badań.

\chapter{Podstawy matematyczne}
\section{Notacja}

Przez $\mathbb{N}$ oznaczamy zbiór liczb naturalnych $\{0,1,2,\dots\}$, a przez $\mathbb{N}_+ = \{1,2,3,\dots\}$.  Moc zbioru $A$ oznaczamy $|A|$. Logarytm naturalny z $x$ oznaczamy $\log(x)$. 
Dla $n\in\mathbb{N}_+$ przez $H_n=1+\frac{1}{2}+\dots+\frac{1}{n}$ oznaczamy $n$'tą liczbę harmoniczną.

Niech $G=(V,E)$ będzie grafem prostym nieskierowanym. Stopień wierzchołka $v \in V$ oznaczamy $\deg(v)$. Zbiór sąsiadów $v\in V$ oznaczamy $\mathrm{N}(v)$. Odległość między $u$ i $v$ oznaczamy $\mathrm{d}(u,v)$ dla $u,v\in V$. Ekscentryczność $v\in V$ oznaczamy $\epsilon(v) = \max_{u\in V} \mathrm{d}(u,v)$. Przez $\delta(G)$ i $\Delta(G)$ oznaczamy odpowiednio minimalny i maksymalny stopień wierzchołka w grafie $G$.

Jeśli $\mathbb{P}$ jest miarą prawdopodobieństwa na przestrzeni $\Omega$ to prawdopodobieństwo zdarzenia $A$ oznaczamy $\mathbb{P}[A]$. Dla zmiennej losowej $X:\Omega\to\mathbb{R}$ jej wartość oczekiwaną oznaczamy $\mathbb{E}[X]$ a jej wariancje $\mathrm{Var}[X]$. Funckję masy prawdopodobieństwa oznaczamy $\mathbb{P}[X=t]$ a dystrybuante $X$ oznaczamy $F_X(t)$ dla $t\in\mathbb{R}$. Jeśli zmienne losowe $X_1,X_2,\dots, X_n$ są niezależne i o jednakowych rozkładach to mówimy, że są IID.


\section{Rodziny grafów}

\graphfamily{Graf ścieżkowy}
Dla $n \in \mathbb{N}_+$ graf ścieżkowy ma zbiór wierzchołków $V = \{1, 2, \dots, n\}$ oraz zbiór krawędzi $E = \{\{i, i+1\} : i \in \{1, 2, \dots, n-1\}\}$. Oznaczamy go przez $\mathrm{P}_n$.

\graphfamily{Graf gwiazda}
Dla $n \in \mathbb{N}_+$ graf gwiazda ma zbiór wierzchołków $V = \{0, 1, \dots, n\}$ oraz zbiór krawędzi $E = \{\{0, i\} : i \in \{1, 2, \dots, n\}\}$. Oznaczamy go przez $\mathrm{S}_n$.

\graphfamily{Graf pełny}
Dla $n \in \mathbb{N}_+$ graf pełny ma zbiór wierzchołków $V = \{1, 2, \dots, n\}$ oraz zbiór krawędzi $E = \{\{i, j\} : i, j \in \{1, 2, \dots, n\} \land i \ne j\}$. Oznaczamy go przez $\mathrm{K}_n$.

\graphfamily{Graf cykliczny}
Dla $n \in \mathbb{N}_+$ graf cykliczny ma zbiór wierzchołków $V = \{1, 2, \dots, n\}$ oraz zbiór krawędzi 
$E = \{\{i, i+1\} : i \in \{1, 2, \dots, n-1\}\} \cup \{\{n, 1\}\}$. Oznaczamy go przez $\mathrm{C}_n$.


\section{Rozkłady prawdopodobieństwa}

\distribution{Rozkład Bernoulliego}
Próba Bernoulliego to doświadczenie losowe, którego wynik może być jednym z dwóch:
\begin{itemize}
    \item sukces z prawdopodobieństwem  $p \in (0;1)$
    \item porażka z prawdopodobieństwem  $1 - p$
\end{itemize}  
Zmienna losowa $X$ przyjmująca wartość $1$ w przypadku sukcesu i $0$ w przypadku porażki ma rozkład Bernoulliego. Oznaczamy $X \sim \mathrm{Ber}(p)$.
 

\distribution{Rozkład dwumianowy}
Rozkład dwumianowy opisuje liczbę sukcesów w $n$ próbach Bernoulliego. Niech $X$ będzie zmienną losową przyjmującą wartości w $\{0,1,\dots,n\}$, a każda próba ma prawdopodobieństwo sukcesu $p \in (0;1)$.  
Wtedy:
\[
\mathbb{P}[X = k] = \binom{n}{k}p^k(1-p)^{n-k}, \quad k \in \{0,1,\dots,n\}.
\]
Wartość oczekiwana i wariancja mają postać:
\[
    \mathbb{E}[X] = np, \quad \mathrm{Var}[X] = np(1-p)
\]
Oznaczamy $X \sim \mathrm{Bin}(n,p)$.

\distribution{Rozkład geometryczny}
Rozkład geometryczny opisuje liczbę prób Bernoulliego potrzebnych do uzyskania pierwszego sukcesu.  
Niech $X$ będzie zmienną losową przyjmującą wartości w $\mathbb{N}_+$, a każda próba ma prawdopodobieństwo sukcesu $p \in (0;1)$.  
Wtedy:
\[
    \mathbb{P}[X = k] = p(1 - p)^{k-1}, \quad k \in \mathbb{N}_+.
\]
Dystrybuanta jest równa:
\[
    \mathbb{P}[X\le t] = 1 = (1-p)^t
\]
Wartość oczekiwana i wariancja mają postać:
\[
    \mathbb{E}[X] = \frac{1}{p}, \quad \mathrm{Var}[X] = \frac{1 - p}{p^2}
\]
Oznaczamy $X \sim \mathrm{Geo}(p)$.

\distribution{Rozkład ujemny dwumianowy}
Rozkład ujemny dwumianowy opisuje liczbę prób Bernoulliego potrzebnych do uzyskania $m$ sukcesów.  
Niech $X$ oznacza liczbę prób, przy czym każda próba ma prawdopodobieństwo sukcesu $p \in (0;1)$, a liczba sukcesów $m \in \mathbb{N}_+$ jest ustalona.  
Wtedy:
\[
\mathbb{P}[X = k] = \binom{k-1}{m-1} p^m (1 - p)^{k - m}, \quad k \ge m.
\]
Wartość oczekiwana i wariancja mają postać:
\[
    \mathbb{E}[X] = \frac{m}{p}, \quad \mathrm{Var}[X] = \frac{m(1 - p)}{p^2}
\]
Oznaczamy $X \sim \mathrm{NegBin}(m, p)$.

\section{Tożsamości i nierówności}

\begin{fact}\label{F:approximation_of_sum_by_an_integral}
Niech $a,b\in\mathbb{N}$, $a<b$ oraz $f:[a;b]\to\mathbb{R}$ będzie funkcją ciągłą i monotoniczą.
Jeśli $f$ jest rosnąca to
\[
    \int_{a}^b f(x)\; \mathrm{d}x \le \sum_{k=a}^{b} f(k)\le f(b) + \int_{a}^b f(x)\; \mathrm{d}x
\]
Jeśli $f$ jest malejąca to 
\[
    \int_{a}^b f(x)\; \mathrm{d}x \le \sum_{k=a}^{b} f(k)\le f(a) + \int_{a}^b f(x)\; \mathrm{d}x
\]
\end{fact}

\begin{fact}\label{F:harmonic_upper_bound}
Niech $n\in\mathbb{N}_+$. Wtedy
\[
    H_n \le 1 + \log(n)
\]
\end{fact}

\begin{fact}\label{F:log_vs_x}
Niech $x \in (0;1)$. Wtedy
\[
    \frac{1}{\log(\frac{1}{1-x})} \le \frac{1}{x}
\]
\end{fact}

\begin{fact}[Nierówność między średnimi]\label{F:AM_GM}
Niech $x_1,x_2,\dots,x_n\ge 0$. Wtedy
\[
    \sqrt[n]{x_1\cdots x_n} \le \frac{x_1  + \cdots + x_n}{n}
\]
Równoważnie możemy zapisać
\[
    \log(x_1\cdots x_n) \le n\cdot \log\left(\frac{x_1 + \cdots + x_n}{n}\right)
\]
\end{fact}

\begin{fact}\label{F:binomial_0}
Niech $n\in\mathbb{N}$ oraz $x,y\in\mathbb{R}$. Wtedy
\[
    \sum_{k=0}^{n} \binom{n}{k} x^k y^{n-k}= (x+y)^n
\]
\end{fact}

\begin{fact}\label{F:binomial_1}
Niech $n\in\mathbb{N}$ oraz $x,y\in\mathbb{R}$. Wtedy
\[
    \sum_{k=0}^{n} k\binom{n}{k} x^k y^{n-k} = nx(x+y)^{n-1}
\]
\end{fact}

\begin{fact}\label{F:max_CDF}
Niech $X_1,X_2,\dots, X_n:\Omega\to\mathbb{R}$ będą IID o CDF równej $F_X$. Zdefiniujmy zmienną losową $Y = \max\{X_1,X_2,\dots, X_n\}$. Wtedy 
\[
    F_Y(t)=F_X^n(t)
\]
\end{fact}

\begin{fact}\label{F:min_CDF}
Niech $X_1,X_2,\dots, X_n:\Omega\to\mathbb{R}$ będą IID o CDF równej $F_X$. Zdefiniujmy zmienną losową $Y = \min\{X_1,X_2,\dots, X_n\}$. Wtedy 
\[
    F_Y(t)=1-(1-F_X(t))^n
\]
\end{fact}

\begin{fact}\label{F:sum_of_geo_RV}
Niech $X_1, X_2, \dots, X_m$ będą niezależnymi zmiennymi losowymi o rozkładzie geometrycznym $\mathrm{Geo}(p)$ oraz $Y=X_1 + X_2 + \cdots + X_m$. Wtedy 
\[
    Y \sim \mathrm{NegBin}(m, p)
\]
\end{fact}


\begin{fact}\label{F:montonicity_of_expectation}
Niech $X,Y:\Omega\to\mathbb{R}$ będą zmiennymi losowymi takim, że dla każdego $\omega\in\Omega$ zachodzi $X(\omega)\le Y(\omega)$. Wtedy. 
\[
    \mathbb{E}[X] \le \mathbb{E}[Y]
\]
\end{fact}

\begin{fact}[Nierówność Jensena dla wartości oczekiwanej]\label{F:Jensen} 
Niech $n\in\mathbb{N}_+$ oraz $g:\mathbb{R}^n\to\mathbb{R}$ będzie funkcją wypukłą zaś $X_1,X_2,\dots, X_n:\Omega\to\mathbb{N}$ będą zmiennymi losowymi (niekoniecznie niezależnymi). Wtedy
\[
    g(\mathbb{E}[X_1],\dots, \mathbb{E}[X_n]) \le \mathbb{E}[g(X_1,\dots,X_n)]
\]
Jeśli $g$ jest wklęsła to nierówność zachodzi w drugą stronę.
\end{fact}

\chapter{Modele propagacji losowej}
Dany jest graf spójny nieskierowany $G = (V, E)$. 
Propagacja na takim grafie jest procesem stochastycznym. 
Zakładamy, że czas dla tego procesu jest dyskretny i mierzony w jednostkach naturalnych, zatem za zbiór chwil przyjmujemy $\mathbb{N}$.  
Niech $\mathcal{Q}$ będzie skończonym zbiorem stanów, jakie mogą przyjmować wierzchołki $G$.  
W każdej chwili $t \in \mathbb{N}$ każdy wierzchołek $v \in V$ znajduje się w pewnym stanie $Q \in \mathcal{Q}$.  
Nie będzie nas interesować przestrzeń zdarzeń elementarnych tego procesu.
Definiujemy realizacje zmiennej losową $\mathbf{X} : \mathbb{N}\times V \to \mathcal{Q}$ określającą stany wierzchołków grafu w czasie.
Mamy $\mathbf{X}_t(v) = Q$ wtedy i tylko wtedy, gdy wierzchołek $v$ w chwili $t$ znajduje się w stanie $Q$.  


\section{Model SI}

Model \textbf{Susceptible—Infected (SI)} opisuje propagację w sieci, w której każdy wierzchołek znajduje się w jednym z dwóch stanów: podatny ($S$) lub zainfekowany ($I$).  
Mamy więc $\mathcal{Q} = \{S, I\}$.  
Początkowo ustalone źródło $s \in V$ znajduje się w stanie $I$, natomiast pozostałe wierzchołki są w stanie $S$. 
A więc
\[
\mathbf{X}_0(v) =
\begin{cases}
I, & \text{jeśli } v = s,\\
S, & \text{jeśli } v \neq s.
\end{cases}
\]
W każdej jednostce czasu dowolny zainfekowany wierzchołek może zarazić każdego swojego sąsiada z prawdopodobieństwem $p$, dla ustalonego $p \in (0;1)$.  
Wierzchołek raz zainfekowany pozostaje w tym stanie na zawsze.  
W modelu SI liczba zainfekowanych wierzchołków jest funkcją niemalejącą w czasie.  
Dla uproszczenia notacji kładziemy 
\[
    q=1-p,\quad \mathcal{S}_t=\{v\in V: \mathbf{X}_t(v) = S\}, \quad \mathcal{I}_t=\{v\in V: \mathbf{X}_t(v) = I\}, \quad i_v=|\mathrm{N}(v) \cap \mathcal{I}_t|.
\]
Rozkład prawdopodobieństwa w tym modelu jest definiowany przez następujące zależności:
\begin{align*}
&\mathbb{P}[\mathbf{X}_{t+1}(v) = S \mid \mathbf{X}_t(v) = S] = q^{i_v}, \\[0.3em]
&\mathbb{P}[\mathbf{X}_{t+1}(v) = I \mid \mathbf{X}_t(v) = S] = 1 - q^{i_v}, \\[0.3em]
&\mathbb{P}[\mathbf{X}_{t+1}(v) = S \mid \mathbf{X}_t(v) = I] = 0, \\[0.3em]
&\mathbb{P}[\mathbf{X}_{t+1}(v) = I \mid \mathbf{X}_t(v) = I] = 1.
\end{align*}
Zdefiniujmy teraz zmienne losowe opisujące istotne dla nas własności.  
Dla każdego $v \in V$ definiujemy zmienną $X_v$ określającą chwilę czasu zarażenia wierzchołka $v$.  
Formalnie
\[
    X_v = \min \{ t \in \mathbb{N} : v \in \mathcal{I}_t \}.
\]
Jeśli taka chwila nie istnieje (tzn.\ w danym przebiegu procesu wierzchołek $v$ nigdy się nie zarazi), to przyjmujemy $X_v = \infty$.  
Później udowodnimy (\cref{theorem:total_infection}), że w modelu SI mamy $\mathbb{P}[X_v=\infty]=0$. 
Wyznaczenie rozkładu $X_v$ jak i $\mathbb{E}[X_v]$ da nam sporo informacji o propagacji na grafie w zależności od jego topologii.
Następnie dla każdego $t\in\mathbb{N}$ definiujemy zmienną losową $Y_t$ oznaczającą liczbę zainfekowanych wierzchołków w chwili $t$. 
Zatem
\[
    Y_t = |\mathcal{I}_t|.
\]
Interesować nas będzie rozkład prawdopodobieństwa $Y_t$ oraz wartość oczekiwana $\mathbb{E}[Y_t]$.
Pokażemy, że $\mathbb{E}[Y_t] \to |V|$ wraz z $t\to\infty$ (\cref{theorem:total_infection}). 
Dlatego też nie będziemy badać asymptotyki $\mathbb{E}[Y_t]$ względem $t$.
Dodatkowo definiujemy zmienną $Z$ opisującą czas całkowitego zarażenia grafu:
\[
    Z = \min \{ t \in \mathbb{N} : \mathcal{I}_t = V\}.
\]
Jeśli ta sytuacją nigdy nie nastąpi to $Z=\infty$.
Dla propagacji SI zainfekowanie całego grafu jest jednakże zdarzeniem pewnym. 
Wyznacznie rozkładu $Z$ oraz jej wartości oczekiwanej dla konkretnych rodzin grafów będzie głównym celem w tym modelu.


\begin{figure}[ht]
\centering 
\begin{tikzpicture}[scale=1.5, node distance=3cm]
    \node[circle, draw, thick, inner sep=8pt, minimum size=20pt] (S) {$S$};
    \node[circle, draw, thick, inner sep=8pt, minimum size=20pt] (I) [right of=S] {$I$};
    \draw[->, thick, loop above] (S) to node[above] {$q^{i_v}$} (S);
    \draw[->, thick, loop above] (I) to node[above] {$1$} (I);
    \draw[->, thick] (S) to node[above] {$1-q^{i_v}$} (I);
\end{tikzpicture}
\caption{Diagram przejść dla modelu SI.}
\end{figure}


\section{Model SIR}

Model \textbf{Susceptible—Infected—Recovered (SIR)} rozszerza model SI o dodanie trzeciego stanu.  
Stanem tym jest $R$ (Recovered). 
W tym modelu mamy $\mathcal{Q} = \{S, I, R\}$.  
Stan $R$ jest pochłaniający — wierzchołek, który wyzdrowiał, nie może już ani się zarazić, ani nikogo zakazić.  
Zarażony wierzchołek może przejść z $I$ do stanu $R$ z prawdopodobieństwem $\alpha \in (0;1)$.  
Przyjmujemy, że w każda runda odbywa się w dwóch etapach.
W pierwszym z nich wierzchołki przekazują infekcje.
W drugim z nich te, które były już wcześnie zainfekowane mogą wyzdrowieć.
Dla uproszczenia notacji kładziemy 
\[
    \beta=1-\alpha,\quad \mathcal{R}_t=\{v\in V: \mathbf{X}_t(v) = R\}.
\]
Rozkład prawdopodobieństwa w tym modelu jest definiowany przez następujące zależności:
\begin{align*}
&\mathbb{P}[\mathbf{X}_{t+1}(v) = S \mid \mathbf{X}_t(v) = S] = q^{i_v}, \\[0.3em]
&\mathbb{P}[\mathbf{X}_{t+1}(v) = I \mid \mathbf{X}_t(v) = S] = 1 - q^{i_v}, \\[0.3em]
&\mathbb{P}[\mathbf{X}_{t+1}(v) = R \mid \mathbf{X}_t(v) = S] = 0, \\[0.3em]
&\mathbb{P}[\mathbf{X}_{t+1}(v) = S \mid \mathbf{X}_t(v) = I] = 0, \\[0.3em]
&\mathbb{P}[\mathbf{X}_{t+1}(v) = I \mid \mathbf{X}_t(v) = I] = \beta, \\[0.3em]
&\mathbb{P}[\mathbf{X}_{t+1}(v) = R \mid \mathbf{X}_t(v) = I] = \alpha, \\[0.3em]
&\mathbb{P}[\mathbf{X}_{t+1}(v) = S \mid \mathbf{X}_t(v) = R] = 0, \\[0.3em]
&\mathbb{P}[\mathbf{X}_{t+1}(v) = I \mid \mathbf{X}_t(v) = R] = 0, \\[0.3em]
&\mathbb{P}[\mathbf{X}_{t+1}(v) = R \mid \mathbf{X}_t(v) = R] = 1.
\end{align*}
Tak jak poprzednio rozważmy zmienne 
\[
    X_v=\min\{t\in\mathbb{N}:v\in\mathcal{I}_t\}.
\]
W kontraście dla modelu SI zachodzi $\mathbb{P}[X_v=\infty]>0$ (\cref{theorem:infection_dies_out_SIR}).
Z tego powodu wartość oczekiwana $\mathbb{E}[X_v]=\infty$ niezależnie od struktury grafu.
Dlatego też poza rozkładem $X_v$ możemy wyznaczyć $\mathbb{E}[X_v|X_v<\infty]$.
Istnienie stanu $R$ narzuca pomysł rozważania podobnej zmiennej na pierwszy czas wyzdrowienia wierzchołka $v$.
Ale transmisja ze stanu $I$ do $R$ na pojedynczym wierzchołku jest rozkładem $\mathrm{Geo}(\alpha)$, a więc zmienna ta była by po prostu sumą rozkładu geometrycznego oraz $X_v$.
Z tego powodu tego nie będziemy jej rozważać. 
Zamiast rozważać liczbę tylko zainfekowanych lub tylko wyzdrowiałych wierzchołków będziemy rozważać liczbe niepodatnych wierzchołków po $t$ krokach.
Kładziemy więc
\[
    Y_t=|\mathcal{I}_t\cup\mathcal{R}_t|.
\]
Mamy $\mathbb{P}[\mathcal{I}_t=\varnothing]\to1$ wraz z $t\to\infty$ (\cref{theorem:infection_dies_out_SIR}) a więc możemy zdefiniować zmienną $Z$ oznaczającą czas wygaśnięcia infekcji.
Dla uproszczenia będziemy rozważać moment w którym żaden nowy wierzchołek nie może być zarażony.
Zatem
\[
    Z = \min\{t\in\mathbb{N}: \forall v\in\mathcal{I}_t \quad \mathcal{S}_t\cap\mathrm{N}(v)=\varnothing\}.
\]
Dodatkowo definiujemy zmienną $W$ będącą liczba finalnie wyzdrowiałych wierzchołków.
\[
    W = |\{v\in V : X_v < \infty\}|.
\]
Rozkłady $Z,W$ jak i ich wartości oczekiwane będą naszym głównym obiektem zainteresowań w tym modelu.

\begin{figure}[ht]
\centering 
\begin{tikzpicture}[scale=1.5, node distance=3cm]
    \node[circle, draw, thick, inner sep=8pt, minimum size=20pt] (S) {$S$};
    \node[circle, draw, thick, inner sep=8pt, minimum size=20pt] (I) [right of=S] {$I$};
    \node[circle, draw, thick, inner sep=8pt, minimum size=20pt] (R) [right of=I] {$R$};
    \draw[->, thick, loop above] (S) to node[above] {$q^{i_v}$} (S);
    \draw[->, thick, loop above] (I) to node[above] {$\beta$} (I);
    \draw[->, thick, loop above] (R) to node[above] {$1$} (R);
    \draw[->, thick] (S) to node[above] {$1-q^{i_v}$} (I);
    \draw[->, thick] (I) to node[above] {$\alpha$} (R);
\end{tikzpicture}
\caption{Diagram przejść dla modelu SIR.}
\end{figure}


\section{Model SIS}

Model \textbf{Susceptible—Infected—Susceptible (SIS)} rozszerza model SI o powracanie wierzchołków zarażonych do stanu podatnego.  
Wierzchołek zainfekowany może powrócić do stanu podatnego z prawdopodobieństwem $\alpha \in (0;1)$.  
Tutaj tak jak w SI mamy $\mathcal{Q} = \{S, I\}$.  
W modelu SIS liczba zainfekowanych wierzchołków może oscylować w czasie i nie musi osiągnąć stanu pełnego zakażenia.  
Dla uproszczenia notacji kładziemy $\beta=1-\alpha$.  
Każda runda składa się z dwóch faz.
W pierwszej z nich wierzchołki zarażone próbują przekazać infekcje.
Natomiast w drugiej te zarażone wierzchołki, które były juz zarażone przez fazą pierwszą, mogą powrócić w stan podatności
Jest to odzwierciedlenie naturalnego stanu rzeczy.
Jeśli ktoś zostanie zakażony chorobą to nie może wyzdrowieć szybciej niż jego zakaziciel.  
Rozkład prawdopodobieństwa w tym modelu jest definiowany przez następujące zależności:
\begin{align*}
&\mathbb{P}[\mathbf{X}_{t+1}(v) = S \mid \mathbf{X}_t(v) = S] = q^{i_v}, \\[0.3em]
&\mathbb{P}[\mathbf{X}_{t+1}(v) = I \mid \mathbf{X}_t(v) = S] = 1 - q^{i_v}, \\[0.3em]
&\mathbb{P}[\mathbf{X}_{t+1}(v) = S \mid \mathbf{X}_t(v) = I] = \alpha, \\[0.3em]
&\mathbb{P}[\mathbf{X}_{t+1}(v) = I \mid \mathbf{X}_t(v) = I] = \beta.
\end{align*}
Zmienne losowe opisujące istotne własności są tutaj podobne jak w modelu SI.  
Dla każdego $v \in V$ kładziemy
\[
X_v = \min \{ t \in \mathbb{N} : v \in \mathcal{I}_t \},
\]
oraz dla $t\in\mathbb{N}$ definiujemy 
\[
    Y_t = |\mathcal{I}_t|.
\]
Wygaśnięcie infekcji zachodzi z prawdopodobieństwem $1$ (\cref{theorem:infection_dies_out_SIS}).
A więc definiujemy zmienną losową opisującą czas wygaśnięcia infekcji:
\[
    Z = \min \{ t \in \mathbb{N} : \mathcal{I}_t = \varnothing\}.
\]
Mamy $\mathbb{P}[X_v=\infty] > 0$ a zatem interesować nas będzie $\mathbb{E}[X_v|X_v<\infty]$.
Dalej wykażemy, że $\mathbb{E}[Y_t]$ dążą do $0$. (\cref{theorem:infection_dies_out_SIS})
Dlatego też także nie będą nas interesować.
Skupimy się tylko na rozkładzie $Y_t$.
Jeśli chodzi o zmienną $Z$ to przyjrzyjmy się zarówno jej rozkładowi jak i wartości oczekiwanej.

\begin{figure}[ht]
\centering 
\begin{tikzpicture}[scale=1.5, node distance=3.5cm]
    \node[circle, draw, thick, inner sep=8pt, minimum size=20pt] (S) {$S$};
    \node[circle, draw, thick, inner sep=8pt, minimum size=20pt] (I) [right of=S] {$I$};
    \draw[->, thick, loop above] (S) to node[above] {$q^{i_v}$} (S);
    \draw[->, thick, loop above] (I) to node[above] {$\beta$} (I);
    \draw[->, thick, bend left=30] (S) to node[above] {$1-q^{i_v}$} (I);
    \draw[->, thick, bend left=30] (I) to node[below] {$\alpha$} (S);
\end{tikzpicture}
\caption{Diagram przejść dla modelu SIS.}
\end{figure}


\chapter{Scenariusz aplikacyjny}

\section{Basic Extrema Propagation}

Wyniki i ustalenia modelu \textbf{SI} możemy wykorzystać dla analizy teoretycznej różnych metod propagacji.
Skoncentrujmy naszą uwagę na algorytmie Basic Extrema Propagation, \textbf{BEP} (\ref{algo:BEP}).

\begin{algorithm}
\caption{Basic Extrema Propagation}
\begin{algorithmic}[1]  
\State\textbf{Input:} Graph $G=(V,E)$,  random function $\text{Rand}$
\State\textbf{Output:} Values ${(\mathcal{X}_v)}_{v\in V}$
\For{$v \in V$}
    \State$\mathcal{X}_v \gets \text{Rand}()$ 
\EndFor%
\State$\text{changed} \gets \text{true}$
\While{$\text{changed}$}
    \State$\text{changed} \gets \text{false}$
    \For{$v \in V$}
        \State$\mathcal{X}_{\text{new}} \gets \mathcal{X}_v$
        \State\text{send} $\mathcal{X}_v$ \text{to every} $u\in\text{N}(v)$
        \State\text{receive} $\mathcal{X}_u$ \text{from every} $u\in\text{N}(v)$
        \State$\mathcal{X}_{\text{new}} \gets \min\{x_{\text{new}}, \{\mathcal{X}_u : u \in \text{N}(v)\}\}$
        \If{$\mathcal{X}_{\text{new}} < \mathcal{X}_v$}
            \State$\mathcal{X}_v \gets \mathcal{X}_{\text{new}}$
            \State$\text{changed} \gets \text{true}$
        \EndIf%
    \EndFor%
\EndWhile%
\State\textbf{return} ${(\mathcal{X}_v)}_{v\in V}$
\end{algorithmic}%
\label{algo:BEP}
\end{algorithm}

Jest to algorytm rozproszony zaprojektowany do obliczania wartości ekstremalnych (np.\ minimalnych lub maksymalnych) w sieciach przy użyciu wyłącznie lokalnych interakcji wierzchołków.
Algorytm ten działa następująco:
Każdy wierzchołek ustala wartość początkową losowo, korzystając z funkcji $\mathrm{Rand}$.
Następnie w synchronicznych rundach węzły przesyłają aktualnie przechowywaną wartość do swoich sąsiadów.
Potem odbierają wszystkie wysłane im wartości i obliczają minimum z nich oraz już posiadanej wartości.
Proces ten po pewnym czasie stabilizuje się, to jest każdy wierzchołek posiada tą samą wartość i dalsza propagacja nie ma sensu.
Wtedy algorytm kończy działanie. 
Jeśli wierzchołki mają początkowo wartości ${(\mathcal{X}_v)}_{v\in V}$ to zwrócona wartość wynosi $\mathcal{X}=\min_{v\in V} \mathcal{X}_v$.
Możemy założyć, że ${(\mathcal{X}_v)}_{v\in V}$ są niezależne i mają taki sam rozkład.
Dobierając odpowiedni rozkład początkowy wierzchołków możemy w wyniku algorytmu uzyskać różne interesujące wartości.
Przykładowo dla $\mathcal{X}_v\sim\mathcal{U}[0;1]$ mamy $\mathbb{E}[\mathcal{X}]=\frac{1}{1+|V|}$ (\cref{fact:min_of_uniforms}).
Daje nam to możliwość łatwego oszacowania na wielkości sieci.
Gdy zaś $\mathcal{X}_v\sim\mathrm{Exp}(\lambda_v)$ to wtedy $\mathcal{X}\sim\mathrm{Exp}(\Lambda)$ przy czym $\Lambda=\sum_{v\in V}\lambda_v$ (\cref{facf:min_of_exponentials}).
Ponadto $\mathbb{E}[\mathcal{X}]=\frac{1}{\Lambda}$.
Tego rozkładu startowego możemy użyć jeśli chcemy wyłuskać sumę pewnych parametrów przechowywanych w węzłach sieci.
Interesuje nas też oczywiście czas działania algorytmu \textbf{BEP}.
Można pokazać~\cite{OnMessageComplexityOfExtremaPropagationTechniques}, że w dowolnym grafie średnia liczba wiadomości wysłanych przez dowolny wierzchołek podczas całego procesu jest rzędu $\mathcal{O}(\log(|V|))$.
W teorii każde poinformowanie zawsze sie powiedzie.
Stabilizacja wartości przechowywanej w  $u\in V$ nastąpi, kiedy $u$ prześle informacje do najdalej oddalonego od niego wierzchołka, to jest po $\epsilon(u)$ rundach.
Oczywiście cały proces zakończy się po najdłuższym z tych czasów.
A zatem algorytm po co najwyżej $\mathrm{diam}(G)$ rundach kończy działanie.
W praktyce natomiast ze względu na szumy i zakłócenia komunikacji przesłanie wiadomości nie zawsze się udaje.
Możemy założyć, że prawdopodobieństwo przesłania informacji w pojedynczej rundzie wynosi $p$.
Krok aktualizacji wartości jest analogiczny do rozprzestrzeniania się infekcji w modelu \textbf{SI}.
Dla ustalonego $s\in V$ zmienne losowe, które zdefiniowaliśmy dostarczają nam sporo informacji o przebiegu algorytmu \textbf{BEP}.
Najistotniejsza jest zmienna $Z$, która jest górnym ograniczeniem na czas zakończenia propagacji.

\chapter{Analiza modelu SI}
\section{Dwa wierzchołki, jedna krawędź}

Na samym początku rozważmy najprostszy graf, czyli taki o dwóch wierzchołkach $u, v$ połączonych krawędzią. 
Za wierzchołek startowy wybierzmy $u$. 
Istnieją tylko dwa możliwe stany systemu: $\mathcal{I}_t=\{u\}$ oraz $\mathcal{I}_t=\{u,v\}$. Przejście ze stanu $\mathcal{I}_t=\{u\}$ do $\mathcal{I}_t=\{u,v\}$ następuje z prawdopodobieństwem $p$ w każdej jednostce czasu. 
Zatem czas zarażenia drugiego wierzchołka $X_v$ ma rozkład geometryczny, $X_v \sim \mathrm{Geo}(p)$.

Rozważmy teraz rozkład $Y_t$. 
Mamy $\mathbb{P}[Y_t=1]=q^t$, ponieważ próba zarażenia musiałaby nie udać się $t$ razy, oraz $\mathbb{P}[Y_t=2]=1-q^t$. 
Stąd $\mathbb{E}[Y_t]=1\cdot q^t + 2 \cdot (1-q^t) = 2-q^t$.

Jeśli chodzi o zmienną $Z$, to zachodzi $Z=\max\{X_u,X_v\}=X_v$, a więc również $Z\sim \mathrm{Geo}(p)$ oraz $\mathbb{E}[Z]=\frac{1}{p}$.


\section{Trójkąt}

Przyjrzyjmy się teraz nieco większemu grafowi — trójkątowi. 
Niech jeden z wierzchołków będzie źródłem $s$, a pozostałe $u, v$. 
Aby poinformować $u$, musimy uzyskać sukces bezpośrednio od $s$ lub zarazić $v$, a następnie $u$. Możemy więc zapisać $X_u = \min\{A,B\}$, gdzie $A\sim \mathrm{Geo}(p)$ oraz $B\sim \mathrm{NegBin}(2,p)$. 
Wiemy, że $\mathbb{P}[A\le t] =1-q^t$. 
Z kolei
\begin{equation*}
\begin{aligned}
\mathbb{P}[B\le t]
&= \sum_{k=2}^{t} (k-1)\cdot p^2q^{k-2}
= \frac{p^2}{q}\cdot \frac{q}{{(1-q)}^2}\cdot((t-1)q^t-tq^{t-1}+1) \\
&=1-q^t-tpq^{t-1},
\end{aligned}
\end{equation*}
gdzie skorzystaliśmy z \Cref{summ:geo_1}. 
Dalej, z \Cref{fact:min_CDF} mamy
\[
    \mathbb{P}[X_u\le t] = 1 - (1-(1-q^t))\cdot (1-(1-q^t-tpq^{t-1})) = 1-q^{2t}-tpq^{2t-1}.
\]

Jeśli chodzi o liczbę zainfekowanych po $t$ krokach, to skoro mamy trzy wierzchołki, mamy też trzy wartości do policzenia. 
Oczywiście $\mathbb{P}[Y_t=1]=q^{2t}$. 
Aby po $t$ chwilach tylko dwa węzły były zainfekowane, musimy zarazić któryś z wierzchołków po $1\le k\le t$ rundach z prawdopodobieństwem $2pq\cdot q^{2\cdot(k-1)}$, a następnie uzyskać $t-k$ porażek. Na każdą z nich mamy szansę równą $q^2$. 
Podsumowując:
\[
    \mathbb{P}[Y_t=2]=\sum_{k=1}^{t} 2pq\cdot q^{2\cdot(k-1)} \cdot q^{2\cdot(t-k)} = 2tpq^{2t-1}.
\]
Na koniec mamy $\mathbb{P}[Y_t=3]=1-q^{2t}-2tpq^{2t-1}$. 
Ponadto:
\[
    \mathbb{E}[Y_t] = 1 \cdot q^{2t} + 2 \cdot 2tpq^{2t-1} + 3\cdot (1-q^{2t}-2tpq^{2t-1}) = 3-2q^{2t}-2tpq^{2t-1}.
\]
Widzimy, że $\mathbb{E}[Y_t] \to 3$ przy $t\to \infty$, co jest zgodne z intuicją.

Propagacja może się zakończyć na dwa sposoby. 
Pierwszy z nich to sytuacja, w której przez $t-1$ jednostek czasu żadne zakażenie nie zaszło, a w chwili $t$ zarażają się oba wierzchołki. 
Prawdopodobieństwo tego przypadku wynosi $p^2q^{2\cdot(t-1)}$. 
Druga możliwość to taka, w której w $k$-tym kroku (dla $1\le k\le t-1$) zaraził się jeden z wierzchołków — z prawdopodobieństwem $2pq\cdot q^{2\cdot(k-1)}$, a potem przez kolejne $t-1-k$ kroków trzeci wierzchołek nie został zainfekowany, na co mamy prawdopodobieństwo ${(q^2)}^{t-1-k}$, aż do chwili $t$. 
To ostatnie przejście ma $1-q^2$ szans. 
Łącznie dostajemy:
\begin{equation*}
\begin{aligned}
\mathbb{P}[Z=t]
&=p^2q^{2t-2}+\sum_{k=1}^{t-1} (2pq q^{2k-2})\cdot (q^{2t-2k-2}) \cdot (1-q^2) \\
&=p^2q^{2t-2}+2pq^{2t-3}\cdot(t-1)(1-q^2).
\end{aligned}
\end{equation*}
Mamy też $\mathbb{P}[Z>t]=\mathbb{P}[Y_t\ne 3] = q^{2t}+2tpq^{2t-1}$. 
Wartość oczekiwana wynosi więc:
\begin{equation*}
\begin{aligned}
\mathbb{E}[Z]
&=\sum_{t=0}^{\infty} \mathbb{P}[Z>t]=\sum_{t=0}^{\infty} q^{2t}+2tpq^{2t-1} \\
&=\frac{1}{1-q^2}+\frac{2p}{q}\cdot \frac{q^2}{{(1-q^2)}^2}
=\frac{-3q^2+2q+1}{{(1-q^2)}^2}
=\frac{4-3p}{p{(2-p)}^2}.
\end{aligned}
\end{equation*}
(patrz \cref{summ:geo_0_inf} oraz \cref{summ:geo_1_inf}).
Wykonaliśmy dość sporo obliczeń jak na tak mały graf. 
Możemy więc zauważyć, że istnienie cykli w grafie znacząco komplikuje sytuację, jeśli chodzi o model \textbf{SI}.\@

\section{Całkowita infekcja pewna}

Dość intuicyjny jest fakt, że każdy wierzchołek zostanie kiedyś zarażony z prawdopodobieństwem $1$.
Co za tym idzie cały graf znajdzie się w stanie $I$ oraz $Z<\infty$.
Postaramy się teraz formalnie tego dowieść.

\begin{theorem}\label{theorem:total_infection}
Niech $G=(V,E)$ będzie grafem spójnym, $s\in V$ źródłem infekcji oraz $v\in V\setminus\{s\}$.
Wtedy zachodzą następujące tożsamości:
\[
    \mathbb{P}[X_v = \infty] = 0, \quad \lim_{t\to\infty} \mathbb{E}[Y_t] = |V|, \quad \mathbb{P}[Z < \infty] = 1.
\]   
\end{theorem}

\begin{proof}
Weźmy $v\in V\setminus\{s\}$.
Wybierzemy ustaloną ścieżkę od $s$ do $v$, $s = u_0u_1u_2\dots u_\ell = v$ gdzie $\ell$ to długość tej ścieżki ($\ell\ge 1$).
Aby $v$ został zainfekowane, wystarczy, że po kolei na tej ścieżce zajdą sukcesy transmisji.
Rozważmy następujące zdarzenia zainfekowania tych wierzchołków w bloku czasu indeksowanym przez $k\in\mathbb{N}$:
\[
    A_k = \{\forall j\in\{1,\dots,\ell\}\quad X_{u_j}=k\ell+j\}.
\]
Próby transmisji w różnych rundach są niezależne a więc $\mathbb{P}[A_k]=p^\ell > 0$.
Jeśli chociaż jedno ze zdarzeń $A_k$ zajdzie to wtedy $v\in\mathcal{I}_t$ dla pewnego $t$.
Mamy zatem
\begin{equation*}
\begin{aligned}
\mathbb{P}[X_v=\infty] &\le \mathbb{P}\Big[\bigcap_{k=0}^{\infty} A_k^c\Big] = \lim_{m\to \infty} \mathbb{P}\Big[\bigcup_{k=0}^m A_k^c\Big] \\
&=\lim_{m\to\infty} \prod_{k=0}^{m} 1-p^\ell = \lim_{m\to\infty} (1-p^\ell)^{m+1}=0.
\end{aligned}
\end{equation*}
Stąd oczywiscie $\mathbb{P}[X_v=\infty]=0$.
Dalej zauważmy, że $0\le Y_t\le |V|$. 
Z Twierdzenia Lebesgue o zbieżności ograniczonej mamy 
\[
    \lim_{t\to\infty} \mathbb{E}[Y_t]=\mathbb{E}[\lim_{t\to\infty} Y_t] = \mathbb{E}[\lim_{t\to\infty} \sum_{v\in V} \mathbf{1}_{v\in\mathcal{I}_t}]=\mathbb{E}[\lim_{t\to\infty}\sum_{v\in V} 1 ] = \mathbb{E} [|V|] = |V|.
\]
Wreszcie 
\[
    \mathbb{P}[Z = \infty] =  \mathbb{P}[\max_{v\in V} X_v = \infty] = 0.
\]
Stąd $\mathbb{P}[Z<\infty]=1$.
\end{proof}


\section{Grafy ścieżkowe}

Jako pierwszą rodzinę grafów rozważmy grafy ścieżkowe $\mathrm{P}_n$.  
Załóżmy, że proces zaczyna się w wierzchołku $s=1$.  
Zatem infekcja rozchodzi się po grafie „od lewej do prawej”.  
Dla tej rodziny grafów uda nam się wyznaczyć dokładny rozkład prawdopodobieństwa.  
Zauważmy, że czasy zarażenia kolejnych wierzchołków tworzą ciąg zmiennych losowych:
\[
X_1 = 0, \quad X_{v} = X_{v-1} + U_k, \quad v\in\{2,3,\dots,n\},
\]
gdzie $U_2,U_3,\dots,U_n \sim \mathrm{Geo}(p)$ oraz $U_2,U_3,\dots,U_n$ są niezależne.  
Widzimy zatem, że $X_v = U_1 + U_2 + \cdots + U_{v-1}$, a więc z \Cref{fact:sum_of_geo_RV} $X_v$ ma rozkład ujemny dwumianowy:
\[
X_v\sim \mathrm{NegBin}(v-1, p).
\]
Ponadto mamy:
\[
    \mathbb{E}[X_v] = \frac{v-1}{p}, \quad \mathrm{Var}[X_v] = \frac{(v-1)(1-p)}{p^2}.
\]

Ustalmy $t\in\mathbb{N}$ i przejdźmy do obliczania rozkładu $Y_t$.  
Zauważmy, że liczba dodatkowych zakażeń poza startowym wierzchołkiem do czasu $t$ to po prostu liczba sukcesów w $t$ niezależnych prób Bernoulliego.  
Musimy jednak pamiętać, że $Y_t$ nie może przekroczyć $n$.  
Zatem mamy dokładnie:
\[
Y_t = \min\{n, 1 + B_t\}, \quad \quad B_t \sim \mathrm{Bin}(t,p).
\]
Pozwala to na wyznaczenie PMF dla $Y_t$.  
Dla $1 \le k \le n-1$ mamy:
\[
\mathbb{P}[Y_t=k] = \mathbb{P}[B_t=k-1] = \binom{t}{k-1} p^{k-1} q^{t-k+1},
\]
oraz dla $k = n$:
\[
\mathbb{P}[Y_t=n] = \mathbb{P}[B_t \ge n-1] = \sum_{j=n-1}^{t} \binom{t}{j} p^j q^{t-j}.
\]
Przejdźmy teraz do obliczania wartości oczekiwanej $Y_t$:
\begin{equation*}
\begin{aligned}
\mathbb{E}[Y_t] 
&= \sum_{k=1}^{n-1} k \cdot \mathbb{P}[Y_t=k] + n \cdot \mathbb{P}[Y_t=n] \\
&= \sum_{k=1}^{n-1} k \cdot \binom{t}{k-1} p^{k-1} q^{t-k+1} 
   + n \cdot \sum_{j=n-1}^{t} \binom{t}{j} p^j q^{t-j} \\
&= \sum_{j=0}^{t} \min\{n, 1+j\} \cdot \binom{t}{j} p^j q^{t-j}.
\end{aligned}
\end{equation*}

Policzmy teraz asymptotykę dla $n \to \infty$.  
Wtedy $n > 1 + j$ dla wszystkich $0 \le j \le t$, a więc:
\begin{equation*}
\begin{aligned}
\lim_{n \to \infty} \mathbb{E}[Y_t] 
    &= \sum_{j=0}^{t} (1+j)\binom{t}{j} p^j q^{t-j} = \sum_{j=0}^{t} \binom{t}{j} p^j q^{t-j} 
       + \sum_{j=0}^{t} j \binom{t}{j} p^j q^{t-j} \\
    &= {(p+q)}^t + t p {(p+q)}^{t-1} = 1 + t p,
\end{aligned}
\end{equation*}
gdzie sumy obliczamy korzystając z \Cref{summ:binomial_0} oraz \Cref{summ:binomial_1}.  
Stąd:
\[
    \lim_{n \to \infty}\mathbb{E}[Y_t] = 1+tp.
\]

Czas całkowitego zainfekowania grafu $\mathrm{P}_n$ to $Z = \max\{X_1,X_2,\dots,X_n\} = X_n$.  
Zatem rozkład zmiennej $Z$ jest już nam znany: $Z\sim\mathrm{NegBin}(n-1,p)$, a wartość oczekiwana wynosi:
\[
    \mathbb{E}[Z]=\frac{n-1}{p}.
\]

Sprawdźmy, czy nasze obliczenia teoretyczne zgadzają się z empirycznie wyznaczonymi wartościami. 
Ustalmy $p=0.2$. 
W celu estymacji $\mathbb{E}[Y_t]$ przeprowadźmy $2000$ symulacji propagacji na grafie $\mathrm{P}_n$. 
Następnie dla $n\in\{1,2,\dots,1000\}$ tą samą liczbą symulacji oszacujmy $\mathbb{E}[Z]$.

\begin{figure}[ht!]
    \centering
    \includegraphics[width=1\textwidth]{../img/path/final_infection_expectations.png}
    \caption{$\mathbb{E}[Y_t]$ dla $\mathrm{P_n}$ w funkcji $t$.}
    \label{fig:path_Yt}
\end{figure}

\begin{figure}[ht!]
    \centering
    \includegraphics[width=1\textwidth]{../img/path/full_infection_expectation.png}
    \caption{$\mathbb{E}[Z]$ dla $\mathrm{P_n}$ w funkcji $n$.}
    \label{fig:path_Z}
\end{figure}

Wyniki eksperymentu niemal idealnie pokrywają się z przewidywanymi kształtami, to jest $1+tp$ dla $\mathbb{E}[Y_t]$ (zob. \cref{fig:path_Yt}) oraz $\frac{n-1}{p}$ dla $\mathbb{E}[Z]$ (zob. \cref{fig:path_Z}).



\section{Grafy gwiezdne}

Następnie rozpatrzmy rodzinę grafów gwiazd $\mathrm{S}_n$. Niech źródłem będzie centralny wierzchołek grafu, czyli $s = 0$. Propagacja rozchodzi się tutaj po każdym ramieniu gwiazdy niezależnie. Stąd mamy $X_v \sim \mathrm{Geo}(p)$ dla każdego $v \in \{1, 2, \dots, n\}$, a zmienne $X_1, X_2, \dots, X_n$ są od siebie niezależne. Otrzymujemy zatem:
\[
    \mathbb{E}[X_v] = \frac{1}{p}, \quad \mathrm{Var}[X_v] = \frac{1-p}{p^2}.
\]

Kwestia zmiennej $Y_t$ jest również prosta. Ponieważ propagacja działa niezależnie na każdym wierzchołku, $Y_t$ odpowiada liczbie sukcesów w $n$ próbach Bernoulliego. Sukcesem pojedynczej próby jest zdarzenie, że zmienna $X_v$ o rozkładzie geometrycznym osiągnie sukces w czasie co najwyżej $t$. Zatem:
\[
    \mathbb{P}[X_v \le t] = 1 - q^t,
\]
a więc
\[
    Y_t = 1 + B_t, \quad B_t \sim \mathrm{Bin}(n, 1 - q^t).
\]
W konsekwencji:
\[
    \mathbb{E}[Y_t] = 1 + n \cdot (1 - q^t).
\]

Przejdźmy teraz do zmiennej $Z$. Mamy $Z = \max\{X_1, X_2, \dots, X_n\}$.  
Ponieważ zmienne te są IID, z \Cref{fact:max_CDF} otrzymujemy:
\[
    \mathbb{P}[Z \le t] = {(1 - q^t)}^n.
\]
Policzmy wartość oczekiwaną całkowitego zainfekowania grafu:
\begin{equation*}
\begin{aligned}
\mathbb{E}[Z]
&= \sum_{k=1}^{\infty} \mathbb{P}[Z \ge k] 
= \sum_{k=1}^{\infty} 1 - \mathbb{P}[Z \le k-1]
= \sum_{k=1}^{\infty} 1 - (1 - q^{k-1})^n \\[2mm]
&= \sum_{k=0}^{\infty} 1 - (1 - q^k)^n
= \sum_{k=0}^{\infty} \Big( 1 - \sum_{j=0}^{n} \binom{n}{j} (-1)^j q^{kj} \Big) \\[2mm]
&= \sum_{k=0}^{\infty} \sum_{j=1}^{n} \binom{n}{j} (-1)^{j+1} q^{kj}
= \sum_{j=1}^{n} \sum_{k=0}^{\infty} \binom{n}{j} (-1)^{j+1} (q^j)^k \\[2mm]
&= \sum_{j=1}^{n} \binom{n}{j} \frac{(-1)^{j+1}}{1 - q^j}.
\end{aligned}
\end{equation*}

Nie jest to jednak szczególnie elegancka forma, więc spróbujmy wyznaczyć asymptotykę $\mathbb{E}[Z]$.  
Zauważmy, że
\[
    \mathbb{E}[Z] = \sum_{k=0}^{\infty} 1 - (1 - q^k)^n\big.
\]
Z \Cref{inequality:approximation_of_sum_by_an_integral} możemy przybliżyć tę sumę całką.  
Niech $f(x) = 1 - {(1 - e^{-\lambda x})}^n$, gdzie $\lambda = -\log(q)$.  
Oczywiście $f(0) = 1$, $f(\infty) = 0$, a funkcja $f$ jest malejąca, więc:
\[
    \int_{0}^{\infty} f(x)\, \mathrm{d}x \le \mathbb{E}[Z] \le 1 + \int_{0}^{\infty} f(x)\, \mathrm{d}x.
\]
Podstawiamy $u = 1 - e^{-\lambda x}$. 
Wtedy $\mathrm{d}u = \lambda e^{-\lambda x} \; \mathrm{d}x$,  a więc $\mathrm{d}x = \frac{1}{\lambda}\cdot\frac{1}{1-u} \; \mathrm{d}u$. 
Ponadto $u(0) = 0$, $u(\infty) = 1$ (bo $\lambda > 0$). 
Zatem całka ma postać:
\[
    \frac{1}{\lambda} \int_{0}^{1} \frac{1 - u^n}{1 - u}\, \mathrm{d}u
    = \frac{1}{\lambda} \sum_{j=0}^{n-1} \frac{1}{j+1}
    = \frac{H_n}{\lambda}.
\]
Zauważmy, że $-\log(q) = \log\big(\tfrac{1}{1-p}\big)$, zatem:
\[
    \frac{H_n}{\log\big(\tfrac{1}{1-p}\big)} \le \mathbb{E}[Z] \le \frac{H_n}{\log\big(\tfrac{1}{1-p}\big)} + 1.
\]
Stąd asymptotyczny czas pełnego zarażenia grafu $\mathrm{S}_n$ ma postać:
\[
    \mathbb{E}[Z] \sim \frac{H_n}{\log\big(\tfrac{1}{1-p}\big)}.
\]

Przeprowadźmy teraz symulacje.  
Ustalmy $p = 0.2$ oraz $n = 1000$.  
Dla każdego $t \in \{1, 2, \dots, \log(n)\}$ wykonajmy $2000$ powtórzeń propagacji na $\mathrm{S}_n$, a następnie dla $n \in \{1, 2, \dots, 1000\}$ oszacujmy $\mathbb{E}[Z]$.  
Wyniki przedstawiono na \Cref{fig:star_Yt} oraz \Cref{fig:star_Z}.

\begin{figure}[!ht]
    \centering
    \includegraphics[width=1\textwidth]{../img/star/final_infection_expectations.png}
    \caption{$\mathbb{E}[Y_t]$ dla $\mathrm{S_n}$ w funkcji $t$.}
    \label{fig:star_Yt}
\end{figure}

\begin{figure}[!ht]
    \centering
    \includegraphics[width=1\textwidth]{../img/star/full_infection_expectation.png}
    \caption{$\mathbb{E}[Z]$ dla $\mathrm{S_n}$ w funkcji $n$.}
    \label{fig:star_Z}
\end{figure}

Dla $\mathbb{E}[Y_t]$ obserwujemy niemal idealne dopasowanie do przewidywanego kształtu.  
Natomiast dla $\mathbb{E}[Z]$ wartości empiryczne są około o~$1$ mniejsze od oczekiwanych, co stanowi bardzo dobre przybliżenie.



\section{Ograniczenia na czas zarażenia}

Po rozważeniu dwóch rodzin grafów dostrzegamy znaczną różnicę w wartościach oczekiwanych zmiennych $Y_t$ oraz $Z$. 
Dla grafów ścieżkowych minimalna liczba rund potrzebnych do zainfekowania całego grafu wynosi $t = n - 1$, natomiast dla gwiazd jest to zaledwie $t = 1$. 
Widzimy więc, że w pewnym sensie najlepszy przypadek sprzyjający szybkiemu rozprzestrzenianiu się infekcji zachodzi wtedy, gdy źródło $s$ jest połączone ze wszystkimi pozostałymi wierzchołkami grafu.  
Z drugiej strony, najgorsza sytuacja ma miejsce, gdy istnieje odległy węzeł z niewielką liczbą ścieżek prowadzących do niego — tak jak w przypadku grafów ścieżkowych. 
Teraz postaramy się uogólnić tę obserwację. 

\begin{theorem}\label{theorem:montonicity_of_total_infection}
Niech $G = (V, E)$ będzie grafem spójnym, a $G' = (V, E')$ jego spójnym podgrafem.  
Załóżmy, że $\mathbf{X}$ opisuje proces stochastyczny w modelu \textbf{SI} prowadzony równocześnie na $G$ oraz $G'$ z tym samym źródłem $s \in V$.  
Jeśli przez $X_v'$, $Y_t'$ oraz $Z'$ oznaczymy odpowiednie zmienne losowe dla $G'$, to zachodzą nierówności:
\[
    X_v \le X_v', \quad Y_t \ge Y_t', \quad Z \le Z'.
\]
\end{theorem}

\begin{proof}
Oznaczmy przez $\mathcal{I}_t$ zbiór zainfekowanych wierzchołków w grafie $G$, a przez $\mathcal{I}_t'$ — w grafie $G'$.  
Wtedy $\mathcal{I}_t' \subseteq \mathcal{I}_t$ dla każdego $t \in \mathbb{N}$.  
Ustalmy $v \in V$ i niech $X_v' = a$.  
Wtedy $v \in \mathcal{I}_a'$, a więc także $v \in \mathcal{I}_a$, co implikuje $X_v \le a = X_v'$.  
Analogicznie, dla ustalonego $t \in \mathbb{N}$ z faktu, że $\mathcal{I}_t' \subseteq \mathcal{I}_t$, mamy $|\mathcal{I}_t'| \le |\mathcal{I}_t|$, a zatem $Y_t' \le Y_t$.  
Na koniec, jeśli $Z' = b$, to $\mathcal{I}_b' = V$, a więc $V \subseteq \mathcal{I}_b$, co prowadzi do $Z \le b = Z'$.
\end{proof}

Intuicyjnie, wynik ten jest oczywisty — mając mniej krawędzi w grafie, potrzebujemy więcej czasu, aby informacja (lub infekcja) rozprzestrzeniła się po całym grafie.  
W praktyce oznacza to, że jeśli znamy średni czas pełnego zainfekowania dowolnego podgrafu $G$, to otrzymujemy górne ograniczenie dla całego grafu.  
Spróbujmy teraz oszacować z góry wartość $\mathbb{E}[Z]$ dla dowolnego grafu.

\begin{theorem}\label{theorem:upper_bound_on_EZ}
Niech $G = (V, E)$ będzie grafem o $n$ wierzchołkach, a $s \in V$ — ustalonym źródłem.  
Oznaczmy $\lambda = \log\!\big(\tfrac{1}{1-p}\big)$ oraz $h = \epsilon(s)$.  
Wtedy zachodzi:
\[
    \mathbb{E}[Z] \le h + \frac{h}{\lambda} \left(\log\!\Big(\frac{n - 1}{h}\Big) + 1\right).
\]
\end{theorem}

\begin{proof}
Dla $0 \le j \le h$ zdefiniujmy zbiory $A_j = \{v \in V : \mathrm{d}(s, v) = j\}$ oraz liczności $a_j = |A_j|$.  
Mamy oczywiście $a_0 = 1$, a więc $a_1 + \cdots + a_h = n - 1$.  
Zdefiniujmy dalej zmienne losowe:
\[
    T_j = \min\{t \in \mathbb{N} : A_j \subseteq \mathcal{I}_t\}.
\]
Zmienna $T_j$ określa czas potrzebny na zainfekowanie wszystkich wierzchołków w odległości $j$ od źródła.  
Udowodnijmy teraz pomocniczy lemat.

\begin{lemma}\label{lemma:helper_lemma}
Niech $U_j = T_j - T_{j-1}$ dla $1 \le j \le h$. Wtedy:
\[
    \mathbb{E}[U_j] \le \frac{H_{a_j}}{\lambda} + 1.
\]
\end{lemma}

\begin{proof}
Zmienna $U_j$ opisuje czas potrzebny na zainfekowanie wierzchołków z $A_j$, zakładając, że wszystkie wierzchołki z $A_{j-1}$ są już zainfekowane.  
Skonstruujmy podgraf $G'$ w taki sposób, by każdy wierzchołek z $A_j$ był połączony dokładnie jedną krawędzią z pewnym wierzchołkiem ze zbioru $A_{j-1}$.  
Wtedy proces propagacji na $G'$ jest izomorficzny z tym na grafie gwiazdy $\mathrm{S}_{a_j}$, gdzie $a_j = |A_j|$.  
Z \Cref{theorem:montonicity_of_total_infection} wynika, że zmienna $U_j$ jest ograniczona przez całkowity czas infekcji w $\mathrm{S}_{a_j}$.  
Zgodnie z wcześniejszymi wynikami, jego wartość oczekiwana wynosi co najwyżej $\frac{H_{a_j}}{\lambda} + 1$, co kończy dowód.
\end{proof}

Przejdźmy do dowodu głównego twierdzenia.  
Mamy $T_h = \sum_{j=1}^{h} U_j$, zatem:
\begin{equation*}
\begin{aligned}
\mathbb{E}[Z] 
&= \mathbb{E}[T_h] 
  = \mathbb{E}\Big[\sum_{j=1}^{h} U_j\Big] 
  = \sum_{j=1}^{h} \mathbb{E}[U_j] 
  \le \sum_{j=1}^{h} \frac{H_{a_j}}{\lambda} + 1 \\
&= h + \frac{1}{\lambda} \sum_{j=1}^{h} H_{a_j} 
  \le h + \frac{1}{\lambda} \sum_{j=1}^{h} 1 + \log(a_j) \\
&= h + \frac{h}{\lambda} \cdot \Big(1 + \sum_{j=1}^{h} \log(a_j) \Big) = h + \frac{h}{\lambda} \cdot \Big(1 + \log \Big(\prod_{j=1}^{h} a_j\Big)\Big)\\
&\le h + \frac{h}{\lambda} \cdot \Big(1 + \log \Big(\frac{1}{h} \sum_{j=1}^{h} a_j\Big) \Big) = h + \frac{h}{\lambda} \cdot \Big(1 + \log \Big(\frac{n-1}{h}\Big) \Big) \\
\end{aligned}
\end{equation*}
gdzie w linijce pierwszej wykorzystujemy \cref{lemma:helper_lemma}, w drugiej \cref{inequality:harmonic_upper_bound} a w piątej nierówność między średnimi (\ref{inequality:AM_GM}).
\end{proof}

Porównajmy teraz powyższy wynik z wcześniejszymi obserwacjami.  
Dla rodziny grafów $\mathrm{P}_n$ mamy $h = n - 1$.  
Korzystając z \Cref{inequality:log_vs_x}, otrzymujemy:
\[
    \mathbb{E}[Z] \le (n - 1) \left(1 + \frac{1}{p}\right).
\]
Faktyczna wartość oczekiwana wynosi $\frac{n - 1}{p}$, więc oszacowanie jest dość dokładne.  
Z kolei dla rodziny grafów $\mathrm{S}_n$ mamy $h = 1$ oraz $n + 1$ wierzchołków, stąd:
\[
    \mathbb{E}[Z] \le 1 + \frac{\log n + 1}{\log\!\big(\tfrac{1}{1 - p}\big)}.
\]
Ponownie otrzymujemy zaskakująco dobre przybliżenie — szczególnie dla grafów rzadkich.



\section{Grafy cykliczne}

Przejdźmy teraz do grafów cyklicznych. 
W rozważaniach dla trójkąta, to jest $\mathrm{C}_3$, mogliśmy zauważyć, że cykl w tym grafie sprawiał trudności. 
W ogólnym przypadku nie jest lepiej. 
Rozważmy graf $\mathrm{C}_n$.
Niech źródłem będzie wierzchołek $n$. 
Ustalmy wierzchołek $v\in\{1,2,\dots,n-1\}$. 
Niech $a=\min\{v,n-v\}$ oraz $b=\max\{v,n-v\}$. 
Oczywiście $a\le b$. 
Od źródła do tego wierzchołka są dwie ścieżki: jedna o długości $a$, druga o długości $b$. Propagacja rozchodzi się po nich równolegle i niezależnie. 
Dla $j\in \{1,2,\dots,n-1\} $ połóżmy $N_j\sim \mathrm{NegBin}(j,p)$. 
Zmienne te są niezależne. 
Mamy wtedy 
\[
    X_v \sim \min\{N_a, N_b\}.
\]
Niech $F_j(t)$ będzie dystrybuantą zmiennej $N_j$. 
Z \Cref{fact:min_CDF} mamy 
\[
    \mathbb{P}[X_v\le t] = 1 - (1-F_a(t))\cdot(1-F_b(t)).
\]
Nie ma co liczyć na wyznaczenie eleganckiej postaci na PMF czy CDF dla $X_v$. 
Postaramy się więc przybliżyć wartość oczekiwaną dla dużych $n$. 
Z centralnego twierdzenia granicznego możemy przybliżyć $N_a \approx A,\; N_b \approx B$ dla $ A\sim \mathcal{N}(\mu_a,\sigma^2_a),\; B \sim \mathcal{N}(\mu_b,\sigma^2_b)$ gdzie 
\[
    \mu_a=\frac{a}{p},\quad \sigma^2_a=\frac{aq}{p^2}, \quad \mu_b=\frac{b}{p},\quad \sigma^2_b=\frac{bq}{p^2}.
\]
Zatem mamy
\[
    \mathbb{E}[X_v] \approx \mathbb{E}[\min\{A,B\}]=\mathbb{E}\Big[\frac{A+B-|A-B|}{2}\Big]=\frac{\mathbb{E}[A]+\mathbb{E}[B]-\mathbb{E}[|A-B|]}{2}.
\]
Połóżmy $C=A-B$. 
Korzystając z \Cref{fact:sum_of_normal_RV} mamy $C \sim \mathcal{N}(\mu_a-\mu_b,\sigma_a^2+\sigma_b^2)$. 
Oznaczmy $\eta=\mu_a-\mu_b$ oraz $\xi=\sqrt{\sigma_a^2+\sigma_b^2}$. 
Potrzebujemy teraz następującego lematu:

\begin{lemma}\label{lemma:abs_normal_E}
Niech $X \sim \mathcal{N}(\mu,\sigma^2)$. 
Wtedy
\[
    \mathbb{E}[|X|] = 2\sigma\cdot \varphi\Big(\frac{\mu}{\sigma}\Big)+\mu\cdot (2\mathbf{\Phi}\Big(\frac{\mu}{\sigma}\Big)-1).
\]    
\end{lemma}
\begin{proof}
\[
    \mathbb{E}[|X|]= \int_{-\infty}^{\infty} \frac{|x|}{\sigma}\varphi\Big(\frac{x-\mu}{\sigma}\Big) \; \mathrm{d}x = \int_{0}^{\infty} \frac{x}{\sigma}\varphi\Big(\frac{x-\mu}{\sigma}\Big) \; \mathrm{d}x - \int_{-\infty}^{0} \frac{x}{\sigma}\varphi\Big(\frac{x-\mu}{\sigma}\Big) \; \mathrm{d}x.
\]
Oznaczmy $c=\frac{\mu}{\sigma}$ oraz podstawmy $z=\frac{x-\mu}{\sigma}$. 
Zatem $x=\mu+\sigma z$, $\mathrm{d}{x}=\sigma \mathrm{d}{z}$. 
Dla $x>0$ mamy $z>-c$ zaś dla $x<0$ mamy $z<-c$. 
Otrzymujemy więc
\begin{align*}
&\int_{-c}^{\infty} (\mu+\sigma z)\varphi(z)\,\mathrm{d}z 
 - \int_{-\infty}^{-c} (\mu+\sigma z)\varphi(z)\,\mathrm{d}z \\[0.3em]
&= \mu \!\int_{-c}^{\infty} \!\varphi(z)\,\mathrm{d}z
   + \sigma \!\int_{-c}^{\infty} \!z\varphi(z)\,\mathrm{d}z
   - \mu \!\int_{-\infty}^{-c} \!\varphi(z)\,\mathrm{d}z
   - \sigma \!\int_{-\infty}^{-c} \!z\varphi(z)\,\mathrm{d}z \\[0.3em]
&= \mu\!\Big(\!\int_{-c}^{\infty}\!\varphi(z)\,\mathrm{d}z 
   - \!\int_{-\infty}^{-c}\!\varphi(z)\,\mathrm{d}z\!\Big)
   + \sigma\!\Big(\!\int_{-c}^{\infty}\!z\varphi(z)\,\mathrm{d}z 
   - \!\int_{-\infty}^{-c}\!z\varphi(z)\,\mathrm{d}z\!\Big) \\[0.3em]
&= \mu\!\Big(\!\int_{-\infty}^{\infty}\!\varphi(z)\,\mathrm{d}z 
   - 2\!\int_{-\infty}^{-c}\!\varphi(z)\,\mathrm{d}z\!\Big)
   + \sigma\!\Big(-\varphi(z)\big|_{-c}^{\infty}
   + \varphi(z)\big|_{-\infty}^{-c}\Big) \\[0.3em]
&= \mu\cdot\!\Big(1 - 2\mathbf{\Phi}(-c)\Big)
   + \sigma\cdot\!\Big(-\varphi(\infty)+\varphi(-c)+\varphi(-c)-\varphi(-\infty)\Big) \\[0.3em]
&= \mu\cdot\!\Big(2\mathbf{\Phi}(c)-1\Big) + 2\sigma\cdot\,\varphi(c).
\end{align*}
gdzie skorzystaliśmy z tożsamości $\mathbf{\Phi}(-x)=1-\mathbf{\Phi}(x)$, $\varphi(-x)=\varphi(x)$, \\ $\varphi(\pm\infty)=0$, $\int_{-\infty}^{\infty}\varphi(x) \;\mathrm{d}x = 1$ oraz $\int x\varphi(x)\;\mathrm{d}x=-\varphi(x)$.
\end{proof}

Z \Cref{lemma:abs_normal_E} dostajemy $\mathbb{E}[C]=2\xi\cdot \varphi\Big(\frac{\eta}{\xi}\Big)+\eta\cdot (2\mathbf{\Phi}\Big(\frac{\eta}{\xi}\Big)-1)$. 
Ostatecznie 
\begin{align*}
\mathbb{E}[X_v] \approx {} &
 \tfrac{1}{2}\!\Big(\mu_a+\mu_b
 -2\xi\,\varphi\!\Big(\tfrac{\eta}{\xi}\Big)
 -\eta\!\Big(2\mathbf{\Phi}\!\Big(\tfrac{\eta}{\xi}\Big)-1\Big)\!\Big) \\[0.4em]
={} &
 \tfrac{\mu_a+\mu_b}{2}
 -\xi\,\varphi\!\Big(\tfrac{\eta}{\xi}\Big)
 -(\mu_a-\mu_b)\!\Big(\mathbf{\Phi}\!\Big(\tfrac{\eta}{\xi}\Big)-\tfrac{1}{2}\Big) \\[0.4em]
={} &
 \mu_a\!\Big(1-\mathbf{\Phi}\!\Big(\tfrac{\eta}{\xi}\Big)\Big)
 +\mu_b\,\mathbf{\Phi}\!\Big(\tfrac{\eta}{\xi}\Big)
 -\eta\,\varphi\!\Big(\tfrac{\eta}{\xi}\Big).
\end{align*}

Przenalizujmy teraz zachowanie asymptotyczne otrzymanego wyrażenia. 
Skoro $a+b=n$ to niech $a=rn$, $b=(1-r)n$ dla pewnego $r\in(0;1)$. 
Dalej 
\[
    \frac{\eta}{\xi}=\frac{\mu_a-\mu_b}{\sqrt{\sigma_a^2+\sigma_b^2}}=\frac{\frac{a}{p}-\frac{b}{p}}{\sqrt{\frac{aq}{p^2}+\frac{bq}{p^2}}}=\frac{(2r-1)\sqrt{n}}{\sqrt{q}}.
\]
Musimy rozważyć dwa przypadki.

Jeśli $a<b$, co za tym idzie $r<\frac{1}{2}$ to $\frac{\eta}{\xi}\to -\infty$ wraz z $n\to \infty$. 
Wtedy też $\varphi\Big(\frac{\eta}{\xi}\Big)\to 0$ oraz $\mathbf{\Phi}\Big(\frac{\eta}{\xi}\Big)\to 0$ a więc $\mathbb{E}[X_v] \to \frac{a}{p}$. 

Zaś gdy $a=b$ to $r=\frac{1}{2}$ jak i  $\frac{\eta}{\xi}=0$. 
Wiemy, że $\varphi(0)=\frac{1}{\sqrt{2\pi}}$ oraz $\mathbf{\Phi}(0)=\frac{1}{2}$. 
Wstawiając otrzymamy $\mathbb{E}[X_v] \to \frac{n}{2p}-\frac{\sqrt{np}}{p\sqrt{2\pi}}$. 
Podsumowując mamy następujący wynik:
\[
    \mathbb{E}[X_v] \sim \frac{\min\{v,n-v\}}{p}.
\]
Jest to całkowicie zgodne z intuicją. 
Wierzchołki w grafie $\mathrm{C}_n$ zachowują się podobnie jak w grafach $\mathrm{P}_n$.

W celu wyznaczenie rozkładu $Y_t$ dokonajmy obserwacji, że gdy dwie drogi zarażania spotkają się to propagacja dobiega końca. 
Każda z tych dróg jak w przypadku grafu ścieżkowego ma rozkład dwumianowy. 
Możemy zapisać zatem
\[
    Y_t \sim \min\{n, 1+ L_t + R_t\}, \quad L_t,R_t\sim \mathrm{Bin}(t,p).
\] 
Z \Cref{fact:sum_of_bin_RV} mamy $L_t+R_t\sim\mathrm{Bin}(2t,p)$. 
Widzimy zatem, że rozkład $Y_t$ dla grafu $\mathrm{C}_n$ pokrywa się ze zmienną $Y_{2t}$ dla grafów typu $\mathrm{P}_n$.
Z wcześniejszego wyniku dla grafów ścieżek dostajemy
\[
    \lim_{n\to\infty} \mathbb{E}[Y_t] = 1+2tp.
\]

Teraz możemy wyznaczyć $\mathbb{E}[Z]$. 
Dla $n$ parzystego najdalej oddalony wierzchołek od źródła to $\frac{n}{2}$ a dla $n$ nieparzystego to $\lfloor \frac{n\pm 1}{2}\rfloor$. 
Asymptotycznie nie ma to znaczenia, możemy przyjąć $v=\frac{n}{2}$. 
Stąd
\[
    \mathbb{E}[Z] \approx \mathbb{E}[X_{\frac{n}{2}}] \approx \frac{n}{2p}-\frac{\sqrt{np}}{p\sqrt{2\pi}}.
\]

Również intuicyjny wynik. 
Aby się upewnić czy nie przesadziliśmy z szacowanie zwróćmy się ku symulacji. 
Dla $p=0.2$, $n\in\{3,4,\dots, 1000\}$ policzmy wartość oczekiwaną całkowitego zarażenia po razy $2000$ razy. 
Z wykresów (zob. \Cref{fig:cycle_Yt}, \Cref{fig:cycle_Z}) widzimy, że empiryczny wynik pokrywa się asymptotycznie z teoretycznym.
\begin{figure}[!ht]
    \centering
    \includegraphics[width=1\textwidth]{../img/cycle/final_infection_expectations.png}
    \caption{$\mathbb{E}[Y_t]$ dla $\mathrm{C_n}$ w funkcji $t$}
    \label{fig:cycle_Yt}
\end{figure}
\begin{figure}[!ht]
    \centering
    \includegraphics[width=1\textwidth]{../img/cycle/full_infection_expectation.png}
    \caption{$\mathbb{E}[Z]$ dla $\mathrm{C_n}$ w funkcji $n$}
    \label{fig:cycle_Z}
\end{figure}


\section{Grafy pełne}

Graf pełny $\mathrm{K}_n$ intuicyjnie powinien mieć najszybszą propagację ze względu na maksymalną liczbę krawędzi. 
Za źródło możemy przyjąć dowolny wierzchołek $s\in V$ ze względu na symetrie. 
Początkowo rozkład $X_v$ pokrywa się z rozkładem gwiazdy natomiast w każdej kolejnej rundzie mocno się komplikuje. 
Zachodzi bowiem $Y_t=a$ to $\mathbb{P}[X_v = t+1 | Y_t = a] = 1 - q^a$. 
Nie mamy co liczyć na jakiekolwiek sensowne wyznaczenie rozkładu $X_v$. 
Podejdźmy do problemu na razie heurystycznie. 
Zauważmy, że jeśli $Y_1=a$ to rozkład zmiennej $Y_2$ wynosi $Y_2 = a + B$ dla $B \sim \mathrm{Bin}(n-a, 1-q^n)$. 
Zatem 
\[
    \mathbb{E}[Y_2\mid Y_1 = a] = n\cdot (1-q^a) + aq^a.
\]
Mamy $Y_1\sim \mathrm{Bin}(n-1,p)$ oraz $\mathbb{E}[Y_1]=1+(n-1)p$. 
Możemy również założyć, że również $a \approx \mathbb{E}[Y_1]$ a co za tym idzie 
\[
    \mathbb{E}[Y_2\mid Y_1 = a] \approx n(1-q^{1+(n-1)p})+(1+(n-1)p)q^{1+(n-1)p} .
\]
Jeśli $n\to\infty$ to wyrażenie to jest bliskie $n$. 
Spodziewamy się zatem, że zaledwie po dwóch rundach cały graf $\mathrm{K}_n$ będzie zainfekowany. Zweryfikujmy teraz ten heurystyczny argument symulacją w Pythonie. 
Ustalmy $p=0.2$ i dla $n\in\{2,3,\dots,1000\}$ odpalmy propagację. 
\begin{figure}[!ht]
    \centering
    \includegraphics[width=1\textwidth]{../img/complete/final_infection_expectations.png}
    \caption{$\mathbb{E}[Y_t]$ dla $\mathrm{K_n}$ w funkcji $t$}
    \label{fig:complete_Yt}
\end{figure}
\begin{figure}[!ht]
    \centering
    \includegraphics[width=1\textwidth]{../img/complete/full_infection_expectation.png}
    \caption{$\mathbb{E}[Z]$ dla $\mathrm{K_n}$ w funkcji $n$}
    \label{fig:complete_Z}
\end{figure}
Widzimy, (zob. \Cref{fig:complete_Yt}, \Cref{fig:complete_Z}) że dla $n>200$ mamy $\mathbb{E}[Z] \approx 2$. 
Możemy więc wysunąć hipotezę: Dla grafu $\mathrm{K}_n$ mamy:
\[
    \lim_{n\to\infty} \mathbb{E}[Z] = 2.
\]
Postarajmy się ją teraz udowodnić. 
Żeby to zrobić najpierw wyznaczmy asymptotyke $\mathbb{E}[Y_2]$. 
Oznaczamy $U=Y_1=1+W$ gdzie $W\sim\mathrm{Bin}(n-1,p)$. 
Z prawa całkowitej wartości oczekiwanej mamy
\[
    \mathbb{E}[Y_2] = n\cdot (1-\mathbb{E}[q^U])+\mathbb{E}[Uq^U].
\]
Musimy wyznaczyć $\mathbb{E}[q^U]$ jak i $\mathbb{E}[Uq^U]$. 
\begin{lemma}\label{lemma:binomial_PGF}
Niech $X \sim \mathrm{Bin}(m, p)$.
Wtedy
\[
    \mathbb{E}[z^X]={(q+pz)}^m, \quad \mathbb{E}[Xz^X]=mpz{(q+pz)}^{m-1}.
\]    
\end{lemma}
\begin{proof}
Do obliczenia tych wartości posłuży nam \cref{summ:binomial_0} jak i \cref{summ:binomial_1}.
\[
    \mathbb{E}[z^X] = \sum_{k=0}^{m} z^k \cdot \binom{m}{k}p^k q^{m-k} = \sum_{k=0}^{m} \binom{m}{k}{(pz)}^k q^{m-k} = {(q+pz)}^m.
\]    
\[
    \mathbb{E}[Xz^X] = \sum_{k=0}^{m} kz^k \binom{m}{k}p^k q^{m-k} = \sum_{k=0}^{m} k \binom{m}{k}{(pz)}^k q^{m-k} = mpz{(q+pz)}^{m-1} .
\]
\end{proof}

W naszym przypadku dostajemy 
\[
    \mathbb{E}[q^U]=\mathbb{E}[q^{1+W}]=q\cdot \mathbb{E}[q^W]=q{(q+pq)}^{n-1}=q^n{(1+p)}^{n-1}.
\]
Dla drugiej wartości mamy zaś
\begin{equation*}
\begin{aligned}
\mathbb{E}[Uq^U] 
&= \mathbb{E}[(1+W)q^{1+W}] = q(\mathbb{E}[q^W]+\mathbb{E}[Wq^W]) \\
& = q(q^{n-1}{(1+p)}^{n-1}+(n-1)pq^{n-1}{(1+p)}^{n-2}) \\
&= q^n{(1+p)}^{n-2}(1+p+(n-1)p)=q^n{(1+p)}^{n-2}(1+np).
\end{aligned}
\end{equation*}
Podstawiając przed chwilą wyrażenia wzory do wzoru na $\mathbb{E}[Y_2]$ dostaniemy
\begin{equation*}
\begin{aligned}
\mathbb{E}[Y_2] 
&= n-nq^n{(1+p)}^{n-1}+q^n{(1+p)}^{n-2}(1+np)= \\
&= n-(n-1)q^n{(1+p)}^{n-2}=n-(n-1){(1+p)}^{-2}{(1-p^2)}^n.
\end{aligned}
\end{equation*}
Połóżmy $\varepsilon_n=(n-1){(1+p)}^{-2}{(1-p^2)}^n$. 
Wtedy $\mathbb{E}[Y_2]=n-\varepsilon_n$. 
Z nierówności Markova (\ref{inequality:Markov}) otrzymujemy $\mathbb{P}[Z\ge 3]=\mathbb{P}[n-Y_2\ge 1]\le \mathbb{E}[n-Y_2]=\varepsilon_n$. 
Dalej zauważmy, że $\mathbb{P}[Z=1]=p^{n-1}$ bo wszystkie próby zarażenia w rundzie pierwszej musiałby by się powieść. 
Ograniczmy teraz z dwóch stron $\mathbb{E}[Z]$. 
Z dołu mamy
\[
    \mathbb{E}[Z]=\sum_{k=1}^{\infty}\mathbb{P}[Z\ge k] \ge \mathbb{P}[Z\ge 1] + \mathbb{P}[Z\ge 2] = 1 + 1 - p^{n-1} = 2 - p^{n-1}.
\]
Zajmijmy się teraz oszacowaniem górnym. 
Zauważmy, że graf $\mathrm{K}_n$ zawiera $\mathrm{P}_n$ jako podgraf. 
Ustalmy jeden z tych podgrafów. 
Niech $Z'$ będzie zmienną losową czasu całkowitego zarażenia dla tego podgrafu. 
Z \Cref{theorem:montonicity_of_total_infection} mamy $Z\le Z'$ a co za tym idzie $\mathbb{E}[Z^2] \le \mathbb{E}[{(Z')}^2]$. 
Przypomnijmy, że $Z' \sim \mathrm{NegBin}(n-1,p)$ a więc $\mathbb{E}[{(Z')}^2]=\frac{{(n-1)}^2+(n-1)q}{p^2}$. 
\begin{equation*}
\begin{aligned}
\mathbb{E}[Z] 
&= \sum_{k=1}^{\infty}\mathbb{P}[Z\ge k] = \mathbb{P}[Z\ge 1] + \mathbb{P}[Z\ge 2] + \sum_{k = 3}^{\infty} \mathbb{P}[Z\ge k]  \\
&= 1 + 1-p^{n-1} + \mathbb{E}[Z \cdot \mathbf{1}_{Z\ge 3}] \le 2 - p^{n-1} + \sqrt{\mathbb{E}[Z^2]}\sqrt{\mathbb{E}[\mathbf{1}_{Z\ge 3}]} \\
&\le 2 - p^{n-1} + \sqrt{\mathbb{E}[{(Z')}^2]}\sqrt{\mathbb{P}[Z\ge 3]} \\
&\le 2 - p^{n-1} + \sqrt{\frac{{(n-1)}^2+(n-1)q}{p^2}}\sqrt{\varepsilon_n}.
\end{aligned}
\end{equation*}
gdzie wykorzystaliśmy nierówność Cauchy’ego-Schwarza (\ref{inequality:Cauchy_Schwarz}). Ostatecznie dostajemy 
\[
    2-p^{n-1}\le \mathbb{E}[Z] \le 2 - p^{n-1} + \sqrt{\frac{{(n-1)}^2+(n-1)q}{p^2}}\sqrt{\varepsilon_n},
\]
a zatem
\[
    \lim_{n \to \infty} \mathbb{E}[Z] = 2.
\]
Jeżeli zaledwie po dwóch rundach cały graf jest poinformowany to rozkłady $X_v$ czy $Y_t$ nie są dla nas istotne. 
Spójrzmy jeszcze na oszacowanie, które otrzymamy stosując \cref{theorem:upper_bound_on_EZ} dla grafu pełnego. 
Wynosi ono 
\[
    1 + \frac{\log(n-1) + 1}{\log(\frac{1}{1-p})}.
\]
Ograniczenie to zdaje się nie być za dobre czego przyczyną jest fakt, że $\mathrm{K}_n$ jest grafem gęstym.

\section{Drzewa}

Rozważmy drzewo $G = (V, E)$ oraz ustalony wierzchołek początkowy $s \in V$, 
który traktujemy jako korzeń drzewa. 
Dla $v\in V$ oznaczmy  $d_v=\mathrm{d}(s,v)$. Ustalmy $v\in V$. 
Skoro $G$ jest drzewem to istnieje dokładnie jedna ścieżka od $s$ do $v$, powiedzmy $s,v_1,\dots,v_k, v$. 
Ponieważ infekcja rozprzestrzenia się od korzenia $s$ wzdłuż krawędzi drzewa, każde zakażenie wymaga sukcesu w niezależnym doświadczeniu Bernoulliego o prawdopodobieństwie $p$.
W konsekwencji, aby infekcja dotarła z $s$ do $v$, musi wystąpić $\mathrm{d}_v$ kolejnych sukcesów. 
Zatem rozkład $X_v$ pokrywa się z rozkładem tej zmiennej dla grafu $\mathrm{P}_{d_v+1}$ na wierzchołkach $\{s,v_1,\dots,v_k, v\}$. 
Stąd 
\[
    X_v\sim \mathrm{NegBin}(\mathrm{d}_v,p),
\]
oraz
\[
    \mathbb{E}[X_v] = \frac{\mathrm{d}_v}{p}, \quad \mathrm{Var}[X_v] = \frac{\mathrm{d}_v\cdot(1 - p)}{p^2}.
\]
\begin{lemma}\label{lemma:Formula_EYt}
Dla dowolnego $t\in\mathbb{N}$ wartość oczekiwana zmiennej $Y_t$ wyraża się wzorem
\[
    \mathbb{E}[Y_t] = \sum_{v\in V} \mathbb{P}[X_v \le t].
\]    
\end{lemma}

\begin{proof}
Mamy $Y_t=|\{v\in V: X_v \le t\}|$ zatem $Y_t=\sum_{v\in V}  \mathbf{1}_{\{X_v\le t\}}$. Nakładając na tą równość operator $\mathbb{E}$ otrzymujemy:
\[
    \mathbb{E}[Y_t] = \mathbb{E}\Big[ \sum_{v\in V}  \mathbf{1}_{\{X_v\le t\}}\Big]= \sum_{v\in V} \mathbb{E}[\mathbf{1}_{\{X_v\le t\}}] = \sum_{v\in V} \mathbb{P}[X_v \le t].
\]    
\end{proof}

Przejdźmy teraz to obliczania średniej liczby zainfekowanych wierzchołków w czasie $t$. 
Oznaczmy przez $F(t;m,p)$ dystrybuante zmiennej o rozkładzie $\mathrm{NegBin}(m,p)$. 
Z \Cref{lemma:Formula_EYt} otrzymujemy
\[
    \mathbb{E}[Y_t] = \sum_{v\in V} F(t; d_v, p).
\]
Połóżmy $a_j = |\{v\in V: d_v=j\}|$ dla $0\le j \le h$.
Wtedy 
\[
    \mathbb{E}[Y_t] = \sum_{j=0}^{h} a_j\cdot F(t; j, p).
\]
Ponadto gdy $t<j\le h$ to $F(t; j, p)$, bo żaden wierzchołek w odległości od korzenia większej niż liczba rund nie może zostać zarażony. 
Możemy więc zmniejszyć granice sumowania 
\[
    \mathbb{E}[Y_t] = \sum_{j=0}^{\min\{h,t\}} a_j\cdot F(t; j, p).
\]

Oszacujmy teraz średni czas całkowity czas propagacji drzewa.
Niech $L=\{u_1,\dots, u_m\}$ będzie zbiorem liści w $G$. 
Wtedy mamy $Z = \max_{u\in L} X_{u}$.
Zauważmy, że $\epsilon(s) = \max_{u\in L} \mathrm{d}_{u}$ i jest to wysokość drzewa. 
Oznaczmy ją przez $h$. 
Z nierówności Jensena (\ref{inequality:Jensen}) otrzymujemy
\[
    \mathbb{E}[Z]=\mathbb{E}[\max_{u\in L} X_{u}] \ge \max_{u\in L} \mathbb{E}[X_{u}] = \max_{u\in L} \frac{\mathrm{d}_{u}}{p} = \frac{h}{p}.
\]
Aby ograniczyć $\mathbb{E}[Z]$ z góry skorzystamy z \Cref{theorem:upper_bound_on_EZ}:
\[
    \mathbb{E}[Z] \le h  + h \cdot \frac{\log(\frac{n-1}{h}) + 1}{\log(\frac{1}{1-p})}.
\]

Ograniczenia te są różnych rzędów wielkości. 
Jednakże nie da się ich poprawić dla ogólnego drzewa znając tylko liczbę jego wierzchołków i wysokość. 
Ustalmy $n\in\mathbb{N}_+$ oraz $h\in\{1,2,\dots,n-1\}$ i poszukajmy drzew o $n$ wierzchołkach i wysokości $h$ osiągających zarówno dolne jak i górne ograniczenie na $\mathbb{E}[Z]$. 
Dla dolnej nierówności możemy wziąć drzewo składające się ze ścieżki długości $h$ oraz $n-1-h$ liści bezpośrednio przy korzeniu. 
Wtedy $\mathbb{E}[Z] \approx \frac{h}{p}$. 
Aby znaleźć drzewo osiągające górne ograniczenie musimy wrócić do dowodu \Cref{theorem:upper_bound_on_EZ}. 
Udowadniając granicę na wartość oczekiwaną korzystamy z trzech nierówności. 
Pierwsza z nich to \cref{inequality:harmonic_upper_bound}. 
Jest ona bardzo ciasna a ponadto nie zależy od grafu. 
Druga z nich to nierówność między średnią arytmetyczną a geometryczną (\ref{inequality:AM_GM}).
Aby uzyskać równość potrzebujemy mieć $a_1=\cdots=a_h$. 
Czyli innymi słowy, nasze drzewo ma tyle samo węzłów na każdej głębokości. 
Połóżmy $a_1=b$. 
Wtedy $hb=n-1$ a więc $b=\lfloor\frac{n-1}{h}\rfloor$. 
Na koniec zostaje nierówność wynikająca z \Cref{lemma:helper_lemma}. 
Sam lemat daje nierówność, której nie da się poprawić, co wiemy poprzez analize dla grafów gwiazd. 
Lecz dla drzewa będzie ona najmniej luźna, jeżeli każdy wierzchołek w warstwie $A_j$ będzie miał dokładnie jedną krawędź łączącą go z wierzchołkiem w warstwie $A_{j+1}$, gdzie $0\le j\le h-1$. 
Zatem drzewo składa się z korzenia oraz $b$ rozłącznych ścieżek, każda o długości $h$. 
I taki graf osiąga ograniczenie górne na $\mathbb{E}[Z]$. 
Widzimy zatem, że nasze ograniczenia nie są do poprawienia bez dodatkowych parametrów grafu. 
Podsumowując możemy następująco szacować przewidywany czas całkowitego poinformowania drzewa o wysokości $h$:
\[
    \frac{h}{p} \le \mathbb{E}[Z] \le h  + h \cdot \frac{\log(\frac{n-1}{h}) + 1}{\log(\frac{1}{1-p})}.
\]

\chapter{Rozkłady wymierające}
W celu ułatwienia analizy modeli \textbf{SIR} oraz \textbf{SIS} wprowadzimy nowe rozkłady Prawdopodobieństwa uwzgledniajace możliwość przerwania propagacji.
Te zmienne losowe będą przyjmować wartość nieskończoność w sytuacji, w której nastąpi przerwanie przed pożądanym rezultatem.
Jako że nastepuje to z dodatnim prawdopodobieństwem, $\mathbb{P}[X=\infty]>0$, to ich wartości oczekiwane wynosić będą nieskończoność, $\mathbb{E}[X]=\infty$.
Interesować nas więc bedzie wartość oczekiwana warunkowa przy warunku, że wartość zmiennej losowej jest skończona a więc $\mathbb{E}[X|X<\infty]$.

\section{Rozkład umierający geometryczny}
Rozkład umierający geometryczny jest wariantem rozkładu geometrycznego, w którym proces może zostać przerwany z prawdopodobieństwem $\alpha\in(0;1)$ po każdej próbie.
Zmienna $X$ ma rozkład umierający geometryczny, jeżeli opisuje liczbę prób Bernoulliego potrzebnych do uzyskania pierwszego sukcesu w przypadku, w którym sukces nastąpi przed przetrwaniem eksperymentu.
Jeśli natomiast proces zostanie zabity szybciej niż pierwsza porażka to wtedy $X=\infty$.
Dla wygody oznaczmy $q=1-p, \; \beta=1-\alpha$.
Aby sukces nastąpił po $k$ rundach to potrzebujemy $k-1$ niepowodzeń próby jak i jej przerwania a następnie sukcesu.
A zatem
\[
    \mathbb{P}[X=k] = p(q\beta)^{k-1}, \quad k \in \mathbb{N}_+.
\]
Dystrybuanta jest więc równa:
\[
    \mathbb{P}[X\le t] = \sum_{k=0}^{t} \mathbb{P}[X=k] = \sum_{k=0}^{t} p(q\beta)^{k-1} = p\frac{1-(q\beta)^t}{1-q\beta}.
\]
Ponadto
\[
    \mathbb{P}[X < \infty] = \sum_{k=0}^{\infty} \mathbb{P}[X=k] = \sum_{k=0}^{\infty} p(q\beta)^{k-1} = \frac{p}{1-q\beta}. 
\]
Suma wszystkich prawdopodobieństw wynosi $1$.
A więc
\[
    \mathbb{P}[X=\infty] = 1 - \frac{p}{1-q\beta} = \frac{q\alpha}{1-q\beta}.
\]
Wartość oczekiwana wynosi nieskończoność.
Natomiast
\begin{equation*}
\begin{aligned}
\mathbb{E}[X| X < \infty]
&= \frac{1}{\mathbb{P}[X<\infty]}\sum_{t=1}^{\infty} t \cdot \mathbb{P}[X_v=t] = \frac{1-q\beta}{p} \sum_{t=1}^{\infty} t {(q\beta)}^{t-1}p\\
& = (1-q\beta)\cdot \frac{1}{{(1-q\beta)}^2}=\frac{1}{1-q\beta}.
\end{aligned}
\end{equation*}
Oznaczamy $X \sim \mathrm{KGeo}(p,\alpha)$.


\section{Rozkład umierający dwumianowy}
Rozkład umierający dwumianowy jest wariantem rozkładu dwumianowego, w którym doświadczenia mogą zostać przerwane z prawdopodobieństwem $\alpha\in(0;1)$ po każdej próbie.
Zmienna $X$ ma rozkład umierający dwumianowy, jeżeli opisuje liczbę sukcesów podczas $n$ prób Bernoulliego, gdzie po każdej próbie proces doświadczeń może zostać zabity.
Dla wygody oznaczmy $q=1-p, \; \beta=1-\alpha$.
Niech $F$ będzie zmienną określającą czas porażki.
Dla $1\le j\le n-1$ mamy $\mathbb{P}[F=j]=\alpha\beta^{j-1}$ oraz $\mathbb{P}[F=n]=\beta^{n-1}$.
Jeśli proces zostanie zabity po $j$ rundach to wtedy $X$ ma rozkład $\mathrm{Bin}(j,p)$.
A zatem korzystając z prawdopodobieństwa całkowitego dla $k\in\{0,1,\dots,n\}$ mamy
\begin{equation*}
\begin{aligned}
\mathbb{P}[X=k] &=\sum_{j=k}^{n} \mathbb{P}[X=k|F=j]\cdot\mathbb{P}[F=j]= \\
&=\binom{n}{k}p^k q^{n-k}\beta^{n-1}+\alpha\sum_{j=k}^{n-1}\binom{j}{k}p^k q^{j-k}\beta^{j-1}.   
\end{aligned}    
\end{equation*}
Dalej mamy $\mathbb{E}[X|F=j]=jp$ oraz
\begin{equation*}
\begin{aligned}
    \mathbb{E}[F]&=\sum_{j=1}^{n}j\cdot\mathbb{P}[F=j]=n\beta^{n-1}+\sum_{j=1}^{n-1} j\alpha\beta^{j-1}=\\
    &=n\beta^{n-1}+\alpha\frac{1-n\beta^{n-1}+(n-1)\beta^n}{(1-\beta)^2}  =\\
    &=\frac{1}{\alpha}(n\alpha\beta^{n-1}+1-n\beta^{n-1}+n\beta^n-\beta^n)=\\
    &=\frac{1}{\alpha}(1-\beta^n+n\beta^{n-1}(\alpha+\beta-1))=\frac{1-\beta^n}{\alpha}.
\end{aligned}    
\end{equation*}
Z prawa całkowitej wartości oczekiwanej mamy
\[
    \mathbb{E}[X]=\mathbb{E}[\mathbb{E}[X|F]]=p\mathbb{E}[F]=\frac{p}{\alpha}(1-\beta^n).
\]
Oznaczamy $X\sim\mathrm{KBin}(n,p,\alpha)$.


\section{Rozkład umierający ujemny dwumianowy}
Rozkład umierający ujemny dwumianowy jest wariantem rozkładu ujemnego dwumianowego, w którym doświadczenia mogą zostać przerwane z prawdopodobieństwem $\alpha\in(0;1)$ po każdej próbie.
Zmienna $X$ ma rozkład umierający ujemny dwumianowy, jeżeli opisuje liczbę prób Bernoulliego potrzebnych do uzyskania $m$ sukcesów próbach Bernoulliego, gdzie po każdej próbie proces może się zakończyć.
Jeśli proces zostanie zabity szybciej niż zajdzie $m$ sukcesów to przyjmujemy $X=\infty$.
Dla wygody oznaczmy $q=1-p, \; \beta=1-\alpha$.
Alternatywnie istnieją niezależne zmienne $Y_1,Y_2,\dots,Y_m\sim\mathrm{KGeo}(p,\alpha)$ takie, że $X=Y_1+Y_2+\dots Y_m$.
Aby wyznaczyć rozkład $X$ ustalmy $k\ge m$ i niech $K=\{\mathbf{y}\in\mathbb{N}_+^m:y_1+\cdots+y_m=k\}$.
Wtedy
\begin{equation*}
\begin{aligned}
    \mathbb{P}[X=k] &=\sum_{\mathbf{y}\in K} \mathbb{P}[Y_1=y_1,\dots,Y_m=y_m]=\sum_{\mathbf{y}\in K} \prod_{j=1}^{m}\mathbb{P}[Y_j=y_j] \\
    &= \sum_{\mathbf{y}\in K} \prod_{j=1}^{m} p(q\beta)^{y_j-1}=\sum_{\mathbf{y}\in K} p^m(q\beta)^{y_1+\cdots +y_m-m}  \\
    &=\sum_{\mathbf{y}\in K}p^m(q\beta)^{k-m} = |K|\cdot p^m(q\beta)^{k-m}.
\end{aligned}    
\end{equation*}
Z \Cref{fact:stars_and_bars} mamy $|K|=\binom{k-1}{m-1}$.
Zatem
\[
    \mathbb{P}[X=k] = \binom{k-1}{m-1} p^m (q\beta)^{k-m}, \quad k\ge m.
\]
Dalej mamy
\begin{equation*}
\begin{aligned}
    \mathbb{P}[X<\infty] &=\sum_{k=m}^{\infty} \mathbb{P}[X=k] = \sum_{k=m}^{\infty} \binom{k-1}{m-1} p^m (q\beta)^{k-m} =\\
    &= p^m(q\beta)^{-m} \sum_{k=m}^{\infty} \binom{k-1}{m-1} (q\beta)^k =p^m (q\beta)^{-m} \frac{(q\beta)^m}{(1-q\beta)^m} \\
    &= \frac{p^m}{(1-q\beta)^m}.
\end{aligned}    
\end{equation*}
co daje nam
\[
    \mathbb{P}[X=\infty]=1-\frac{p^m}{(1-q\beta)^m}.
\]
Jeśli chodzi o wartość oczekiwaną to
\begin{equation*}
\begin{aligned}
    &\mathbb{E}[X|X<\infty]=\frac{1}{\mathbb{P}[X<\infty]}\sum_{k=m}^{\infty} k\cdot \mathbb{P}[X=k]\\
    &=\frac{(1-q\beta)^m}{p^m}\sum_{k=m}^{\infty} k \binom{k-1}{m-1} p^m (q\beta)^{k-m} = \\
    &=\frac{(1-q\beta)^m}{(q\beta)^m} \sum_{k=m}^{\infty} k \binom{k-1}{m-1} (q\beta)^k = \\
    &=\frac{(1-q\beta)^m}{(q\beta)^m} \cdot \frac{m(q\beta)^m}{(1-q\beta)^{m+1}}=\frac{m}{1-q\beta}. 
\end{aligned}    
\end{equation*}
Oznaczamy $X\sim\mathrm{KNegBin}(m,p,\alpha)$.

\chapter{Analiza modelu SIR}
\section{Dwa wierzchołki, jedna krawędź}

W celu oswojenia się z bardziej skomplikowanym modelem, jakim jest SIR, przeanalizujmy graf o jednej krawędzi. 
Niech $V=\{u,v\}$ oraz niech $u$ będzie wierzchołkiem startowym. 
Oczywiście $X_u=0$. 
Pierwsza runda jest identyczna jak w modelu SI, a więc $\mathbb{P}[X_v=1]=p$. 
Jeżeli węzeł $v$ nie zostanie poinformowany w rundzie pierwszej, a propagacja nie wygaśnie, to sytuacja się powtórzy. 
Zachodzi to z prawdopodobieństwem $q\beta$. Aby $X_v=t$, potrzebujemy, by ten cykl nastąpił $t-1$ razy. 
Widzimy zatem, że $X_v$ ma rozkład umierający geometryczny:
\[
    X_v\sim \mathrm{KGeo}(p,\alpha), \quad \mathbb{E}[X_v|X_v<\infty]=\frac{1}{1-q\beta}.
\]
Z założeń modelu $p,\alpha\in(0;1)$, a zatem $\mathbb{P}[X_v=\infty]\ne 0$. 
Jest cecha kontrastująca SIR względem SI.

Przyjrzyjmy się teraz zmiennej $Y_t$.
Zauważmy, że $\mathbb{P}[Y_t=2]=\mathbb{P}[X_v\le t]$ a więc
\[
    \mathbb{P}[Y_t=2] = p\frac{1-q^t\beta^t}{1-q\beta}, \quad \mathbb{P}[Y_t=1]=1-p\frac{1-q^t\beta^t}{1-q\beta}.
\]
Stąd
\[
    \mathbb{E}[Y_t] = 1 + p\frac{1-q^t\beta^t}{1-q\beta}.
\]
Przechodząc w granicę $t\to\infty$ dostajemy
\[
    \mathbb{P}[W=1]=1-\frac{p}{1-q\beta}, \quad \mathbb{P}[W=2] = \frac{p}{1-q\beta}.
\]
Co oczywiście daje nam natychmiast
\[
    \mathbb{E}[W] = 1 + \frac{p}{1-q\beta}.
\]

Jeśli chodzi o zmienną $Z$ to propagacja zakończy się gdy wierzchołek $v$ wyzdrowieje lub $u$ zostanie poinformowany.
Dzieje się to z prawdopodobieństwem $1-q\beta$ a więc 
\[
    Z\sim\mathrm{Geo}(1-q\beta) \quad \mathbb{E}[Z]=\frac{1}{1-q\beta}.
\]


\section{Pewne wygaśnięcie}

W modelu SIR wygaśnięcie infekcji jest zdarzeniem pewnym. 
Sformalizujmy ten fakt prostym twierdzeniem.

\begin{theorem}\label{theorem:infection_dies_out_SIR}
Niech $G=(V,E)$ będzie grafem spójnym, $s\in V$ źródłem infekcji oraz $v\in V\setminus\{s\}$.
Wtedy zachodzą następujące tożsamości:
\[
    \mathbb{P}[X_v = \infty] > 0, \quad \mathbb{P}[Z < \infty] = 1.
\]
Ponadto dla $k\in\mathbb{N}$ zachodzi 
\[
    \mathbb{P}[W=k]=\lim_{t\to\infty}\mathbb{P}[Y_t=k], \quad \mathbb{E}[W]=\lim_{t\to\infty}\mathbb{E}[Y_t].
\]
\end{theorem}

\begin{proof}
Ustalmy $v\in V\setminus\{s\}$. 
Jeśli po pierwszej rundzie źródło wyzdrowieje i nie przekaże dalej infekcji to $X_v=\infty$.
Zatem $\mathbb{P}[X_v=\infty]\ge \alpha q > 0$.
Jeśli rozważamy propagacje jako łańcuch Markova to prawdopodobieństwo przejścia do stanu absorbującego, czyli $\mathcal{I}_t=\varnothing$, w czasie $t$ wynosi
\[
    \alpha^{|\mathcal{I}_t|}\prod_{u\in \mathcal{I}_t} q^{|\mathrm{N}(u)\cap\mathcal{S}_t|}.
\]
Istotne jest to, że jest to dodatnia liczba. 
A więc mamy $\mathbb{P}[Z<\infty]=1$.
Zauważmy, że 
\[
    \lim_{t\to\infty} \mathbf{1}_{v\in\mathcal{I}_t\cup\mathcal{R}_t} \xrightarrow{\text{a.s}} \mathbf{1}_{X_v<\infty}.
\]
Ponadto $Y_t\le Y_{t+1}$ oraz $Y_t\le |V|$.
A więc z Twierdzenia Lebesgue dostajemy wzory na rozkład i wartość oczekiwaną zmiennej $W$.
\end{proof}


\section{Grafy ścieżkowe}

Przyjrzyjmy się teraz grafom $\mathrm{P}_n$.
Oczywiście za źródło propagacji wybieramy wierzchołek $s=1$.
Podobnie jak w modelu SI propagacja rozchodzi się miedzy kolejnymi wierzchołkami niezależnie a więc
\[
    X_1=0, \quad X_v=X_{v-1}+U_v, \quad v\in\{2,3,\dots,n\},
\]
gdzie jednak $U_2,U_3,\dots,U_n\sim\mathrm{KGeo}(p,\alpha)$ oraz są niezależne.
Stąd $X_v=U_2+U_3+\cdots+U_v$ a zatem $X_v$ ma rozkład umierający ujemny dwumianowy:
\[
    X_v\sim\mathrm{KNegBin}(v-1,p,\alpha), \quad \mathbb{E}[X_v|X_v<\infty] = \frac{v-1}{1-q\beta}.
\]

Jeśli chodzi o zmienną $Y_t$ to zauważmy, że
\[
    Y_t=\max\{j\in\{1,2,\dots,n\}: X_j\le t\}.
\]
A więc dla $k\in\{1,2,\dots,n\}$ mamy
\[
    \mathbb{P}[Y_t\ge k+1] = \mathbb{P}[X_{k+1}\le t] = \sum_{j=k}^{t} \binom{j-1}{k-1}p^k {(q\beta)}^{j-k}.
\]
Dalej mamy
\[
    \mathbb{E}[Y_t] = \sum_{k=0}^{n-1} \mathbb{P}[Y_t\ge k+1] = 1+\sum_{k=1}^{n-1}\sum_{j=k}^{t} \binom{j-1}{k-1}p^k {(q\beta)}^{j-k}.
\]
Nie ma zbytniej nadziei na zwartą formę tej sumy.
Przejdźmy teraz do zmiennej $W$.
\[
    \mathbb{P}[W\ge k+1] = \lim_{t\to\infty} \sum_{j=k}^{t} \binom{j-1}{k-1}p^k {(q\beta)}^{j-k} = \Big{(\frac{p}{1-q\beta}\Big)}^k.
\]
Oznaczmy $\theta = \frac{p}{1-q\beta}$.
Wtedy
\[
    \mathbb{P}[W=k]=(1-\theta)\theta^{k-1}, \quad k\in\{1,2,\dots,n-1\}, \quad \mathbb{P}[W=n]=\theta^{n-1}.
\]
Ponadto
\[
    \mathbb{E}[W] = \sum_{k=0}^{n-1} \mathbb{P}[W\ge k+1] = \sum_{k=0}^{n-1} \theta^k = \frac{1-\theta^n}{1-\theta}.
\]

Skoncentrujmy naszą uwagę na zmiennej $Z$.
Niech $T_j$ oznacza czas przejścia ze stanu w którym wierzchołek $j$ został dopiero co zainfekowany do kolejnego stanu, gdzie $j\in V$.
Formalnie $T_j=\min\{t\in\mathbb{N}:i=X_j\land \neg (j\in\mathcal{I}_{t+i} \land j+1\in\mathcal{S}_{t+i})\}$.
Wtedy $T_j\sim \mathrm{Geo}(1-q\beta)$, a ponadto $T_1,T_2,\dots$ są niezależne.
Zauważmy, że jeśli $W<n$ to $Z=T_1+\cdots+T_W$ a gdy zaś $W=n$ to $Z=T_1+\cdots+T_{n-1}$.
Zatem $Z=T_1+T_2+\cdots+T_{Q}$ gdzie $Q=\min\{W,n-1\}$.
Będziemy potrzebować rozkładu i wartości oczekiwanej zmiennej $Q$.
Możemy zapisać $Q=W-\mathbf{1}_{W=n}$ a wtedy
\[
    \mathbb{E}[Q]=\mathbb{E}[W]-\mathbb{P}[W=n]=\frac{1-\theta^n}{1-\theta}-\theta^{n-1}=\frac{1-\theta^{n-1}}{1-\theta}.
\]
Dalej $\mathbb{P}[Q=k]=\mathbb{P}[W=k]=(1-\theta)\theta^{k-1}$ dla $k\in\{1,2,\dots,n-2\}$ oraz $\mathbb{P}[Q=n-1]=\mathbb{P}[W\ge n-1] = \theta^{n-2}$.
Zachodzi zatem $\mathbb{P}[Z=t|Q=m]=\mathbb{P}[T_1+\cdots+T_m=t]$ ale $T_1+\cdots+T_m\sim\mathrm{NegBin}(m,1-q\beta)$ (\cref{fact:sum_of_geo_RV}).
Daje nam to $\mathbb{P}[Z=t|Q=m]=\binom{t-1}{m-1} {(1-q\beta)}^m {(q\beta)}^{t-m}$.
Z prawdopodobieństwa całkowitego dostajemy
\[
    \mathbb{P}[Z=t]=\sum_{m=1}^{n-1} \binom{t-1}{m-1}{(1-q\beta)}^m {(q\beta)}^{t-m} \cdot \mathbb{P}[Q=m].
\]
Natomiast wartość oczekiwana zmiennej $Z$ wynosi
\[
    \mathbb{E}[Z]=\mathbb{E}[Q]\cdot\mathbb{E}[T_1]=\frac{1-\theta^{n-1}}{1-\theta} \cdot \frac{1}{1-q\beta} = \frac{1-\theta^{n-1}}{\frac{q\alpha}{1-q\beta}} \cdot \frac{1}{1-q\beta} = \frac{1-\theta^{n-1}}{q\alpha}.
\]


\section{Grafy gwiezdne}

Rozpatrzmy propagacje na rodzinie $\mathrm{S}_n$, z źródłem $0$.
Dla $v\in\{1,2,\dots,n\}$ mamy 
\[
    X_v\sim\mathrm{KGeo}(p,\alpha), \quad \mathbb{E}[X_v|X_v<\infty] = \frac{1}{1-q\beta}.
\]
Zauważmy natomiast, że zmienne $X_1,X_2,\dots,X_n$ sa zależne, bo gdy centralny wierzchołek wyzdrowieje żaden z liści nie może już zostać zainfekowany.

Przyjrzyjmy się teraz zmiennej $Y_t$. 
Połóżmy $C=\min\{\tau\in\mathbb{N}:0\in\mathcal{R}_\tau\}$.
Jeśli centrum zarazi się po $j$ rundach to liście gwiazdy mogą byc zarażane przez $\min\{j,t\}$ rund.
$Y_t$ zachowuje się tak samo jak w modelu SI dla grafów gwiezdnych.
Przypomnijmy, że rozkład ten wynosi $1+B_{j,t}$ gdzie $B_{j,t}\sim\mathrm{Bin}(n,1-q^{\min\{j,t\}})$.
A więc
\[
    \mathbb{P}[Y_t=k+1|C=j] = \binom{n}{k}{(1-q^{\min\{j,t\}})}^k{(q^{\min\{j,t\}})}^{n-k}.
\]
Ponadto $C\sim\mathrm{Geo}(\alpha)$ a więc $\mathbb{P}[C=j]=\alpha\beta^{j-1}$.
Z prawdopodobieństwa całkowitego dostajemy
\begin{equation*}
\begin{aligned}
\mathbb{P}[Y_t=k+1] &=\sum_{j=1}^\infty \mathbb{P}[Y_t=k+1|C=j]\cdot\mathbb{P}[C=j] = \\
&=\sum_{j=1}^\infty \binom{n}{k}{(1-q^{\min\{j,t\}})}^k{(q^{\min\{j,t\}})}^{n-k}\alpha\beta^{j-1} = \\
&= \frac{\alpha}{\beta}\binom{n}{k} \sum_{j=1}^\infty {(1-q^{\min\{j,t\}})}^k{(q^{\min\{j,t\}})}^{n-k}\beta^j.
\end{aligned}
\end{equation*}
Aby obliczyć wartość oczekiwana skorzystamy z \Cref{lemma:Formula_EYt}.
Mamy $\mathbb{P}[X_0\le t]=1$ a dla $v\ne 0$ mamy $\mathbb{P}[X_v\le t] = p\frac{1-{(q\beta)}^t}{1-q\beta}$.
A więc
\[
    \mathbb{E}[Y_t]=\sum_{v\in V}\mathbb{P}[X_v\le t] = 1 + np\frac{1-{(q\beta)}^t}{1-q\beta}.
\]
Licząc granice tych wyrażeń dla $t\to\infty$ dostajemy
\[
    \mathbb{P}[W=k+1] = \frac{\alpha}{\beta}\binom{n}{k} \sum_{j=1}^\infty {(1-q^j)}^k{(q^j)}^{n-k}\beta^j, \quad \mathbb{E}[W]=1+\frac{np}{1-q\beta}.
\]

Propagacja się zakończy gdy centrum wyzdrowieje lub wszystkie liście zostaną zarażone. 
Stąd mamy
\begin{equation*}
\begin{aligned}
\mathbb{P}[Z>t] &=\mathbb{P}[C>t]\cdot\mathbb{P}[Y_t<n+1|C>t]= \\
&=\beta^t(1-\mathbb{P}[Y_t=n+1|C>t]) = \beta^t(1-{(1-q^t)}^n).
\end{aligned}
\end{equation*}
Bezpośrednie liczenie wartości oczekiwanej zmiennej $Z$ nie da nam zwiezłej postaci.
Postarajmy się ją za to oszacować korzystając z \Cref{inequality:approximation_of_sum_by_an_integral}.
Oznaczmy $a=-\log(q),\; b=-\log(\beta)$ oraz $g(x)=e^{-bx}(1-{(1-e^{-ax})}^n)$.
Wtedy
\[
    \mathbb{E}[Z] \approx \int_{0}^{\infty} g(x) \; \mathrm{d}x.
\]
Dalej
\[
    \int_{0}^{\infty} e^{-bx}(1-{(1-e^{-ax})}^n)\; \mathrm{d}x = \int_{0}^{\infty} e^{-bx}\mathrm{d}x - \int_{0}^{\infty} e^{-bx}{(1-e^{-ax})}^n\mathrm{d}x.
\]
Wartość pierwszej całki wynosi $\frac{1}{b}$.
Aby policzyć drugą podstawmy $u=1-e^{-ax}$.
Oczywiście $u(0)=0, \; u(\infty)=1$.
Wtedy też $x=-\frac{1}{a}\log(1-u)$ oraz $\mathrm{d}x=\frac{1}{a}e^{ax}\mathrm{d}u$.
Mamy zatem $e^{-bx}={(1-u)}^{\frac{b}{a}}, \; e^{ax}=\frac{1}{1-u}$.
A więc całka wynosi
\begin{align*}
&\frac{1}{a}\int_{0}^{1} u^n {(1-u)}^{\frac{b}{a}-1} \;\mathrm{d}u = \frac{1}{a} \mathrm{B}\Big(n+1,\frac{b}{a}\Big) = \frac{1}{a} \frac{\Gamma(n+1)\Gamma(\frac{b}{a})}{\Gamma(n+1+\frac{b}{a})} \\[0.3em]
&=\frac{n!}{a} \cdot\frac{\Gamma(\frac{b}{a})}{(n+\frac{b}{a})(n-1+\frac{b}{a})\dots(\frac{b}{a})\Gamma(\frac{b}{a})} = \frac{n!}{b} \cdot \frac{1}{{(\frac{b}{a}+1)}_n}.
\end{align*}
gdzie $\Gamma(x)$ jest funkcją Gamma, $\mathrm{B}(x,y)$ jest funkcją Beta oraz ${(x)}_n=x(x+1)\dots(x+n-1)$ jest symbolem Pochhammer'a.
Ostatecznie otrzymujemy
\[
    \mathbb{E}[Z] \approx \frac{1}{b} \cdot \Big(1 - \frac{n!}{{(\frac{b}{a}+1)}_n}\Big) = \frac{1}{\log(\beta)} \cdot \frac{n!}{{\big(\frac{\log(\beta)}{\log(q)}+1\big)}_n} - \frac{1}{\log(\beta)}.
\]


\section{Eksperymenty}

W modelu SIR symulować będziemy rozkłady zmiennych $W,Z$ oraz ich wartości oczekiwane.
Ustalmy $p=0.2$ oraz $\alpha=0.05$.
Do przeprowadzenia eksperymentu używamy \cref{algo:SIR}.
Rozkłady wyznaczymy na na grafach o $n=100$ wierzchołkach natomiast wartości oczekiwane dla $n\in\{1,2,\dots,300\}$.

Jeśli chodzi o rodzinę $\mathrm{P}_n$ to rozkład $W$ pokrywa się idealnie z przewidywaniami.
Widzimy nawet kumulację masy w ogonie.
Wartości oczekiwane również zbiegają zgodnie z wyznaczonym wzorem.
Podobnie mamy dla zmiennej losowej $Z$.

Dla grafów gwiazd rozkłady zarówno $W$ jak i $Z$ są dobrze przybliżone.
Wartości oczekiwane $W$ są niemal idealnie dopasowane.
Możemy zauważyć dobre przybliżenie na $\mathbb{E}[W]$.


\begin{algorithm}
\caption{Propagacja SIR}
\begin{algorithmic}[1]
\State\textbf{Input:} Graf $G=(V,E)$, prawdopodobieństwo infekcji $p$, prawdopodobieństwo wyzdrowienia $\alpha$, źródło $s\in V$
\State\textbf{Output:} Zbiór wyzdrowiałych wierzchołków $(\mathcal{R}_t)$, czas wymarcia infekcji $Z$
\State$\mathcal{S}_0 \gets V\setminus\{s\}$
\State$\mathcal{I}_0 \gets \{s\}$
\State$\mathcal{R}_0 \gets \varnothing$
\State$t \gets 0$
\While{$\mathcal{I}_t \ne \varnothing$ \textbf{and} $\exists v\in\mathcal{I}_t: \mathcal{S}_t\cap\mathrm{N}(v)\neq\varnothing$}
    \State$\mathcal{I}' \gets \varnothing$
    \For{\textbf{each} $u \in \mathcal{I}_t$}
        \For{\textbf{each} $v \in \mathrm{N}(u)$}
            \If{$v \in \mathcal{S}_t$ \textbf{and} $\text{random}() < p$}
                \State$\mathcal{I}' \gets \mathcal{I}' \cup \{v\}$
            \EndIf%
        \EndFor%
    \EndFor%
    \State$\mathcal{R}' \gets \varnothing$
    \For{\textbf{each} $u \in \mathcal{I}_t$}
        \If{$\text{random}() < \alpha$}
            \State$\mathcal{R}' \gets \mathcal{R}' \cup \{u\}$
        \EndIf%
    \EndFor%
    \State$\mathcal{S}_{t+1} \gets \mathcal{S}_t \setminus \mathcal{I}'$
    \State$\mathcal{I}_{t+1} \gets (\mathcal{I}_t \cup \mathcal{I}') \setminus \mathcal{R}'$
    \State$\mathcal{R}_{t+1} \gets \mathcal{R}_t \cup \mathcal{R}'$
    \State$t \gets t + 1$
\EndWhile%
\State$\mathcal{R}_t \gets \mathcal{R}_t \cup \mathcal{I}_t$
\State$Z \gets t$
\State\Return$(\mathcal{R}_t),\, Z$
\end{algorithmic}%
\label{algo:SIR}
\end{algorithm}

\begin{figure}[ht!]
    \centering
    \begin{subfigure}{0.48\textwidth}
        \centering
        \includegraphics[width=\textwidth]{../img/SIR/path/W_dist.png}
        \caption{Rozkład zmiennej $W$.}
    \end{subfigure}
    \hfill
    \begin{subfigure}{0.48\textwidth}
        \centering
        \includegraphics[width=\textwidth]{../img/SIR/path/W_expectation.png}
        \caption{Wartość oczekiwana $W$.}
    \end{subfigure}
    \caption{Rozkład i wartość oczekiwana zmiennej $W$ dla $\mathrm{P}_n$.}%
    \label{fig:SIR_path_W}
\end{figure}

\begin{figure}[ht!]
    \centering
    \begin{subfigure}{0.48\textwidth}
        \centering
        \includegraphics[width=\textwidth]{../img/SIR/path/Z_dist.png}
        \caption{Rozkład zmiennej $Z$.}
    \end{subfigure}
    \hfill
    \begin{subfigure}{0.48\textwidth}
        \centering
        \includegraphics[width=\textwidth]{../img/SIR/path/Z_expectation.png}
        \caption{Wartość oczekiwana $Z$.}
    \end{subfigure}
    \caption{Rozkład i wartość oczekiwana zmiennej $Z$ dla $\mathrm{P}_n$.}%
    \label{fig:SIR_path_Z}
\end{figure}

\begin{figure}[ht!]
    \centering
    \begin{subfigure}{0.48\textwidth}
        \centering
        \includegraphics[width=\textwidth]{../img/SIR/star/W_dist.png}
        \caption{Rozkład zmiennej $W$.}
    \end{subfigure}
    \hfill
    \begin{subfigure}{0.48\textwidth}
        \centering
        \includegraphics[width=\textwidth]{../img/SIR/star/W_expectation.png}
        \caption{Wartość oczekiwana $W$.}
    \end{subfigure}
    \caption{Rozkład i wartość oczekiwana zmiennej $W$ dla $\mathrm{S}_n$.}%
    \label{fig:SIR_star_W}
\end{figure}

\begin{figure}[ht!]
    \centering
    \begin{subfigure}{0.48\textwidth}
        \centering
        \includegraphics[width=\textwidth]{../img/SIR/star/Z_dist.png}
        \caption{Rozkład zmiennej $Z$.}
    \end{subfigure}
    \hfill
    \begin{subfigure}{0.48\textwidth}
        \centering
        \includegraphics[width=\textwidth]{../img/SIR/star/Z_expectation.png}
        \caption{Wartość oczekiwana $Z$.}
    \end{subfigure}
    \caption{Rozkład i wartość oczekiwana zmiennej $Z$ dla $\mathrm{S}_n$.}%
    \label{fig:SIR_star_Z}
\end{figure}

\clearpage

\chapter{Analiza modelu SIS}
\section{Dwa wierzchołki, jedna krawędź}

W celu oswojenia się z bardziej skomplikowanym modelem, jakim jest \textbf{SIS}, przeanalizujmy graf $\mathrm{K}_2$. 
Niech $V=\{u,v\}$ oraz niech $u$ będzie wierzchołkiem startowym. 
Istnieją cztery możliwe stany systemu: $\mathcal{I}_t=\varnothing, \; \mathcal{I}_t=\{u\}, \; \mathcal{I}_t=\{v\}, \; \mathcal{I}_t=\{u,v\}$. 
Stan, w którym żaden wierzchołek nie jest zainfekowany, jest stanem absorbującym. 
Ponadto, z każdego pozostałego stanu możemy przejść do dowolnego innego. 
Oczywiście $X_u=0$. 
Pierwsza runda jest identyczna jak w modelu \textbf{SI}, a więc $\mathbb{P}[X_v=1]=p$. 
Jeżeli węzeł $v$ nie zostanie poinformowany w rundzie pierwszej, a propagacja nie wygaśnie, to sytuacja się powtórzy. 
Zachodzi to z prawdopodobieństwem $q\beta$. Aby $X_v=t$, potrzebujemy, by ten cykl nastąpił $t-1$ razy. 
Widzimy zatem, że $X_v$ ma rozkład umierający geometryczny, 
\[
    X_v\sim \mathrm{KGeo}(p,\alpha).
\]
A zatem
\[
    \mathbb{P}[X_v=t]=p{(q\beta)}^{t-1}, \quad t\ge 1.
\]
oraz
\[
    \mathbb{P}[X_v < \infty ] = \frac{p}{1-q\beta}, \quad \mathbb{P}[X_v=\infty] =\frac{q\alpha}{1-q\beta}.
\] 
Z założeń modelu $q,\alpha\in(0;1)$, a zatem $\mathbb{P}[X_v=\infty]\ne 0$.  
Mamy $\mathbb{E}[X_v] = \infty$. 
Powinniśmy się więc spodziewać, że niezależnie od wartości parametrów $p$ oraz $\alpha$ infekcja się nie rozprzestrzeni. 
Nie jest to jednak pożądany rezultat. 
Wartość oczekiwaną warunkowa wynosi:
\[
    \mathbb{E}[X_v| X_v < \infty] = \frac{1}{1-q\beta}.
\]

Następnie spróbujmy wyznaczyć rozkład $Y_t$. 
Zauważmy, że $Y_t\in\{0,1,2\}$. 
Prawdopodobieństwa przejść są następujące:
\begin{equation*}
\begin{aligned}
\mathbb{P}[Y_{t+1} = 0 | Y_t = 0 ] &= 1,\\
\mathbb{P}[Y_{t+1} = 0 | Y_t = 1 ] &= q\alpha,\\
\mathbb{P}[Y_{t+1} = 0 | Y_t = 2 ] &= \alpha^2,\\
\mathbb{P}[Y_{t+1} = 1 | Y_t = 0 ] &= 0,\\
\mathbb{P}[Y_{t+1} = 1 | Y_t = 1 ] &= p\alpha+q\beta,\\
\mathbb{P}[Y_{t+1} = 1 | Y_t = 2 ] &= 2\alpha\beta,\\
\mathbb{P}[Y_{t+1} = 2 | Y_t = 0 ] &= 0,\\
\mathbb{P}[Y_{t+1} = 2 | Y_t = 1 ] &= p\beta,\\
\mathbb{P}[Y_{t+1} = 2 | Y_t = 2 ] &= \beta^2.
\end{aligned}
\end{equation*}
Z wzoru na prawdopodobieństwo całkowite dla $k\in \{0,1,2\}$ mamy:
\[
    \mathbb{P}[Y_{t+1} = k] = \sum_{j=0}^{2} \mathbb{P}[Y_{t+1}=k | Y_t=j]\cdot \mathbb{P}[Y_t=j].
\]
Oznaczmy $a_t=\mathbb{P}[Y_t=0], \; b_t=\mathbb{P}[Y_t=1], \; c_t=\mathbb{P}[Y_t=2]$. 
Oczywiście $a_0=0, \; b_0=1, \; c_0=0$. 
Stąd otrzymujemy układ równań rekurencyjnych:
\[
\begin{cases}
    a_{t+1} = a_t + q\alpha\cdot b_t + \alpha^2 \cdot c_t\\
    b_{t+1} = (p\alpha+q\beta)\cdot b_t + 2\alpha\beta\cdot c_t\\
    c_{t+1} = p\beta\cdot b_t + \beta^2\cdot c_t.
\end{cases}
\]
Połóżmy
\[
\mathbf{P} = 
\begin{bmatrix}
1 & q\alpha & \alpha^2 \\
0 & p\alpha+q\beta  & 2\alpha\beta \\
0 & p\beta & \beta^2
\end{bmatrix}, \quad \mathbf{y}_t = \begin{bmatrix}
a_t \\
b_t \\
c_t
\end{bmatrix}.
\]
Wtedy $\mathbf{y}_{t+1}=\mathbf{P}\mathbf{y}_t$, a więc $\mathbf{y}_{t}=\mathbf{P}^t\mathbf{y}_0$. 
Jeśli chodzi o wartość oczekiwaną, to mamy $\mathbb{E}[Y_t]=0\cdot a_t + 1\cdot b_t + 2\cdot c_t = b_t+2c_t$. 
Z \Cref{theorem:infection_dies_out} mamy
\[
    \lim_{t\to\infty} \mathbb{E}[Y_t] = 0.
\]

Przyjrzyjmy się teraz zmiennej $Z$. 
Mamy $\mathbb{P}[Z>t]=\mathbb{P}[Y_t\ne 0] = b_t + c_t$. Oznaczmy
\[
\mathbf{Q} = 
\begin{bmatrix}
p\alpha+q\beta  & 2\alpha\beta \\p\beta & \beta^2
\end{bmatrix}, 
\quad \mathbf{z}_t = \begin{bmatrix}
b_t \\
c_t
\end{bmatrix},
\quad \mathbf{1} = \begin{bmatrix}
1 \\
1
\end{bmatrix}.
\]
Wtedy $\mathbb{P}[Z>t] = \mathbf{1}^\top \mathbf{z}_t = \mathbf{1}^\top \mathbf{Q}^t \mathbf{z}_0$. 
Mamy więc
\[
    \mathbb{E}[Z] = \sum_{t=0}^{\infty} \mathbb{P}[Z>t] = \sum_{t=0}^{\infty} \mathbf{1}^\top\mathbf{Q}^t \mathbf{z}_0 = \mathbf{1}^\top{(\mathbf{I}-\mathbf{Q})}^{-1} \mathbf{z}_0.
\]
Dalej:
\[
    {(\mathbf{I} - \mathbf{Q})}^{-1} = 
    \begin{bmatrix}
    1 - p\alpha-q\beta & -2\alpha\beta \\ -p\beta & 1 -\beta^2
    \end{bmatrix}^{-1}
    =   \frac{1}{\Delta}  
    \begin{bmatrix}
    1 -\beta^2 & 2\alpha\beta \\ p\beta & 1 - p\alpha-q\beta 
    \end{bmatrix},
\]
oraz
\[
    \mathbf{1}^\top{(\mathbf{I}-\mathbf{Q})}^{-1} \mathbf{z}_0 = \mathbf{1}^\top \frac{1}{\Delta} \begin{bmatrix}
        1 -\beta^2 \\
        p\beta
    \end{bmatrix}
    = \frac{1-\beta^2+p\beta}{\Delta},
\]
gdzie $\Delta=\det(\mathbf{I} - \mathbf{Q})$. 
Obliczmy teraz ten wyznacznik:
\begin{equation*}
\begin{aligned}
\Delta &=(1 - p\alpha-q\beta)(1 -\beta^2) - (-2\alpha\beta)(-p\beta) \\
&= 1-p\alpha-q\beta-\beta^2+p\alpha\beta^2+q\beta^3-2p\alpha\beta \\
&=(1-\beta^2)(1-q\beta)-p\alpha(1+\beta^2)\\
&= (1-{(1-\alpha)}^2)(1-(1-p)(1-\alpha))-p\alpha(1+{(1-\alpha)}^2) \\
&=\alpha(2-\alpha)(p+\alpha-p\alpha)-p\alpha(2-2\alpha+\alpha^2) \\
&=\alpha(2p+2\alpha-2p\alpha-\alpha p-\alpha^2+p\alpha^2-2p+2\alpha p-p\alpha^2) \\
&=\alpha^2(2-p-\alpha)=\alpha^2(q+\beta).
\end{aligned}
\end{equation*}
Ostatecznie otrzymujemy:
\[
    \mathbb{E}[Z] = \frac{1-\beta^2+p\beta}{\alpha^2(q+\beta)}.
\]


\section{Pewne wygaśnięcie}

W powyższych obliczeniach mogliśmy zauważyć istotną różnicę pomiędzy modelem \textbf{SI} a \textbf{SIS}.\@
Mianowicie jest niezerowe prawdopodobieństwo, że wierzchołki w grafie nigdy nie zostaną zainfekowane, a oczekiwana liczba zarażonych dąży wraz z upływem czasu do zera.
Dla tak małego grafu jakim jest $\mathrm{K}_2$ jest to dość niespodziewany rezultat. 
Postaramy się teraz udowodnić tą obserwację dla dowolnego grafu.

\begin{theorem}\label{theorem:infection_dies_out}
Niech $G=(V,E)$ będzie grafem spójnym, $s\in V$ źródłem infekcji oraz $v\in V\setminus\{s\}$.
Wtedy zachodzą następujące tożsamości:
\[
    \mathbb{P}[X_v = \infty] > 0, \quad \lim_{t\to\infty} \mathbb{E}[Y_t] = 0, \quad \mathbb{P}[Z < \infty] = 1.
\]
\end{theorem}

\begin{proof}
Weźmy $v\in V\setminus\{s\}$. 
Jeśli $\{s,v\}\notin E$ to jedną z możliwości, która sprawi, że $X_v=\infty$, to wygaśnięcie infekcji po pierwszej jednostce czasu. 
Zatem $\mathbb{P}[X_v=\infty]\ge \alpha$.
Gdy zaś $\{s,v\}\in E$ to $\mathbb{P}[X_v=\infty]\ge q\alpha$, bo musimy jeszcze zagwarantować, że $v$ nie zostanie zainfekowany zanim infekcja w $s$ wygaśnie.
Skoro $q,\alpha>0$ to otrzymujemy $\mathbb{P}[X_v = \infty] > 0$.

Rozważmy propagację jako łańcuch Markova, gdzie stanem w czasie $t$ jest $\mathcal{I}_t$. 
Prawdopodobieństwo przejścia do stanu bez infekcji wynosi
\[
    r(\mathcal{I}_t)=\alpha^{|\mathcal{I}_t|}\prod_{u\in \mathcal{I}_t} q^{|\mathrm{N}(u)\cap\mathcal{S}_t|}.
\]
Istotne jest to, że jest to dodatnia liczba. 
Kładziemy
\[
    r^*=\min\{r(\mathcal{I}_t):\mathcal{I}_t\subseteq V\}.
\]
Prawdopodobieństwo przetrwania infekcji po $k$ rundach wynosi ${(1-r^*)}^k$.
Ale skoro $r^*>0$ to wartość ta dąży do $0$ wraz z $k\to \infty$.
Mamy więc $\mathbb{P}[Z=\infty] = 0$ co jest równoważne z $\mathbb{P}[Z<\infty]=1$.

Skoro infekcja prawie na pewno wymrze to $\mathbf{1}_{Y_t} \xrightarrow{\text{a.s}} 0$ a więc i $Y_t \xrightarrow{\text{a.s}} 0$.
Zauważmy, że $0\le Y_t \le |V|$ dla każdego $t$.
Zatem z Twierdzenia Lebesgue'a o zbieżności ograniczonej otrzymujemy
\[
    \lim_{t\to\infty} \mathbb{E}[Y_t] = \mathbb{E}[\lim_{t\to\infty} Y_t] = \mathbb{E}[0]=0.
\]
\end{proof}


\section{Grafy pełne}

Po analizie dla $\mathrm{K}_2$ postarajmy się ją uogólnić dla $\mathrm{K}_n$.
Ze względu na symetrie dowolny stan z taką samą liczbą zainfekowanych wierzchołków jest izomorficzny.
Z \Cref{theorem:infection_dies_out} wiemy, że $\mathbb{E}[Y_t]\to 0$.
Ustalmy $t\in\mathbb{N}$.
Niech $A_t$ będzie zmienną losową oznaczająca liczbę wierzchołków, które przetrwają rundę $t$ oraz $B_t$ liczbę nowo zainfekowanych wierzchołków w tej rundzie.
Formalnie 
\[
    A_t=|\{v\in V: v \in \mathcal{S}_t \cap \mathcal{S}_{t+1}\}|, \quad B_t=|\{v\in V: v\in\mathcal{S}_t\cap \mathcal{I}_{t+1}\}|.
\]
Oznaczmy $i=Y_t$ oraz $j=Y_{t+1}$. 
Każdy zarażony wierzchołek może przetrwać niezależnie od siebie w formacie próby Bernoulliego.
A więc $A_t\sim \mathrm{Bin}(i,\beta)$.
Dalej każdy podatny wierzchołek może zostać zarażony niezależnie przez $n-i$ sąsiadów.
Szansa, że któremuś z nich się uda wynosi $1-q^i$.
Stąd $B_t\sim\mathrm{Bin}(n-i,1-q^i)$.
Ponadto $Y_{t+1}=A_t+B_t$.
Możemy policzyć warunkową liczbę oczekiwanych zarażeń w $t+1$ kroku:
\[
    \mathbb{E}[Y_{t+1}|Y_t=i]=\mathbb{E}[A_t+B_t]=i\beta+(n-i)(1-q^i).
\]
Oznaczmy $P_{i\to j}=\mathbb{P}[Y_{t+1}=j|Y_t=i]$ dla $j\in\{0,1,\dots,n\}$.
Wtedy korzystając z \Cref{fact:sum_of_bin_RV_2} mamy
\[
    P_{i\to j} = \sum_{\ell=\max\{0,j-(n-i)\}}^{\min\{i,j\}} \binom{i}{\ell}\binom{n-i}{j-\ell}\beta^\ell\alpha^{i-\ell}(1-q^i)^{j-\ell}(q^i)^{n-i-(j-\ell)}.
\]
Jest to uogólnienie prawdopodobieństw przejść, które wyznaczyliśmy wcześniej dla $n=2$.
Ponownie z prawdopodobieństwa całkowitego dla $j\in\{0,1,\dots,n\}$ mamy
\[
    \mathbb{P}[Y_{t+1}=j]=\sum_{i=0}^{n}\mathbb{P}[Y_{t+1}=j|Y_t=i]\cdot\mathbb{P}[Y_t=i].
\]
Zdefiniujmy wektor
\[
    \mathbf{y}_t= \begin{bmatrix}
        \mathbb{P}[Y_t=0] \\
        \mathbb{P}[Y_t=1] \\
         \vdots \\ 
        \mathbb{P}[Y_t=n] \\
    \end{bmatrix}
\]
Wtedy $\mathbf{y}_t^{(k)}=\mathbb{P}[Y_t=k]$ dla $k\in\{0,1,\dots,n\}$.
W uproszczonej notacji możemy zapisać
\[
    \mathbf{y}_{t+1}^{(j)}=\sum_{i=0}^{n} P_{i\to j} \cdot \mathbf{y}_t^{(i)}.
\]
Macierz przejść dla naszego łańcucha Markova dana jest
\[
    \mathbf{P}=\begin{bmatrix} P_{i\to j}\end{bmatrix}_{0\le i,j\le n}.
\]
Mamy $\mathbf{y}_{t+1}=\mathbf{P}\mathbf{y}_t$ jak i $\mathbf{y}_t=\mathbf{P}^t\mathbf{y}_0$.
Zbudowanie tej macierzy jest możliwe w czasie $\mathcal{O}(n^3)$.

Jeśli chodzi o zmienną $Z$ to mamy $\mathbb{P}[Z>t]=\mathbb{P}[Y_t\ne 0]$.
Kładziemy
\[
    \mathbf{Q}=\begin{bmatrix}P_{i\to j}\end{bmatrix}_{1\le i,j\le n}, \quad \mathbf{z}_t=\begin{bmatrix}\mathbb{P}[Y_t]=k\end{bmatrix}_{1\le k \le n}, \quad \mathbf{1} = \begin{bmatrix} 1 \end{bmatrix}_{1\le k \le n}.
\]
Licząc wartość oczekiwaną dostajemy
\[
    \mathbb{E}[Z] = \sum_{t=0}^{\infty} \mathbb{P}[Z>t] = \sum_{t=0}^{\infty} \mathbf{1}^\top\mathbf{z}_t= \sum_{t=0}^{\infty} \mathbf{1}^\top\mathbf{Q}^t \mathbf{z}_0 = \mathbf{1}^\top{(\mathbf{I}-\mathbf{Q})}^{-1} \mathbf{z}_0.
\]

\chapter{Podsumowanie}
Celem pracy było wyznaczenie rozkładów i wartości oczekiwanych zmiennych $X_v$, $Y_t$, $Z$, dla modeli SI, SIR, SIS oraz zmiennej $W$ dla modelu SIR, dla konkretnych rodzin grafów.
W większości rozważanych przypadków udało się znaleźć dokładne wzory na szukane wartości.
Jednakże dla niektórych z nich wyznaczyliśmy przybliżenia lub asymptotykę.
Ponadto udowodniliśmy pewność całkowitej infekcji lub jej wymarcia adekwatnie do modelu.
W modelu SI znaleźliśmy też uniwersalne ograniczenie na czas całkowitej propagacji.
Symulacje w Pythonie metodą Monte Carlo potwierdziły poprawność wyników.
Przeprowadzona analiza pomogła zrozumieć wpływ topologii grafu na szybkość, skuteczność i zasięg rozchodzenia się propagacji.

Głównym problemem teoretycznej analizy było występowanie cykli w grafie, co już w modelu SI sprawiło znaczący problem.
W dalszych badaniach trzeba wziąć pod uwagę, że w rodzinach grafów zawierających sporo cykli wyznaczenie dokładnych rozkładów prawdopodobieństwa zapewne nie będzie możliwe.
W symulacji modelu SIS również kłopotliwy był dobór wielkości parametrów $p$ i $\alpha$ w taki sposób, żeby zapewnić wymarcie infekcji w sensownym czasie.

Kontynuacja pracy może obejmować rozpatrzenie bardziej realistycznych rodzajów sieci takich jak np. grafy Erdősa-Rényiego.
Można także rozważyć uzależnienie parametrów propagacji od wierzchołków i krawędzi grafu, tj. $p_e$ dla $e\in E$ oraz $\alpha_v$ dla $v\in V$.
Na myśl nasuwa się także możliwość zbadania bardziej skomplikowanych modeli epidemiologicznych, takich jak SEIR (Susceptible-Exposed-Infected-Recovered).


Podsumowując, modele probabilistycznej propagacji w grafach stanowią ogromne pole badawcze, które łączy ze sobą rachunek prawdopodobieństwa, teorie grafów i procesy stochastyczne. 
Niniejsza praca omawia podstawowe metody analizy propagacji i prezentuje wnioski uzyskane w toku badania.

\nocite{*}
\printbibliography[heading=bibintoc,title={Bibliografia}]

\end{document}